\documentclass{article}
\usepackage[a4paper, margin=0.5in]{geometry}
\usepackage{amsmath}
\usepackage{amsfonts}
\usepackage{tabularx}
\usepackage{tikz}
\usepackage{xcolor}
\setlength{\tabcolsep}{10pt}
\renewcommand{\arraystretch}{2}
\title{That Weird Trigonometry}
\author{A list of some weird but seemingly useful trigonometric identities \& charts}
\date{\today}
\begin{document}

\maketitle
\section{Angle conversions}
\begin{itemize}
	\item $1^\circ = 60'$ (minute) $= 3600''$ (seconds)
	\item Degree to radian: $x^\circ = \dfrac{\pi x}{180}$ rad
	\item Radian to degree: $x$ rad $ = \left(\dfrac{180x}{\pi}\right)^\circ$
\end{itemize}

\maketitle
\section{Basic Identities}
\begin{itemize}
\item $\tan(\theta)=\dfrac{\sin(\theta)}{\cos(\theta)}=\dfrac{1}{\cot(\theta)}$
\item $\cot(\theta)=\dfrac{\cos(\theta)}{\sin(\theta)}=\dfrac{1}{\tan(\theta)}$
\item $\sec(\theta)=\dfrac{1}{\cos(\theta)}$
\item cosec$(\theta) = \csc(\theta)=\dfrac{1}{\sin(\theta)}$

\end{itemize}

\maketitle
\section{Trigonometric table}
\begin{tabularx}{1\textwidth} { 
  | *9{>{\centering\arraybackslash}X|}
}
\hline Degree & $0^\circ$   & $30^\circ$            & $45^\circ$            & $60^\circ$            & $90^\circ$       & $180^\circ$ & $270^\circ$        & $360^\circ$ \\
\hline Radian & 0           & $\dfrac{\pi}{6}$      & $\dfrac{\pi}{4}$      & $\dfrac{\pi}{3}$      & $\dfrac{\pi}{2}$ & $\pi$       & $\dfrac{3\pi}{2}$  & $2\pi$ \\[0.2cm]
\hline sin    & 0           & $\dfrac{1}{2}$        & $\dfrac{1}{\sqrt{2}}$ & $\dfrac{\sqrt{3}}{2}$ & 1                & 0           & $-1$               & 0 \\[0.2cm]
\hline cos    & 1           & $\dfrac{\sqrt{3}}{2}$ & $\dfrac{1}{\sqrt{2}}$ & $\dfrac{1}{2}$        & 0                & $-1$        & 0                  & 1 \\[0.2cm]
\hline tan    & 0           & $\dfrac{1}{\sqrt{3}}$ & 1                     & $\sqrt{3}$            & Not defined      & 0           & Not defined        & 0 \\[0.2cm]
\hline cosec  & Not defined & 2                     & $\sqrt{2}$            & $\dfrac{2}{\sqrt{3}}$ & 1                & Not defined & $-1$               & Not defined \\[0.2cm]
\hline sec    & 1           & $\dfrac{2}{\sqrt{3}}$ & $\sqrt{2}$            & 2                     & Not defined      & $-1$        & Not defined        & 1 \\[0.2cm]
\hline cot    & Not defined & $\sqrt{3}$            & 1                     & $\dfrac{1}{\sqrt{3}}$ & 0                & Not defined & 0                  & Not defined \\[0.2cm]
\hline
\end{tabularx}
\maketitle
\section{Pythagorian Relations}
\begin{itemize}
\item $\sin^2(\theta)+\cos^2(\theta) = 1$
\item $\sec^2(\theta)-\tan^2(\theta) = 1$
\item $\csc^2(\theta)-\cot^2(\theta) = 1$
\end{itemize}

\maketitle
\section{Sign of trigonometric functions for negative angles}

\begin{itemize}
\item $\sin(-x) = -\sin(x)$
\item $\cos(-x) = \cos(x)$
\item $\tan(-x) = -\tan(x)$
\item $\csc(-x) = -\csc(x)$
\item $\sec(-x) = \sec(x)$
\item $\cot(-x) = -\cot(x)$
\end{itemize}

\maketitle
\section{Expansion for trigonometric functions with two angles}

\begin{itemize}
\item $\cos(x+y)=\cos(x)\cos(y)-\sin(x)\sin(y)$
\item $\cos(x-y)=\cos(x)\cos(y)+\sin(x)\sin(y)$
\item $\sin(x+y)=\sin(x)\cos(y)+\cos(x)\sin(y)$
\item $\sin(x-y)=\sin(x)\cos(y)-\cos(x)\sin(y)$
\item $\tan(x+y)=\dfrac{\tan(x)+\tan(y)}{1-\tan(x)\tan(y)}$
\item $\tan(x-y)=\dfrac{\tan(x)-\tan(y)}{1+\tan(x)\tan(y)}$
\item $\cot(x+y)=\dfrac{\cot(x)\cot(y)-1}{\cot(x)+\cot(y)}$
\item $\cot(x-y)=\dfrac{\cot(x)\cot(y)+1}{\cot(y)-\cot(x)}$
\item \subsection*{With {\huge $\pi$}}
\begin{enumerate}
  \item $\sin\left(\dfrac{\pi}{2} + \theta\right) = \cos(\theta)$
  \item $\sin\left(\dfrac{\pi}{2} - \theta\right) = \cos(\theta)$
  \item $\sin(\pi - \theta) = \sin(\theta)$
  \item $\sin(\pi + \theta) = -\sin(\theta)$
  \item $\sin(2\pi - \theta) = -\sin(\theta)$
  \item $\cos\left(\dfrac{\pi}{2} + \theta\right) = -\sin(\theta)$
  \item $\cos\left(\dfrac{\pi}{2} - \theta\right) = \sin(\theta)$
  \item $\cos(\pi - \theta) = -\cos(\theta)$
  \item $\cos(\pi + \theta) = -\cos(\theta)$
  \item $\cos(2\pi - \theta) = \cos(\theta)$
\end{enumerate}
\end{itemize}
\vspace{-15.3cm}
\hspace{10cm}
$
\renewcommand{\arraystretch}{1.6}
\color{teal}
 \begin{array}{|l|}
 \hline
 c(x+y) = cc-ss \\
 c(x-y) = cc+ss \\
 s(x+y) = sc+cs \\
 s(x-y) = sc-cs \\
 \hline
 \end{array}^\textbf{*}
$
\vspace{12cm}
\maketitle
\section {Product formula}
\begin{itemize}
  \item $\sin(x + y) + \sin(x - y) = 2\sin(x)\cos(y)$
  \item $\sin(x + y) - \sin(x - y) = 2\cos(x)\sin(y)$
  \item $\cos(x + y) + \cos(x - y) = 2\cos(x)\cos(y)$
  \item $\cos(x + y) - \cos(x - y) = -2\sin(x)\sin(y)$
\end{itemize}
\vspace{-3cm}
\hspace{10cm}
$
\color{teal}
 \renewcommand{\arraystretch}{1.6}
  \begin{array}{|l|}
  \hline
  s+s = 2sc \\
  s-s = 2cs \\
  c+c = 2cc \\
  c-c = -2ss \\
  \hline
 \end{array}^\textbf{*}
$

\maketitle
\section {Sum formula}
\begin{itemize}
  \item $\sin(a) + \sin(b) = 2\sin\left(\dfrac{a + b}{2}\right)\cos\left(\dfrac{a - b}{2}\right)$
  \item $\sin(a) - \sin(b) = 2\cos\left(\dfrac{a + b}{2}\right)\sin\left(\dfrac{a - b}{2}\right)$
  \item $\cos(a) + \cos(b) = 2\cos\left(\dfrac{a + b}{2}\right)\cos\left(\dfrac{a - b}{2}\right)$
  \item $\cos(a) - \cos(b) = -2\sin\left(\dfrac{a + b}{2}\right)\sin\left(\dfrac{a - b}{2}\right)$
\end{itemize}
\vspace{-5cm}
\hspace{10cm}
$
 \renewcommand{\arraystretch}{2.7}
 \color{teal}
 \begin{array}{|l|}
 \hline
  s+s = 2sc \\
  s-s = 2cs \\
  c+c = 2cc \\
  c-c = -2ss \\
 \hline
 \end{array}^\textbf{*}
$
\maketitle
\section {Expansion for multiple angles}

\begin{itemize}
  \item $\sin(2x)\ =\ 2\sin(x)\cos(x)\ =\ \dfrac{2\tan(x)}{1 + tan^2(x)}$
  \item $\sin(3x) = 3\sin(x) - 4\sin^3(x)$
  \item $\cos(2x)\ =\ \cos^2(x) - \sin^2(x)\ =\ 2\cos^2(x)-1\ =\ 1-2\sin^2(x)\ =\ \dfrac{1 - \tan^2(x)}{1 + \tan^2(x)}$
  \item $\cos(3x) = 4\cos^3(x) - 3\cos(x)$
  \item $\tan(2x) = \dfrac{2\tan(x)}{1 - \tan^2(x)}$
  \item $\tan(3x) = \dfrac{3\tan(x) - \tan^3(x)}{1 - 3\tan^2(x)}$
  \item $\sin^2(x) = \dfrac{1 - \cos(2x)}{2}$
  \item $\cos^2(x) = \dfrac{1 + \sin(2x)}{2}$  
\end{itemize}

\maketitle
\section {Law of sines}
\hspace*{12cm}
\vspace*{1cm}
\begin{tikzpicture}
  \draw (0,0) node[anchor=north east] {B} -- (3, 0) node[anchor=north] {a} -- (6,0)node[anchor=west] {C} -- (4.5,1.5) node[anchor=south west] {b} -- (3,3)node[anchor=south] {A} -- (1.5,1.5)node[anchor=south] {c} -- cycle;
\end{tikzpicture}
\vspace*{-5cm}
\par In any triangle, sides are proportional to to the sins of the angles\\
\begin{align}
  \hspace{-7cm}
\dfrac{\sin(A)}{a} = \dfrac{\sin(B)}{b} = \dfrac{\sin(C)}{c}
\hspace{1cm}or\hspace{1cm}
\dfrac{a}{\sin(A)} = \dfrac{b}{\sin(B)} = \dfrac{c}{\sin(C)}
\end{align}
\par From above equations we also get:
\par \begin{itemize}
	\item $\dfrac{a - b}{c} = \dfrac{\sin\left(\dfrac{A-B}{2}\right)}{\cos\left(\dfrac{C}{2}\right)}$
	\item $\sin\left(\dfrac{B-C}{2}\right) = \dfrac{b-c}{c}\cos\left(\dfrac{A}{2}\right)$
	\item $\tan\left(\dfrac{A-B}{2}\right) = \dfrac{a-b}{a+b}\cot\left(\dfrac{C}{2}\right)$
\end{itemize}
\vspace*{1cm}

\maketitle
\section {Law of cosines}
\hspace*{12cm}
\vspace*{1cm}
\begin{tikzpicture}
  \draw (0,0) node[anchor=north east] {B} -- (2.5, 0) node[anchor=north] {a} -- (4,0)node[anchor=west] {C} -- (3.25,1.25) node[anchor=south west] {b} -- (2.5,2.5)node[anchor=south] {A} -- (1.25,1.25)node[anchor=south] {c} -- cycle;
\end{tikzpicture}
\vspace*{-4cm}
\par
For any $\Delta$ABC \\
$$\hspace{-10cm} a^2 = b^2 + c^2 -bc\cos(A)$$
$$\hspace{-10cm} b^2 = a^2 + c^2 -ac\cos(B)$$
$$\hspace{-10cm} c^2 = a^2 + b^2 -ab\cos(C)$$ 
\vspace{.1cm}
\par also \\
$$\hspace{-10cm} \cos(A) = \dfrac{b^2 + c^2 - a^2}{2bc}$$
$$\hspace{-10cm} \cos(B) = \dfrac{a^2 + c^2 - b^2}{2ac}$$
$$\hspace{-10cm} \cos(C) = \dfrac{a^2 + b^2 - c^2}{2ab}$$
\par From above equations, we also get:
\vspace*{0.5cm}
$$\hspace{-10cm} \dfrac{\cos(A)}{a} = \dfrac{\cos(B)}{b} = \dfrac{\cos(C)}{c} = \dfrac{a^2 + b^2 + c^2}{2abc}$$

\maketitle
\section{Solutions of some trigonometric equations}
\begin{itemize}
  \item $\sin(x) = 0 \implies x = n\pi,\ n \in \mathbb{Z}$
  \item $\cos(x) = 0 \implies x = (2n + 1)\dfrac{\pi}{2},\ n \in \mathbb{Z}$
  \item $\sin(x) = \sin(y) \implies x = n\pi + (-1)^n y,\ n \in \mathbb{Z}$
  \item $\cos(x) = \cos(y) \implies x = n\pi \pm y,\ n \in \mathbb{Z}$
  \item $\tan(x) = \tan(y) \implies x = n\pi + y,\ n \in \mathbb{Z}$
  \item $\tan(x) = 0 \implies x = n\pi,\ n \in \mathbb{Z}$
\end{itemize}
\maketitle
\section{Other useful stuffs...}
\par These are results obtained from equations above.
\begin{itemize}
	\item $\cos(2n\pi+x) = \cos(x),\hspace{0.2cm}n \in \mathbb{Z}$
	\item $\sin(2n\pi+x) = \sin(x),\hspace{0.2cm}n \in \mathbb{Z}$
  \item $\tan\left(\dfrac{\pi}{4} - x\right) = \dfrac{1 - \tan(x)}{1 + \tan(x)}$
  \item $\tan\left(\dfrac{\pi}{4} + x\right) = \dfrac{1 + \tan(x)}{1 - \tan(x)}$
  \item $\dfrac{\tan\left(\dfrac{\pi}{4} + x\right)}{\tan\left(\dfrac{\pi}{4} - x\right)} = \left[\dfrac{1 + \tan(x)}{1 - \tan(x)}\right]^2$
  \item $\sin(x + y)\sin(x - y) = \sin^2(x) - \sin^2(y)$
  \item $\cos(x + y)\cos(x - y) = \cos^2(x) - \sin^2(y)$
  \item $\tan(3x)\tan(2x)\tan(x) = \tan(3x) - \tan(2x) - \tan(x)$
  \item $\tan(4x) = \dfrac{4\tan(x)(1-\tan^2(x)}{1-6\tan^2(x)+\tan^4(x)}$
  
\end{itemize}

\maketitle
\section{Some \textit{Condensed} trigonometric identities}
\begin{itemize}
\item $\sin(x \pm y)=\sin(x)\cos(y)\pm\cos(x)\sin(y)$
\item $\cos(x \pm y)=\cos(x)\cos(y)\mp\sin(x)\sin(y)$
\item $\tan(x\pm y)=\dfrac{\tan(x)\pm\tan(y)}{1\mp\tan(x)\tan(y)}$
\item $\cot(x\pm y)=\dfrac{\cot(x)\cot(y)\mp 1}{\cot(x)\pm\cot(y)}$
\item $\tan\left(\dfrac{\pi}{4} \pm x\right) = \dfrac{1 \pm \tan(x)}{1 \mp \tan(x)}$
\end{itemize}

\vspace{18cm}
\hline
\color{teal}
\begin{itemize}
	\item *: Table in teal shows some mnemonics to remember the formulae easily
\end{itemize}
\end{document}
