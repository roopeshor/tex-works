\chapter{Details about the Internship}

This internship was a fast-paced, hands-on project focused on tangible outcomes.
The work was structured around the SUAS 2025 competition, demanding robustness in design, execution, and teamwork under the mentorship of AI Aerial Dynamics.
The space for work was provided by KSUM for a period of 2 weeks, and rest of the work was carried out at CUSAT considering the logistics of transportation and ease of work.

\section{Project Overview and Team Structure}

The project involved building a robust quadcopter from the ground up which satisfies certain constraints put forward by the competition.
The basic constraints were following:
\begin{enumerate}
	\item Total weight of drone must be under 20kg
	\item Drone should be able to fly autonomously
	\item Drone must have Flight Termination Failsafes
\end{enumerate}

The team was divided into three core sub-teams: Electronics, Mechanics, and Software.
This division was not rigid, and team members could contribute to any subteam.
I was a member of the Electronics team, with some contributions to the mechanical and software aspects as well.

\section{My Role and Responsibilities}

My responsibilities spanned the full lifecycle of the drone's development, from design and documentation to integration and testing.

\noindent
\begin{enumerate}
	\item \textbf{Electronics Design, Implementation \& Documentation:}
	      \begin{itemize}
		      \item Created list of electronic components required to meet competition criteria and prepare datasheets for them.
		      \item Designed the electronic interconnects for the drone.
		      \item Developed a electronic system for payload mechanism.
	      \end{itemize}
	\item  \textbf{Software Contributions:}
	      \begin{itemize}
		      \item Optimized the real-time video pipeline using ffmpeg tools. Through adjusting parameters, I successfully reduced the end-to-end video latency from approximately 1000ms down to 600ms, which was almost sufficient for our purpose of object detection task.
		      \item Setup Mission Planner software for occassional flight tests.
	      \end{itemize}
	\item \textbf{Mechanical and Design Contributions:}
	      \begin{itemize}
		      \item Helped in the full mechanical assembly of the drone frame and contributed some concepts to the payload mechanism's physical layout.
		      \item Helped in designing a custom-fit protective casing for the various electronic modules.
	      \end{itemize}
\end{enumerate}

\raggedbottom
\chapter{System Architecture and Components}
The drone was built prioritizing robustness, practical availability, future reusage and some transportation factors.
Electronic components of the drone were chosen based on whether they have extensive documentation, community support as well as track record of reliability, making them a good investment for future projects.
A critical, overriding factor in component selection was local availability within India, which was essential to meet our compressed project timeline as well.
Our mentor also helped up picking the right components based on their previous experience with drone manufacturing.

\vspace*{12pt}
The design of electronic interconnect was first done in Figma.
An odd choice, but it was done in Figma to facilitate collaboration among other team members and iterative design.
This allowed all team members to review the architecture in real-time, provide feedback and helped us keep track of the project status.

\vspace*{12pt}
The mechanical componets were designed in Autodesk Fusion 360. The mechanical components included frame and payload mechanism.
Initially we had uncertainity on which mechanism to use for payload deployment.
Hence we created 4 different mechanism and tested them individually. Out of them, we choose winch mechanism.
It was relatively simple and fits our purpose.
In the beginning we tried desinging our own frame, but due to multiple issues (possible manufacturing delay and cost) we decided to buy a ready made frame.
The payload mechanism were 3D printed on 2 different fabs for fallback.

\vspace*{12pt}
Figure \ref{Payload mechanism} shows a section of payload mechanism. There are 4 of such mechanism the drone. 
Figure \ref{A high-level system architecture diagram of drone} shows the detailed block diagram of the final system.
Table \ref{comp} lists the important components used in the drone

\blockimage{payload}{0.5}{Payload mechanism}. 

\begin{longtable}[h]{|l|l|p{5cm}|} \hline
	\textbf{Component}       & \textbf{Model / Type}  & \textbf{Purpose}                                        \\ \hline
	Flight Controller        & CubePilot Cube Orange+ & Central processing unit for flight control and autonomy \\ \hline
	Video/Data Link          & SiYi HM30              & Long-range video and telemetry transmission             \\ \hline
	Camera                   & ADTi Surveyor 24L      & High-resolution aerial mapping and object detection     \\ \hline
	Manual Control/Telemetry & Skydroid T12           & Telemetry and manual flight control                     \\ \hline
	Power Module             & Mauch Power Cube       & Power distribution and battery elimination (BEC)        \\ \hline
	Propulsion System        & Hobbywing X8           & Provided thrust for the quadcopter                      \\ \hline
	Frame                    & EFT E410P              & Provided the structural foundation for the drone        \\ \hline
	\caption{Key Components of the UAV}
	\label{comp}
\end{longtable}



\blocksvg{block}{1}{A high-level system architecture diagram of drone}
