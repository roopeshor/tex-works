\chapter{Overview of the Industry}
The Unmanned Aerial Vehicle (UAV), or drone, industry represents a frontier of modern engineering which blends vast areas of engineering such as electronics, mehcanics and computer science. 
Globally, UAVs have transitioned from niche military hardware to indispensable tools across various sectors, including logistics, infrastructure inspection, precision agriculture, and disaster management.
This rapid commercialization is driven by advancements in flight autonomy, sensor technology, and improved data analysis.
Nowadays drone technology - which was limited hands of a few - has expanded and became accessible to wider community.
Recently, in our country, huge amount of investment is going towards drone technology,
partially in response to growing political tesion as well as need for self reliability.

To make students exposed to this area, various competition, hackathons and sessions are now being conducted.
International competitions like the Student Unmanned Aerial Systems (SUAS) competition play a major role in bringing together talents across the world.
They serve as hub for innovation, knowledge sharing and pushing the boundaries of what is possible and fostering the next generation of engineers.
These events challenge teams to solve complex, real-world problems, creating a high-stakes environment where learning and development happens beyond the standard academic curriculum.
This internship at AI Aerial Dynamics was an immersion into this competitive and innovative world, focusing on the practical application of engineering principles to a specific, challenging goal - representing ourself at the competition - SUAS.
