% \newpage
\chapter{Components}

\section{Soil Moisture Sensor}

\setlength\intextsep{0pt}
\setlength{\columnsep}{.6cm}
\floatfig{r}{.4}{sms.png}{1}{Soil moisture sensor}{.4cm}

The Resistive Soil Moisture Sensor Module consists of two probes
which are used to measure the moisture content of water. Resistive
soil moisture sensors work by measuring the resistance of the soil,
which changes with moisture content, with higher moisture leading to
lower resistance. The sensor uses two probes to pass a current
through the soil, and the resistance encountered is used to estimate
moisture levels. The sensing circuit is calibrated to different soil
types for proper measurement by adjusting the built in potentiometer

\section{FS1000A: Transmitter \& Receiver module}

\floatfig{l}{.45}{fs100a.png}{.9}
{FS1000A receiver \&\\ transmitter pair}{.2cm}

This is a combination of two modules used for data transmission and
reception. The  core of the transmitter module is a SAW resonator
tuned to operate at 433MHz. Apart from that, it has a switching
transistor and some passive components. When the DATA input is high,
the oscillator generates a constant RF output carrier wave at 433MHz,
and when the DATA input is low, the oscillator ceases operation;
resulting in an amplitude modulated wave. The module can cover a
minimum of 10 meters and with proper antenna and power supplies.
Theoretically it can transmit up to 100 meters.

The Receiver module consists of an RF tuned circuit and an Op-Amp
that amplify the received carrier wave. The amplified signal is then
fed into a PLL Phase Lock Loop, which allows the demodulator to
“lock” onto frequency of incoming signal. The demodulated signal is
preset at two output pins which are internally connected
% \newpage

\section{HT12E: Encoder IC}

\floatfigsvg{r}{.45}{ht12e.svg}{.8}{HT12E encoder IC pinout}{-1cm}

The function of HT12E is to encode a 4-bit data and 8-bit address and
send it out through the output pin. The IC has operating voltage
ranging from 2.4V to 5V with typical value of 3V. The Transmission
Enable pin (pin 14) is pulled to ground to activate transmission. The
4-bit data that has to be sent is given to the pins AD8 to AD11 and
an address of 8-bit is set using the pins A0 to A7.  Moreover a
resistor has to be connected between OSC1 and OSC2 pins to set the
operation frequency of encoder

\vspace*{1.5cm}

\section{HT12D: Decoder IC}

\floatfigsvg{r}{.45}{ht12d.svg}{.8}{HT12D decoder IC pinout}{0cm}

The function of HT12D is to decode the data obtained from a receiver
circuit and send it through the output pins. The IC has a wide range
of operating voltage from 2.4V to 12V with typical voltage being 5V.
A signal on the DIN pin activates the oscillator which starts
decoding of the incoming signal and obtains data and address in it.
The decoder will then check if the received address matches with the
preset address three times continuously. If it matches, the 4-bits of
data are decoded to activate the output pins D8-D11 and the VT pin is
set high to indicate a valid transmission.

\section{BC547B: Transistor}

\floatfig{r}{.4}{bc547.png}{.45}{BC547 transistor}{-1cm}

The BC547B, an NPN transistor, can handle current up to 200mA. The
gain current of this transistor is from 110 to 800. The maximum base
to emitter reverse voltage it can handle is 6V and maximum collector
to emitter voltage is 45V. The transistor is available in the TO-92 package.

\vspace*{1cm}

\section{18650 Battery and case}

\floatfig{r}{.45}{battery.png}{.9}{18650 battery and its holder}{-1cm}

An 18650 battery is a cylindrical, lithium-ion rechargeable battery,
named for its dimensions (18mm diameter and 65mm length), known for
its high energy density and long lifespan. Here we have used a 3.7V
18650 battery with 2600mAh capacity. This battery was chosen at the
transmitter circuit as it can power the circuit for months without
recharge in (ideal conditions)

\vspace*{1cm}

\section{JQC-3FC(T73)DC05V: Relay}

\setlength{\columnsep}{.3cm}
\floatfig{r}{.45}{relay.png}{.6}{PCB Mount Relay}{0cm}

This is an electromechanical relay that mechanically switches contact
between two points. The relay is operated by applying a voltage
across the coil pins. When coil gets energized producing a magnetic
field that actuates a contact lever. When the coil is de-energized
the contact returns to normal position. The relay we used can handle
switching of 120V 10A AC supply or 24V 10A DC supply. The relay can
switch at most 60Hz. The coil has to be powered by a 5V supply in
order for it to actuate the contact

\section{DC Motor Pump}

\floatfig{r}{.4}{pump.png}{.6}{DC motor pump}{-1cm}

We used a 6V DC motor pump to supply water to the field. The motor
consumes 4-5W of power. The motor can work from 3V to 6V. It can
supply water up to a height of 110cm at maximum operating voltage.
For the purpose of demonstration, the motor is powered by a regulated
power supply that can supply 5V DC

\vspace*{1cm}

\section{NodeMCU}
The NodeMCU ESP32 is a development board based on the ESP32
microcontroller. It is programmable with various programming
languages. This board has 2.4 GHz dual-mode Wifi and a BT wireless
connection. In addition, a 512 KB SRAM and a 4MB flash memory are
integrated into the microcontroller development board. The board has
21 pins for interface connection, including I2C, SPI, UART, DAC, and
ADC. The NodeMCU's ADC (Analog-to-Digital Converter) input is
configured for the 0-3.3V range with appropriate signal conditioning
to match sensor output characteristics.  The module can be powered
using an external supply through its micro USB port or with 5V and
GND pins in the board
