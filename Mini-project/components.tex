\pagebreak
\section{Components}
The following section describes the specification of various componets choosen for implementing the project. This includes sensors, radio modules, encoder/decoder ICs, Wi-fi module, etc.
\subsection{Soil Moisture Sensor}

\setlength\intextsep{0pt}
\setlength{\columnsep}{.6cm}

The Resistive Soil Moisture Sensor Module consists of two probes which are used to measure the moisture content of soil. A photo of such sensor is given in \fref{Soil moisture sensor}. Resistive soil moisture sensors work by measuring the resistance of the soil, which changes with moisture content, with higher moisture leading to lower resistance. The sensor uses two probes to pass a current through the soil, and the resistance encountered is used to estimate moisture levels. The sensing circuit is calibrated to different soil types for proper measurement by adjusting the built in potentiometer\\[5pt]
\blockimage{sms.png}{.65}{Soil moisture sensor}


\subsection{FS1000A: Transmitter \& Receiver module}

FS1000A is a combination of two modules used for data transmission and reception. The core of the transmitter module is a SAW resonator tuned to operate at 433MHz. Apart from that, it has a switching transistor and some passive components. When the DATA input is high, the oscillator generates a constant RF output carrier wave at 433MHz, and when the DATA input is low, the oscillator ceases operation; resulting in an amplitude modulated wave. The module can cover a minimum of 10 meters and with proper antenna and power supplies. Theoretically it can transmit up to 100 meters\cite{fs1000a}. \fref{FS1000A receiver and transmitter pair} shows a photo of the device.

\blockimage{fs100a.png}{.45}{FS1000A receiver and transmitter pair}

The Receiver module consists of an RF tuned circuit and an Op-Amp that amplify the received carrier wave. The amplified signal is then fed into a PLL Phase Lock Loop, which allows the demodulator to “lock” onto frequency of incoming signal. The demodulated signal is preset at two output pins which are internally connected

\subsection{HT12E: Encoder IC}

The function of HT12E is to encode a 4-bit data and 8-bit address and send it out through the output pin. The IC has operating voltage ranging from 2.4V to 5V with typical value of 3V. The Transmission Enable pin (pin 14) is pulled to ground to activate transmission. The 4-bit data that has to be sent is given to the pins AD8 to AD11 and an address of 8-bit is set using the pins A0 to A7.  Moreover a resistor has to be connected between OSC1 and OSC2 pins to set the operation frequency of encoder. The value of this resistor is chosen according to the oscillator frequency characteristics provided in the datasheet\cite{ht12e}. The pinout of the IC is shown in \fref{HT12E encoder IC pinout}

\vspace{12pt}

\blocksvg{ht12e}{.33}{HT12E encoder IC pinout}

\subsection{HT12D: Decoder IC}

The function of HT12D is to decode the data obtained from a receiver circuit and send it through the output pins. The IC has a wide range of operating voltage from 2.4V to 12V with typical voltage being 5V. A signal on the DIN pin activates the oscillator which starts decoding of the incoming signal and obtains data and address in it. The decoder will then check if the received address matches with the preset address three times continuously. If it matches, the 4-bits of data are decoded to activate the output pins D8-D11 and the VT pin is set high to indicate a valid transmission\cite{ht12d}. The pinout of the IC is shown in \fref{HT12D decoder IC pinout}

\vspace{12pt}
\blocksvg{ht12d}{.33}{HT12D decoder IC pinout}

\subsection{BC547B: Transistor}

% \floatfig{r}{.35}{bc547.png}{.4}{BC547 transistor}{-.5cm}
The BC547B, an NPN transistor, can handle current up to 200mA. The gain current of this transistor is from 110 to 800. The maximum base to emitter reverse voltage it can handle is 6V and maximum collector to emitter voltage is 45V. The transistor is available in the TO-92 package\cite{bc547}. \fref{BC547 transistor} shows a photo of the transistor in TO-92 package. The transistor is employed in the circuit to drive the relay. \\
\blockimage{bc547.png}{.25}{BC547 transistor}

\subsection{JQC-3FC(T73)DC05V: Relay}

% \floatfig{r}{.35}{relay.png}{.7}{PCB Mount Relay}{.5cm}

This is an electromechanical relay that mechanically switches contact between two points. The relay is operated by applying a voltage across the coil pins. When coil gets energized producing a magnetic field that actuates a contact lever. When the coil is de-energized the contact returns to normal position. The relay we used can handle switching of 120V 10A AC supply or 24V 10A DC supply. The relay can switch at most 60Hz. The coil has to be powered by a 5V supply in order for it to actuate the contact\cite{jqc}. The relay is driven by a transistor switch as the ICs used cannot handle the current required to drive the relay. \fref{PCB Mount Relay} shows the relay used in this project.
\blockimage{relay.png}{.27}{PCB Mount Relay}


\subsection{18650 Battery and case}

% \floatfig{r}{.35}{battery.png}{.8}{18650 battery and its holder}{-1cm}
An 18650 battery is a cylindrical, lithium-ion rechargeable battery, named for its dimensions (18mm diameter and 65mm length), known for its high energy density and long lifespan. Here we have used a 3.7V 18650 battery with 2600mAh capacity. \fref{18650 battery and its holder} shows battery and its case. This battery was chosen for the transmitter circuit as it can power the circuit for months without recharge in (ideal conditions)
\vspace*{10pt}
\blockimage{battery.png}{.4}{18650 battery and its holder}

\subsection{DC Motor Pump}

% \floatfig{r}{.35}{pump.png}{.9}{DC motor pump}{0cm}
A 6V DC motor pump is utlilized to supply water to the field. The motor consumes 4-5W of power. The motor can work from 3V to 9V. It can supply water up to a height of 110cm at maximum operating voltage.\cite{pump} For the purpose of demonstration, the motor is powered by a regulated power supply that can supply 5V DC. \fref{DC motor pump} shows a view of the motor.\\
\blockimage{pump.png}{.35}{DC motor pump}

\subsection{NodeMCU Development Board}

The NodeMCU ESP32 is a development board based on the ESP32 microcontroller. It is programmable with various programming languages. This board has 2.4 GHz dual-mode Wi-Fi and a BT wireless connection. In addition, a 512 KB SRAM and a 4MB flash memory are integrated into the microcontroller development board. The board has 21 pins for interface connection, including I2C, SPI, UART, DAC, and ADC. The NodeMCU's ADC (Analog-to-Digital Converter) input is configured for the 0-3.3V range with absolute maximum rating of 3.6V\cite{esp32}. The module can be powered using an external supply through its micro USB port or with 5V and GND pins in the board. A view of NodeMCU used in the project is shown in \fref{NodeMCU ESP32}
\vspace{12pt}

\blockimage{nodemcu.png}{.5}{NodeMCU ESP32}