\section{Tuning circuit parameters}
For the proper working of the circuit some values of components has to be calculated using data given in the datasheet. This section describes those aspects of the design.
\subsection{Encoder IC oscillator}


The encoder is powered by a voltage supply of 3.7V. An arbitrary value of 1MΩ as recommended by the manufacturer is chosen for the resistor across the oscillator pins OSC1 and OSC2 (pins 16 and 15). From the Oscillator Frequency vs Supply Voltage graph of the encoder IC shown in \fref{Plot of Oscillator frequency vs supply voltage for HT12E}, the oscillation frequency corresponding to the oscillator resistor value of 1MΩ and supply voltage of 3.7V is found to be 2.71kHz. This value is crucial for calculating the oscillator resistor for  the decoder IC

\subsection{Decoder IC oscillator}
The recommended oscillator frequency for the decoder is 50 times that of the encoder's frequency. In this case, for and encoder frequency of 2.71kHz, the decoder frequency has to be 135kHz. The decoder is powered by a voltage supply of 5V. From the Oscillator Frequency vs Supply Voltage graph of the decoder shown in \fref{Plot of Oscillator frequency vs supply voltage for HT12D}, it can be seen that the resistor value corresponding to a 5V supply voltage and calculated oscillation frequency (135kHz) is 62kΩ. However  68kΩ was the resistor value available while the construction of prototype which is nearest the calculated value. Upon testing, this slight change in resistance did not produce much deterioration in the communication system. This resistor is connected across the pins OSC1 and OSC2
\vspace*{10pt}

\blocksvg{enc-osc}{.7}{Plot of Oscillator frequency vs supply voltage for HT12E}
\vspace*{36pt}
\blocksvg{dec-osc}{.7}{Plot of Oscillator frequency vs supply voltage for HT12D}

\subsection{Switching circuit}


\blocksvg{switch}{.43}{Switching circuit}
% \begin{figure}[ht!]
% 	\centering
% 	\begin{minipage}[t]{0.43\linewidth}
% 		\addsvg{images/switch}{.6\linewidth}
% 		\captionof{figure}{Switching circuit}
% 	\end{minipage}
%   \hfill
%   \begin{minipage}[t]{0.54\linewidth}
% 		\addsvg{images/bc547.svg}{.9\linewidth}
% 		\caption{Plot of V\textsubscript{CE} vs I\textsubscript{C} for BC547}
% 	\end{minipage}
% 	\label{Switching circuit}
% \end{figure}
\vspace*{12pt}
The switching circuit shown in \fref{Switching circuit} is designed such that the relay is only turned on when both the sensor data as well as the valid transmission, VT, is high. To achieve this two transistors are placed in an AND gate configuration. The transistors are cascaded so that both of them have to be biased (by VT and D8) for the relay to turn on.
\vspace*{12pt}

\blocksvg{bc547}{.7}{Plot of CE voltage vs collector current for BC547}

% \vspace*{.5cm}

The relay needs 5V to operate, and its coil has a resistance of 60Ω. Therefore the current flowing through should be around 80mA. This means that the transistor must allow at least 80mA or more. From the V\textsubscript{CE} vs I\textsubscript{C} plot shown in \fref{Plot of CE voltage vs collector current for BC547}, it can be inferred that a base current of 400uA and more should be applied. The output voltage from the decoder IC is 5V. To get the desired 400uA at the base, a resistance can be connected in series. The resistor value is calculated to be 12kΩ. Therefore, a lower value of 10kΩ is chosen for the base resistor. It was noticed during irrigation the relay would occasionally flicker due to noise in the communication. To eliminate this, a capacitor of 220uF is placed in the base of the transistors, effectively acting as a low pass filter.