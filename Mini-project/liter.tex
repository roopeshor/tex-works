\chapter{Literature Review}
\setlength\parindent{1cm}

Irrigation is a vital agricultural practice that enables controlled water delivery to crops, ensuring their optimal growth. Traditionally, surface irrigation methods such as flood irrigation, furrow irrigation, and basin irrigation were the most widely used due to their simplicity and low cost. However, these methods are highly inefficient, leading to significant water loss through evaporation, deep percolation, and runoff.  They often result in overwatering or underwatering, soil erosion, affecting both crop yield and soil health. The addition of soil monitoring sensors can provide more accurate decisions on irrigation scheduling in a micro irrigated field.\cite{gupta2022}

\floatfig{r}{.35}{micro-irr.png}{1}{Drip irrigation}{-.4cm}
Micro-irrigation has emerged as a critical solution for improving water use efficiency in agriculture, especially under increasing water scarcity. One such method known as Drig irrigation is shown in \fref{Drip irrigation}. The advantages over conventional methods include precise water delivery, reduced evaporation and minimal runoff. Recent advances such as real-time soil moisture sensors, automation, and nano-filtration have expanded its potential. The study emphasizes the need for innovation, policy support, and cost-effective solutions to broaden micro-irrigation's reach and impact. To overcome these limitations, researchers have explored technology-assisted irrigation systems, which use sensors and actuators to automate water delivery based on real-time soil and environmental parameters. These systems can ensure timely and precise irrigation, significantly reducing water wastage and improving crop health. Moreover, by utilizing microcontrollers and wireless communication modules, such systems can  operate remotely and autonomously with minimal human intervention.\cite{kanthal2024}

% Recent literature emphasizes the need for automated irrigation systems that incorporate real-time soil moisture sensing and remote control mechanisms. Such systems have been proposed as a solution to reduce manual labor, optimize water use, and improve yield consistency. The use of wireless modules allows for remote monitoring and control, making the system ideal for large or remote farms. This approach promotes efficient water usage, reduces the need for manual labor, and supports sustainable agriculture. It also aligns with modern irrigation practices by improving system management, reducing delays, and conserving resources.Irrigation governance in developing countries has experienced multiple reform waves—such as Participatory Irrigation Management (PIM), Irrigation Management Transfer (IMT), Water Markets, and Public-rivate Partnerships (PPP)—to address inefficiencies in large public irrigation systems. These reforms aim to involve users more directly, enhance cost recovery, and improve water allocation. However, challenges like institutional resistance, weak legal frameworks, limited user capacity, and infrastructure deterioration have often hindered their success (Playán et al., 2018).
Recent literature emphasizes the growing need for automated irrigation systems equipped with real-time soil moisture sensors and remote control mechanisms. These systems reduce labor requirements, improve water-use efficiency, and enhance crop yield consistency. Wireless modules enable remote monitoring and control, making such setups highly suitable for large-scale or remote farms. This approach aligns well with sustainable agriculture goals and modern irrigation strategies, offering a practical solution to operational delays and water resource inefficiencies\cite{water2018}.

In the project, an automatic irrigation system is proposed using RF communication and a NodeMCU-based monitoring unit. The system tries addresses the shortcomings of traditional irrigation methods by delivering water based on real-time soil moisture conditions, thereby minimizing wastage and ensuring consistent irrigation.