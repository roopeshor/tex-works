\chapter{Literature Review}
\setlength\parindent{1cm}

Irrigation is a vital agricultural practice that enables controlled water delivery to crops, ensuring their optimal growth. Traditionally, surface irrigation methods such as \textit{flood irrigation}, \textit{furrow irrigation}, and \textit{basin irrigation} were the most widely used due to their simplicity and low cost. However, these methods are highly inefficient, leading to significant water loss through \textit{evaporation}, \textit{deep percolation}, and \textit{runoff}. They often result in \textit{overwatering or underwatering}, soil erosion, and salinity buildup, affecting both crop yield and soil health \cite{gupta2022}.

To overcome these limitations, researchers have explored \textit{technology-assisted irrigation systems}, which use sensors and actuators to automate water delivery based on real-time soil and environmental parameters. These systems can ensure timely and precise irrigation, significantly reducing water wastage and improving crop health. Moreover, by leveraging microcontrollers and wireless communication modules, such systems can operate remotely and autonomously with minimal human intervention \cite{kanthal2024,kanda2022}. Studies have demonstrated that low-power and modular designs, like those using NodeMCU and RF modules, offer a promising balance between affordability, scalability, and effectiveness, especially in resource-constrained agricultural settings \cite{nair2023}.

In countries like India and Ethiopia, large-scale public irrigation systems suffer from poor infrastructure, lack of maintenance, and inefficient water governance, which further diminishes the effectiveness of these systems \cite{ethiopia2022}. Furthermore, centralized systems lack adaptability to local soil and crop conditions and do not leverage modern sensing or automation technologies.

Recent literature emphasizes the need for \textit{automated irrigation systems} that incorporate real-time soil moisture sensing and remote control mechanisms. Such systems have been proposed as a solution to reduce manual labor, optimize water use, and improve yield consistency \cite{water2018}. However, challenges remain in terms of affordability, scalability, and power efficiency for small and marginal farmers.

In this project, an automatic irrigation system is proposed using RF communication and a NodeMCU-based monitoring unit. This system directly addresses the shortcomings of traditional irrigation methods by delivering water based on real-time soil moisture conditions, thereby minimizing waste and ensuring consistent irrigation.