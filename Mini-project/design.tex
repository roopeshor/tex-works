\newpage
\chapter{Circuit Design}
\section{Encoder IC oscillator}
\begin{figure}[ht]
	\centering
	\addsvg{images/enc-osc.svg}{.8\linewidth}
	\caption{Plot of Oscillator frequency vs supply voltage for HT12E}
\end{figure}

\vspace*{.5cm}
The encoder is powered by a voltage supply of 3.7V. An arbitrary
value of 1MΩ is chosen for the resistor across the oscillator pins
OSC1 and OSC2 (pins 16 and 15). From the Oscillator Frequency vs
Supply Voltage graph of the encoder IC, the oscillation frequency
corresponding to the particular resistance and voltage values is seen
to be 2.71kHz.

\newpage

\section{Decoder IC oscillator}
\begin{figure}[ht]
	\centering
	\addsvg{images/dec-osc.svg}{.8\linewidth}
	\caption{Plot of Oscillator frequency vs supply voltage for HT12E}
\end{figure}

\vspace*{.5cm}
The recommended oscillator frequency for the decoder is 50 times that
of the encoder's frequency. In this case it has to be 135kHz. The
decoder is powered by a voltage supply of 5V. From the Oscillator
Frequency vs Supply Voltage graph of the decoder, it can be seen that
the resistor value corresponding to a 5V supply voltage and
calculated oscillation frequency (135kHz) is 62kΩ. The nearest
resistance value which we had was 68kΩ and we proceeded to use it
.This resistor is connected across the pins OSC1 and OSC2 (pins 16 and 15).

\section{Switching circuit}
The switching circuit is designed such that the relay is only turned
on when both the sensor data as well as the valid transmission, VT,
is high. To achieve this two transistors are placed in an AND gate
configuration. The transistors are cascaded so that both of them have
to be biased (by VT and D8) for the relay to turn on.

The relay needs 5V to operate, and its coil has a resistance of 60Ω.
Therefore the current flowing through should be around 80mA. This
means that the transistor must allow at least 80mA or more. From the
IC vs VCE graph, it can be inferred that a base current of 400uA and
more should be applied. The output voltage from the decoder IC is 5V.
To get the desired 400uA at the base, a resistance can be connected
in series. The resistor value is calculated to be ~12kΩ. Therefore,
the base resistor is chosen to be 10kΩ.

It was noticed that during transmission, during irrigation the relay
would occasionally flicker due to some noise in the communication. To
eliminate this, a capacitor of  220uF is placed in the base of the
transistors, effectively acting as a low pass filter.
