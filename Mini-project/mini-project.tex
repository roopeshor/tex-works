\documentclass[12pt, a4paper]{report}
\usepackage{geometry}
\usepackage[utf8]{inputenc}
\usepackage{amsmath}
\usepackage{amsfonts}
\usepackage{graphicx}
\usepackage{color}
\usepackage{accents}
\usepackage{cite}
\usepackage{svg}
\usepackage{fontspec}
\usepackage{color}
\usepackage[hidelinks]{hyperref}
\usepackage{setspace}
\usepackage{etoolbox}
\usepackage{tocloft}
\usepackage{titlesec}
\usepackage{indentfirst}
\usepackage[format=plain, font=it]{caption}
\usepackage{wrapfig}

\newcommand{\subtitlefont}{\fontsize{18}{18}\selectfont}
\newcommand{\chapterFont}{\fontsize{18}{18}\selectfont}
\newcommand{\sectionFont}{\fontsize{14}{14}\selectfont}
\newcommand{\titlefont}{\fontsize{25}{25}\selectfont}
\newcommand{\titlepagefont}{\fontsize{13}{13}\selectfont}
\newcommand{\bodyfont}{\fontsize{13}{13}\selectfont}
\newcommand{\certificateTitle}[1]{{\centering \fontsize{18}{18}\selectfont #1}}
\newcommand{\headingc}[1]{
  \begin{center}
    \chapterFont #1
  \end{center}
}
\newcommand{\doindent}{\setlength\parindent{1cm}}
\newcommand{\addsvg}[2]{\includesvg[inkscapearea=drawing, width=#2]{#1}}
\newcommand{\addimg}[2]{\includegraphics[width=#2]{#1}}
\newcommand{\floatfig}[6]{
  %{l|r}{w}{img}{size}{cap}{bot}
  \begin{wrapfigure}{#1}{#2\textwidth}
    \centering
    \addimg{images/#3}{#4\linewidth}
    \caption{\centering #5}
    \vspace*{#6}
  \end{wrapfigure}
}
\newcommand{\floatfigsvg}[6]{
  %{l|r}{w}{img}{size}{cap}{bot}
  \begin{wrapfigure}{#1}{#2\textwidth}
    \centering
    \addsvg{images/#3}{#4\linewidth}
    \caption{\centering #5}
    \vspace*{#6}
  \end{wrapfigure}
}

\titleformat{\chapter}[display]
{\chapterFont \centering} % Font settings: Huge size, bold, centered
{\chaptername\ \thechapter} % Numbering format
{10pt} % Space between number and title
{\chapterFont} % Title font settings
\titlespacing*{\chapter}{0pt}{0in}{20pt}

\titleformat*{\section}{\sectionFont}

\makeatletter
\patchcmd{\chapter}{\if@openright\cleardoublepage\else\clearpage\fi}{}{}{}

\begin{document}
\pagenumbering{gobble}
\linespread{1.2}
\setmainfont{Times New Roman}
\begin{titlepage}
  \titlepagefont
  \newgeometry{top=1.7in, bottom=1in, left=1in, right=1in}
  \centering
  {\subtitlefont Mini Project Report on \par}
  \vspace{.3cm}

  {\titlefont {Automatic Irrigation System} \par}

  \vspace{1cm}

  {\textit{Submitted by} \par}
  \vspace{.5cm}

  \textbf {Roopesh O R (20323085)} \par
  \textbf {Roshna Palatty Santhosh (20323086)} \par
  \textbf {Sumayya Punnoth (20323100)} \par
  \textbf {Merella Jobi (20323113)} \par

  \vspace{1cm}
  {\textit{in partial fulfillment of requirement for the award of the
  degree} \par}
  {\textit{Of}} \par
  \vspace{.6cm}
  {\textbf{BACHELOR OF TECHNOLOGY} \par}
  {\textbf{in}} \par
  {\textbf{ELECTRONICS AND COMMUNICATION} \par}
  \vspace{1cm}

  \addsvg{cusat}{1.5in}\par

  \vspace{1cm}

  {\textbf{Division of Electronics Engineering} \par}
  {\textbf{School of Engineering} \par}
  {\textbf{Cochin University of Science and Technology} \par}
  {\textbf{Kochi - 682022} \par}

  \vspace{.5cm}

  {\textbf{April 2025} \par}

  \vfill

\end{titlepage}
%% certificate
\newgeometry{top=3cm, bottom=3cm, left=3cm, right=3cm}
\vspace*{3cm}
\begin{center}

  \certificateTitle{
    DIVISION OF ELECTRONICS ENGINEERING \\
    SCHOOL OF ENGINEERING \\
    COCHIN UNIVERSITY OF SCIENCE AND TECHNOLOGY \\
    KOCHI-682022 \\
  }
  \vspace{1cm}
  \addsvg{cusat}{1.5in}\par
  \vspace{1cm}
  \certificateTitle{\textbf {Certificate}}
  \vspace{.3cm}

  \titlepagefont
  \setstretch{1.5}
  \textit{
    Certified that the project report entitled \strong{“Automatic
    Irrigation System"}
    is a bonafide work of \strong{Roopesh O R, Roshna Palatty
    Santhosh, Sumayya Punnoth, Merella Jobi} towards the partial
    fulfillment for the award of the
    degree of B.Tech in Electronics and Communication of Cochin
    University of Science and Technology, Kochi-682022.
  }

  \vspace{3cm}
  \begin{minipage}[t][][r]{.4\linewidth}
    Project Co-ordinator
  \end{minipage}
  \hfill
  \begin{minipage}[t][][l]{.4\linewidth}
    \begin{flushright}
      Head of Division\\
      Dr. Deepa Sankar
    \end{flushright}
  \end{minipage}
\end{center}

\newpage
\setstretch{1.2}
\headingc{Abstract}

This project details the design and implementation of an automatic
irrigation system with remote monitoring, addressing the critical
need for efficient water management in agriculture. The system
consists of a Transmitter, Receiver, Soil Moisture Sensor, NodeMCU,
Encoder, Decoder, Relay and Motor. The transmitter utilizes a soil
moisture sensor to continuously monitor the moisture level of soil,
encodes and transmits the data. The receiver receives this data,
decodes and activates the motor through a relay which pumps the water
till the threshold water level is reached. This data can be accessed
through a webpage hosted by NodeMCU.

\newpage

\renewcommand{\contentsname}{Table of contents}
\renewcommand{\cfttoctitlefont}{\chapterFont}
\renewcommand{\cftbeforetoctitleskip}{0pt}
\renewcommand{\cftbeforeloftitleskip}{0pt}
\renewcommand{\cftbeforelottitleskip}{0pt}
\renewcommand{\cftaftertoctitleskip}{10pt}
\renewcommand{\cftafterlottitleskip}{10pt}
\renewcommand{\cftafterloftitleskip}{10pt}
\renewcommand{\cftlottitlefont}{\chapterFont}
\renewcommand{\cftloftitlefont}{\chapterFont}

\renewcommand{\cftdotsep}{4}
\renewcommand{\cftsecleader}{\cftdotfill{\cftdotsep}}
\renewcommand{\cftsubsecleader}{\cftdotfill{\cftdotsep}}
\renewcommand{\cftpartleader}{\cftdotfill{\cftdotsep}} % for parts
\renewcommand{\cftchapleader}{\cftdotfill{\cftdotsep}}

\newgeometry{top=2cm}

\tableofcontents

\newpage
\pagenumbering{roman}

\cleardoublepage
\phantomsection
\addcontentsline{toc}{chapter}{\listfigurename}
\listoffigures

\newpage

\cleardoublepage
\phantomsection
\addcontentsline{toc}{chapter}{\listtablename}
\listoftables
\newpage

\doindent
\pagenumbering{arabic}
\chapter{Introduction}
\setlength\parindent{1cm}

Efficient Irrigation is very crucial in modern agriculture for
maximizing both production and profit. Water scarcity and inefficient
irrigation practices pose significant challenges. Traditional
irrigation methods often lead to overwatering or underwatering,
resulting in poor crop yield and excessive water consumption. With
the increasing demand for sustainable farming solutions, automated
irrigation systems provide an efficient way to optimize water usage
while ensuring healthy plant growth.

The Automatic Irrigation System automates plant watering based on
real-time soil moisture levels. The system uses 433MHz RF for
wireless communication. In the transmitter unit the soil moisture
sensor continuously measures whether the moisture content in the soil
is above or below a set reference and produces a corresponding
digital signal. Then an encoder (HT12E) encodes this data in the
transmitter unit and transmits it using an RF transmitter module. The
data is received and decoded at the receiver unit and according to
the received data the receiver unit activates a relay, turning on a
water pump to irrigate the soil. Once the moisture level reaches the
desired value, the transmitter detects this change and sends an
update to the receiver, which then deactivates the relay, switching
off the pump. This data can be accessed through a webpage hosted by a
NodeMCU. This system ensures efficient water management, reducing
wastage and minimizing manual intervention, making it an ideal
solution for smart agriculture.

\newpage
\chapter{Objective}
\noindent
The main objectives of this project are:
\begin{enumerate}
  \item To improve irrigation efficiency while reducing operational
    costs and labor.
  \item Reduce water wastage through sensing and control.
  \item Enable remote monitoring of agriculture fields
\end{enumerate}

\newpage
\chapter{Block diagram}

\begin{figure}[ht]
	\centering
	\addsvg{images/block}{.9\linewidth}
	\caption{Block Diagram}
  \end{figure}

The soil moisture sensor placed in the soil in the field produces a
digital signal representing whether soil is dry or not is generated
by comparing moisture level and a set reference voltage. This signal
is encoded and transmitted over radio. This signal is received and
decoded at a receiver setup in the home. Based on the signal
received, the receiver drives a switch which actuates the relay and
hence the water pump, supplying water to the field. The soil moisture
status is read by NodeMCU and updates the status on a website hosted in NodeMCU

\newpage
\chapter{Circuit Diagram}

\section{Transmitter}
\begin{figure}[ht]
  \centering
  \addsvg{images/tx}{.7\linewidth}
  \caption{Transmitter circuit}
\end{figure}
The resistive soil moisture sensor placed in the soil continuously
senses the water level of the soil, providing a digital output from
the DO pin of the sensor based on a set threshold level. The sensor
gives a high output if the water content falls below the threshold
level, and a low output otherwise. This digital signal is directly
fed into the input data pin AD8 (pin 10) of the HT12E encoder. The
address pins A0 to A7 (pins 1 to 8) of the encoder are set to 0. The
HT12E encoder then takes this digital signal and encodes it with the
address creating a data packet. The data packet obtained from DOUT
(pin 17) is then sent to an RF transmitter module’s data input pin
DATA for modulation and wireless transmission. This RF transmitter
modulates the digital data onto a radio frequency carrier signal of
433Mhz and then transmits it wirelessly via an antenna of length
20cm. This transmission occurs continuously based on the data from
the moisture sensor.

\newpage

\section{Receiver}
\begin{figure}[ht]
  \centering
  \addsvg{images/rx}{.8\linewidth}
  \caption{Receiver circuit}
\end{figure}

The RF receiver module tuned at 433MHz receives the radio frequency
signal transmitted by the transmitter unit. The RF module demodulates
the received signal. This decoded data is then fed into the HT12D
decoder’s data input pin DIN (pin 14). The decoder is configured with
the same address as the HT12E encoder in the transmitter by setting
the address pins A0 to A7 (pins 1 to 8) to ground. It checks the
address portion of the received data to ensure it matches its own. If
the addresses match, the decoder extracts the data portion of the
signal and presents it through its output data pins D8-D11. This
extracted digital data represents the soil moisture status and is
then taken from D8 (pin 10). The data is then given to a relay such
that it only turns on if data is high (i.e., water content is low)
and transmission is valid. This is achieved by using two transistors
in an AND gate configuration that takes the values from VT (pin 17)
and D8 (pin 10) as input.

The relay module is wired in a normally OFF configuration. It is
switched to ON state when both VT and D8 pins are high. This turns on
a water pump that is connected to the relay and pumps water to the
plants. When the received digital data from the soil moisture sensor
becomes low and the switching circuit turns off relay and it switches
back to OFF state which turns the water pump off.

\newpage

\chapter{Components}

\section{Soil Moisture Sensor}

\setlength\intextsep{0pt}
\setlength{\columnsep}{.6cm}
\floatfig{r}{.4}{sms.png}{1}{Soil moisture sensor}{.4cm}

The Resistive Soil Moisture Sensor Module consists of two probes
which are used to measure the moisture content of water. Resistive
soil moisture sensors work by measuring the resistance of the soil,
which changes with moisture content, with higher moisture leading to
lower resistance. The sensor uses two probes to pass a current
through the soil, and the resistance encountered is used to estimate
moisture levels. The sensing circuit is calibrated to different soil
types for proper measurement by adjusting the built in potentiometer

\section{FS1000A: Transmitter \& Receiver module}

\floatfig{l}{.45}{fs100a.png}{.9}
{FS1000A receiver \&\\ transmitter pair}{.2cm}

This is a combination of two modules used for data transmission and
reception. The  core of the transmitter module is a SAW resonator
tuned to operate at 433MHz. Apart from that, it has a switching
transistor and some passive components. When the DATA input is high,
the oscillator generates a constant RF output carrier wave at 433MHz,
and when the DATA input is low, the oscillator ceases operation;
resulting in an amplitude modulated wave. The module can cover a
minimum of 10 meters and with proper antenna and power supplies.
Theoretically it can transmit up to 100 meters.

The Receiver module consists of an RF tuned circuit and an Op-Amp
that amplify the received carrier wave. The amplified signal is then
fed into a PLL Phase Lock Loop, which allows the demodulator to
“lock” onto frequency of incoming signal. The demodulated signal is
preset at two output pins which are internally connected
\newpage

\section{HT12E: Encoder IC}

\floatfigsvg{r}{.45}{ht12e.svg}{.8}{HT12E encoder IC pinout}{-1cm}

The function of HT12E is to encode a 4-bit data and 8-bit address and
send it out through the output pin. The IC has operating voltage
ranging from 2.4V to 5V with typical value of 3V. The Transmission
Enable pin (pin 14) is pulled to ground to activate transmission. The
4-bit data that has to be sent is given to the pins AD8 to AD11 and
an address of 8-bit is set using the pins A0 to A7.  Moreover a
resistor has to be connected between OSC1 and OSC2 pins to set the
operation frequency of encoder

\vspace*{1.5cm}

\section{HT12D: Decoder IC}

\floatfigsvg{r}{.45}{ht12d.svg}{.8}{HT12D decoder IC pinout}{0cm}

The function of HT12D is to decode the data obtained from a receiver
circuit and send it through the output pins. The IC has a wide range
of operating voltage from 2.4V to 12V with typical voltage being 5V.
A signal on the DIN pin activates the oscillator which starts
decoding of the incoming signal and obtains data and address in it.
The decoder will then check if the received address matches with the
preset address three times continuously. If it matches, the 4-bits of
data are decoded to activate the output pins D8-D11 and the VT pin is
set high to indicate a valid transmission.

\section{BC547B: Transistor}

\floatfig{r}{.45}{bc547.png}{.5}{BC547 transistor}{-2cm}

The BC547B, an NPN transistor, can handle current up to 200mA. The
gain current of this transistor is from 110 to 800. The maximum base
to emitter reverse voltage it can handle is 6V and maximum collector
to emitter voltage is 45V. The transistor is available in the TO-92 package.

\vspace*{1cm}

\section{18650 Battery and case}

\floatfig{r}{.45}{battery.png}{.9}{18650 battery and its holder}{-1cm}

An 18650 battery is a cylindrical, lithium-ion rechargeable battery,
named for its dimensions (18mm diameter and 65mm length), known for
its high energy density and long lifespan. Here we have used a 3.7V
18650 battery with 2600mAh capacity. This battery was chosen at the
transmitter circuit as it can power the circuit for months without
recharge in (ideal conditions)

\vspace*{1cm}

\section{JQC-3FC(T73)DC05V: Relay}

\setlength{\columnsep}{.3cm}
\floatfig{r}{.45}{relay.png}{.6}{PCB Mount Relay}{0cm}

This is an electromechanical relay that mechanically switches contact
between two points. The relay is operated by applying a voltage
across the coil pins. When coil gets energized producing a magnetic
field that actuates a contact lever. When the coil is de-energized
the contact returns to normal position. The relay we used can handle
switching of 120V 10A AC supply or 24V 10A DC supply. The relay can
switch at most 60Hz. The coil has to be powered by a 5V supply in
order for it to actuate the contact

\section{DC Motor Pump}

\floatfig{r}{.4}{pump.png}{.6}{DC motor pump}{-1cm}

We used a 6V DC motor pump to supply water to the field. The motor
consumes 4-5W of power. The motor can work from 3V to 6V. It can
supply water up to a height of 110cm at maximum operating voltage.
For the purpose of demonstration, the motor is powered by a regulated
power supply that can supply 5V DC

\vspace*{1cm}

\section{NodeMCU}
The NodeMCU ESP32 is a development board based on the ESP32
microcontroller. It is programmable with various programming
languages. This board has 2.4 GHz dual-mode Wifi and a BT wireless
connection. In addition, a 512 KB SRAM and a 4MB flash memory are
integrated into the microcontroller development board. The board has
21 pins for interface connection, including I2C, SPI, UART, DAC, and
ADC. The NodeMCU's ADC (Analog-to-Digital Converter) input is
configured for the 0-3.3V range with appropriate signal conditioning
to match sensor output characteristics.  The module can be powered
using an external supply through its micro USB port or with 5V and
GND pins in the board

\newpage
\chapter{Circuit Design}
\section{Encoder IC oscillator}
\begin{figure}[ht]
	\centering
	\addsvg{images/enc-osc.svg}{.8\linewidth}
	\caption{Plot of Oscillator frequency vs supply voltage for HT12E}
  \end{figure}

\vspace*{.5cm}
The encoder is powered by a voltage supply of 3.7V. An arbitrary
value of 1MΩ is chosen for the resistor across the oscillator pins
OSC1 and OSC2 (pins 16 and 15). From the Oscillator Frequency vs
Supply Voltage graph of the encoder IC, the oscillation frequency
corresponding to the particular resistance and voltage values is seen
to be 2.71kHz.

\newpage

\section{Decoder IC oscillator}
\begin{figure}[ht]
	\centering
	\addsvg{images/dec-osc.svg}{.8\linewidth}
	\caption{Plot of Oscillator frequency vs supply voltage for HT12E}
  \end{figure}

\vspace*{.5cm}
The recommended oscillator frequency for the decoder is 50 times that
of the encoder's frequency. In this case it has to be 135kHz. The
decoder is powered by a voltage supply of 5V. From the Oscillator
Frequency vs Supply Voltage graph of the decoder, it can be seen that
the resistor value corresponding to a 5V supply voltage and
calculated oscillation frequency (135kHz) is 62kΩ. The nearest
resistance value which we had was 68kΩ and we proceeded to use it
.This resistor is connected across the pins OSC1 and OSC2 (pins 16 and 15).

\section{Switching circuit}
The switching circuit is designed such that the relay is only turned
on when both the sensor data as well as the valid transmission, VT,
is high. To achieve this two transistors are placed in an AND gate
configuration. The transistors are cascaded so that both of them have
to be biased (by VT and D8) for the relay to turn on.

The relay needs 5V to operate, and its coil has a resistance of 60Ω.
Therefore the current flowing through should be around 80mA. This
means that the transistor must allow at least 80mA or more. From the
IC vs VCE graph, it can be inferred that a base current of 400uA and
more should be applied. The output voltage from the decoder IC is 5V.
To get the desired 400uA at the base, a resistance can be connected
in series. The resistor value is calculated to be ~12kΩ. Therefore,
the base resistor is chosen to be 10kΩ.

It was noticed that during transmission, during irrigation the relay
would occasionally flicker due to some noise in the communication. To
eliminate this, a capacitor of  220uF is placed in the base of the
transistors, effectively acting as a low pass filter.

\newpage
\chapter{Construction and Testing}
\section{PCB Layout}

\section{Assembled circuit}

\newpage
\chapter{Conclusion}
With this project we have successfully implemented an automatic
irrigation system enabling efficient water management through remote
monitoring and wireless communication. By using a soil moisture
sensor, the system monitors the moisture level of soil and supplies
water based on soil conditions. This ensures optimal plant hydration
reducing manual labour and water wastage. The system not only
optimizes water usage but also promotes sustainable growth by
preventing overwatering and underwatering.

\newpage
\chapter{Limitations}
Currently the maximum range of stable transmission is 10 meters. This
could be further improved by proper designing of antennas. This could
also avoid cases where transmission can occasionally halt for a brief
time. Also the system can fail during precipitation due to limited
waterproofing..

\newpage
\chapter{Future scope}
\begin{enumerate}
  \item Create an enclosure: \\
    Since the system operates in an external agricultural
    environment, exposure to    dust, soil and moisture is
    inevitable. A well sealed-enclosure prevents water damage and
    dust accumulation ensuring smooth working of sensitive components.
  \item Integrating additional sensors: \\
    Sensors such as temperature, humidity, light, pH sensors etc. can
    be helpful in optimizing irrigation based and also helps in
    gathering information about environment of farmland
  \item Remote control: \\
    If needed, the user can optionally turn on the motor remotely.
    This is useful in cases where the user has prior knowledge of
    possible power supply failure later in the day. Hence user can
    water plants in advance.
  \item Using drip irrigation and recycling of water: \\
    In  drip irrigation water is supplied to the roots of plants drop
    by drop. This can further reduce wastage of water through
    evaporation or runoff
\end{enumerate}

\newpage
\chapter{References}
\begin{enumerate}
  \item “Electronics For You Express”, february 2021 edition, page 56
    - https://online.\\fliphtml5.com/oxomv/fjlb/
  \item HT12E data encoder - https://www.farnell.com/datasheets/1899539.pdf
  \item HT12D data decoder - https://www.farnell.com/datasheets/57850.pdf
  \item BC547B NPN transistor -
    https://cdn.sparkfun.com/assets/d/5/e/5/d/BC547.pdf
  \item FS1000A radio transmitter and receiver -
    https://robu.in/wp-content/uploads/2016/05/\\FS1000A-datasheet.pdf
  \item JQC-3FC(T73)DC05V sugar cube relay
    https://pdf.voron.ua/files/pdf\\/relay/JQC-3F(T73).pdf
\end{enumerate}

\end{document}
