\chapter*{Abstract}
With the increasing demand for food production and the growing unpredictability of natural rainfall patterns, efficient irrigation techniques have become more important than ever. Traditional irrigation methods such as flood irrigation, furrow irrigation, and basin irrigation are highly inefficient and often lead to water wastage, uneven water distribution, nutrient leaching and increased labour requirement. An automatic irrigation system overcomes these issues by addressing the critical need for efficient water management in agriculture. It aims to automate the plant watering process and minimise human intervention. The system is designed and implemented with wireless communication and remote monitoring to ensure optimal water utilization. The key components used in the project are RF Transmitter and Receiver modules, Encoder, Decoder, Relay and Motor. The system also makes use of a NodeMCU for remote monitoring. The transmitter section works by utilising a soil moisture sensor to continuously monitor the moisture level of soil, encoding and transmitting the data. The receiver section receives this data, decodes and activates the motor through a relay which pumps the water. Once sufficient moisture is detected, the water pump is turned off. This data can be accessed through a webpage hosted by NodeMCU. By using RF communication, wiring constraints are eliminated, making the system suitable for remote and wide-area applications. The project demonstrates how automation and wireless technology can be effectively combined to achieve efficient and sustainable irrigation.
