\newpage
\chapter{Results and Discussion}

With this project we successfully implemented an automatic irrigation system enabling efficient water management through remote monitoring and wireless communication. By using a soil moisture sensor, the system monitored the moisture level of soil and supplies water when soil was dry. The status was also indicated on a LED on the board. The level of soil moisture had to set through potentiometer on the soil moisture sensor. It took multiple attempts to tune the sensor properly.

The use of high capacity rechargeable battery in transmitter allows a sustainable way to operate the device. Also the battery can run for months at a stretch hence reducing maintenance labour. In the receiver, 5V supply is needed for the operation. For this a USB type-C connector was utilized so that any 5V adapter such as phone charger can be utilized for the product. Thus user can use old phone chargers to operate the device increasing ease of use. This reduces cost of the product as a separate power supply need not be included with the product. Additionally a simple voltage limiting circuit can be utilized to reduce damage caused by sudden voltage fluctuation in case the adapter fails. \fref{Prototype of the transmitter and receiver section} shows the Prototype of the transmitter and receiver created in perf board.

\vspace*{.8cm}
\blockimage{assmb.jpg}{.7}{Prototype of the transmitter and receiver section}
\vspace*{.8cm}

A web interface was also developed which displays the status of connectivity between radio modules, soil moisture and state of the pump. The site also includes a logging feature to check the past activity of the pump. This data is retained throughout the runtime of the NodeMCU. The site will be automatically refreshed every 3 seconds and fetches the data from the sensors. \fref{Web interface showing status of the system} Shows the designed web page.

\vspace*{20pt}
\blockimage{s2.png}{.7}{Web interface showing status of the system}[6pt]

\vspace*{12pt}
One issue is that the address of the webpage has to be obtained from NodeMCU. With the current prototype it can be done either by mapping the network or connecting NodeMCU to a computer and reading the USB Serial output as shown in \fref{USB Serial Output}. Here connecting NodeMCU to a Wi-Fi network is done by manually storing Wi-Fi credentials inside the NodeMCU. 
\vspace*{20pt}
\blockimage{s1.png}{.6}{USB Serial Output}