\chapter{Introduction}
\setlength\parindent{1cm}

Efficient Irrigation is very crucial in modern agriculture for
maximizing both production and profit. Water scarcity and inefficient
irrigation practices pose significant challenges. Traditional
irrigation methods often lead to overwatering or underwatering,
resulting in poor crop yield and excessive water consumption. With
the increasing demand for sustainable farming solutions, automated
irrigation systems provide an efficient way to optimize water usage
while ensuring healthy plant growth.

The Automatic Irrigation System automates plant watering based on
real-time soil moisture levels. The system uses 433MHz RF for
wireless communication. In the transmitter unit the soil moisture
sensor continuously measures whether the moisture content in the soil
is above or below a set reference and produces a corresponding
digital signal. Then an encoder (HT12E) encodes this data in the
transmitter unit and transmits it using an RF transmitter module. The
data is received and decoded at the receiver unit and according to
the received data the receiver unit activates a relay, turning on a
water pump to irrigate the soil. Once the moisture level reaches the
desired value, the transmitter detects this change and sends an
update to the receiver, which then deactivates the relay, switching
off the pump. This data can be accessed through a webpage hosted by a
NodeMCU. This system ensures efficient water management, reducing
wastage and minimizing manual intervention, making it an ideal
solution for smart agriculture.
