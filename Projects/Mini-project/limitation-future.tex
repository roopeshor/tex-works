\chapter{Limitations \& Future Scope}
\section{Limitations}
Currently the Project is not fully ready for production purposes. Listed below are few limitations of the current implementation. These can be considered as a part of future improvement.
\begin{enumerate}
  \item \textbf{Limited Wireless Range} \\
    The current system's wireless transmission is limited to approximately 10 meters, restricting its application in large-scale agricultural settings without signal amplification or alternative communication technologies.
  \item \textbf{Environmental Protection} \\
    The prototype lacks comprehensive weatherproofing, making it vulnerable to malfunction during rainfall or in high-humidity environments, potentially damaging exposed electronic components.
  \item \textbf{Feedback Mechanisms} \\
    The system includes a basic feedback mechanism where the soil moisture status is read by the NodeMCU and updated in real time on a hosted website. However, it currently lacks a dedicated feedback method to confirm the successful execution of control commands such as verifying whether the pump has actually turned on when triggered which can limit reliability in remote monitoring or automation scenarios
  \item \textbf{Monitoring Capabilities} \\
    The current version monitors only soil moisture levels. Incorporation of additional environmental parameters (temperature,humidity,pH) would enhance decision-making capabilities and system versatility.
\end{enumerate}

\section{Future scope}
\begin{enumerate}
  \item \textbf{Create an enclosure}: \\
    Since the system operates in an external agricultural
    environment, exposure to    dust, soil and moisture is
    inevitable. A well sealed-enclosure prevents water damage and
    dust accumulation ensuring smooth working of sensitive components.
  \item \textbf{Integrating additional sensors}: \\
    Sensors such as temperature, humidity, light, pH sensors etc. can
    be helpful in optimizing irrigation and also helps in
    gathering information about environment of farmland
  \item \textbf{Extending range of radio system}: \\
    Currently the system can operate at a range of 10m. However the module itself can transmit at a maximum range of 100m. This can be improved by using better antennas. Moreover, there are newer versions of the radio modules that are more reliable.
  \item \textbf{Improving feedback mechanism}: \\
    Creating a mechanism to check the actual status of the motor is useful for troubleshooting.
  \item \textbf{Improving Wi-Fi connectivity}: \\
    In the current implementation the Wi-Fi credentials have to be manually embedded into NodeMCU. This could be avoided by using an application that creates secure Wi-Fi Hotspot and enabling NodeMCU to connect to Wi-Fi.	
  \item \textbf{Using cheaper alternative to ESP32}: \\
    ESP32 has more features such as bluetooth which are redundant and contributes to increased idle current. Alternatives to ESP32 such as ESP8266 can be employed in this case
  \item \textbf{Remote control}: \\
    If needed, the user can optionally turn on the motor remotely.
    This is useful in cases where the user has prior knowledge of
    possible power supply failure later in the day. Hence user can
    water plants in advance.
\end{enumerate}
