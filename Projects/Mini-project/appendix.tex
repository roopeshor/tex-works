\newpage
\phantomsection
\addcontentsline{toc}{chapter}{Appendix}
\chapter*{Appendix}
\setlength\parindent{1cm}
\section*{Source code for web server}
\setmonofont{IBM Plex Mono}
\fontsize{11}{11}\selectfont
\setstretch{1.3}
\begin{minted}[breaklines]{cpp}
#include <WiFi.h>
#include <WebServer.h>

const char* ssid = "home";
const char* password = "psw@shell12";

WebServer server(80);

String pumpStatus = "OFF";
String pump = "OFF";
String logData = "";

void handleRoot() {
  int sensor1=digitalRead(32); //valid transmission
  int sensor2=digitalRead(35); //moisture
 
  String vt = sensor1==1 ? "Connected" : "Disconnected";
  String moist = sensor2==1 ? "Dry" : "Wet";
  
  if (sensor1==1 && sensor2==1) {
    pump = "ON";
  } else {
    pump = "OFF";
  }

  server.send(200, "text/html",
  "<meta http-equiv='refresh' content='3'>"
  "<script>"
    "function updateTime() {"
    "  const now = new Date();"
    "  document.getElementById('timestamp').innerText = 'Last updated: "
    "' + now.toLocaleString();"
    "}"
    "window.onload = updateTime;"
  "</script>"
  "<h1>Status: " + vt + "</h1>"
  "<h1>" + moist + " soil</h1>"
  "<h1>Pump: " + pump + "</h1>"
  "<p id='timestamp'></p>"
  "<pre>" + logData + "</pre>"
  );
}

void checkForChanges() {
  int sensor1 = digitalRead(32);
  int sensor2 = digitalRead(35);

  unsigned long now = millis();
  String timeStr = String(now / 1000) + "s since boot";

  if (pump != pumpStatus) {
    logData += "Pump turned " + pump + " at " + timeStr + "\n";
    pumpStatus = pump;
  }

  if (logData.length() > 2000) {
    logData = "Log trimmed...\n" +
      logData.substring(logData.length() - 1000);
  }
}

void setup() {
  pinMode(32, INPUT);
  pinMode(35, INPUT);

  Serial.begin(115200);

  WiFi.begin(ssid, password);
  Serial.print("Connecting to WiFi..");

  while (WiFi.status() != WL_CONNECTED) {
    delay(500);
    Serial.print(".");
  }

  Serial.println("");
  Serial.println("WiFi connected.");
  Serial.println("IP address: ");
  Serial.println(WiFi.localIP());

  server.on("/", handleRoot);

  server.begin();
  Serial.println("HTTP server started");
}

void loop() {
  checkForChanges();
  server.handleClient();
}

\end{minted}