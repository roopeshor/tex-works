\section{Circuit Diagrams}
In this section, circuit diagrams of both transmitter and receiver circuits are shown along with a detailed explanation of their operation. The circuits were designed using KiCad software.
\subsection{Transmitter}

\blocksvg{tx}{.9}{Transmitter circuit}

\fref{Transmitter circuit} shows the circuit diagram of the transmitter circuit. The resistive soil moisture sensor placed in the soil continuously senses the water level of the soil, providing a digital output from the DO pin of the sensor based on a set threshold level. The sensor gives a high output if the water content falls below the threshold level, and a low output otherwise. This digital signal is directly fed into the input data pin AD8 (pin 10) of the HT12E encoder. The address pins A0 to A7 (pins 1 to 8) of the encoder are set to 0. The HT12E encoder then takes this digital signal and encodes it with the address creating a data packet. The data packet obtained from DOUT (pin 17) is then sent to an RF transmitter module's data input pin DATA for modulation and wireless transmission. This RF transmitter modulates the digital data onto a radio frequency carrier signal of 433MHz and then transmits it wirelessly via an antenna of length 20cm. This transmission occurs continuously based on the data from the moisture sensor.

\subsection{Receiver}

\fref{Receiver circuit} shows the circuit diagram of the transmitter circuit. The RF receiver module tuned at 433MHz receives the radio frequency signal transmitted by the transmitter unit. The RF module demodulates the received signal. This decoded data is then fed into the HT12D decoder's data input pin DIN (pin 14). The decoder is configured with the same address as the HT12E encoder in the transmitter by setting the address pins A0 to A7 (pins 1 to 8) to ground. It checks the address portion of the received data to ensure it matches its own. If the addresses match, the decoder extracts the data portion of the signal and presents it through its output data pins D8-D11. This extracted digital data represents the soil moisture status and is then taken from D8 (pin 10). The data is then given to a relay such that it only turns on if data is high (i.e., water content is low) and transmission is valid. This is achieved by using two transistors in an AND gate configuration that takes the values from VT (pin 17) and D8 (pin 10) as input.

The relay module is wired in a normally OFF configuration. It is switched to ON state when both VT and D8 pins are high. This turns on a water pump that is connected to the relay and pumps water to the plants. When the received digital data from the soil moisture sensor becomes low and the switching circuit turns off relay and it switches back to OFF state which turns the water pump off.

\blocksvg{rx}{.9}{Receiver circuit}