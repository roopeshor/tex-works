\documentclass{standalone}
\usepackage[utf8]{inputenc}
\usepackage{standalone}
\usepackage{circuitikz}
\usepackage[semibold,tt=false]{libertine}
\usepackage{libertinust1math}
\usepackage{bold-extra}
\begin{document}

%%%% Arduino drawer %%%%
\newcommand{\pinasgn}[3]{\anchor{#1}{\pgfpoint{#2}{#3}}}%
\newcommand{\linepoint}[4]{
	\pgfpathmoveto{\pgfpoint{#1}{#2}}%
	\pgfpathlineto{\pgfpoint{#3}{#4}}%
}%
\newcommand{\textttbf}[1]{\texttt{\textbf {#1}}}%
\newcommand{\pinlabel}[3]{
	\ifdim #1 < 0cm
		\pgftext[left,at={\pgfpoint{#1}{#2}}]{\fontsize{8.6}{8.6} \textttbf{#3}}
	\else
		\ifdim #1 > 0cm
			\pgftext[right,at={\pgfpoint{#1-.15cm}{#2}}]{\fontsize{8.6}{8.6} \textttbf{#3}}
		\else
			\pgftext[center,at={\pgfpoint{#1}{#2}}]{\fontsize{8.6}{8.6} \textttbf{#3}}
		\fi
	\fi
}%
% xstrt, xend, start, step, count
\newcommand{\pinarray}[5]{
	\pgfmathsetmacro{\next}{(#3) + (#4)}
	\pgfmathsetmacro{\last}{(#3) + (#4) * (#5)}
	\foreach \y in {#3, \next, ..., \last}{
			\linepoint{#1}{\y cm}{#2}{\y cm}
		}
}%

\newcommand{\fixedsm}{\fontsize{9}{9}}%
\newcommand{\ardwidth}{4cm}%
\newcommand{\ardheight}{8.8cm}%
\newcommand{\pinstartx}{\dimexpr\ardwidth/2\relax}%
\newcommand{\pinendx}{\dimexpr\pinstartx + .4cm\relax}%
\newcommand{\textmargin}{\dimexpr\pinstartx\relax}%
\newcommand{\ardabovemin}{\ardheight/2+.4cm}%

\newcommand{\mxwidth}{4cm}%
\newcommand{\mxheight}{2cm}%
\newcommand{\mxpinendx}{\dimexpr\mxpinstartx + 0.5cm\relax}%
\newcommand{\mxpinstartx}{\dimexpr\mxwidth / 2\relax}%
\newcommand{\mxmidheight}{\dimexpr\mxheight/4\relax}%
\newcommand{\boxsw}{1.5pt}%
\newcommand{\linesw}{1.3pt}%

\pgfdeclareshape{max7219}{%
	\anchor{center}{\pgfpointorigin}% within the node, (0,0) is the center
	\anchor{text}{
		\pgfpoint{-.5\wd\pgfnodeparttextbox}
		{-.5\ht\pgfnodeparttextbox}
	}%
	% Pin Assignments

	\pinasgn{CLK}{-\mxpinendx}{\mxmidheight}%
	\pinasgn{Data}{-\mxpinendx}{0}%
	\pinasgn{CS}{-\mxpinendx}{-\mxmidheight}%
	\pinasgn{5V}{\mxpinendx}{\mxmidheight}%
	\pinasgn{GND}{\mxpinendx}{-\mxmidheight}%
	\foregroundpath{
		%%% Container
		\pgfsetlinewidth{\boxsw}
		\pgfpathrectanglecorners
		{\pgfpoint{\mxpinstartx}{\mxheight/2}}
		{\pgfpoint{-\mxpinstartx}{-\mxheight/2}}
		\pgfusepath{draw}

		%%% pins

		\pgfsetlinewidth{\linesw}
		\linepoint{-\mxpinstartx}{\mxmidheight}{-\mxpinendx}{\mxmidheight}
		\linepoint{-\mxpinstartx}{0cm}{-\mxpinendx}{0cm}
		\linepoint{-\mxpinstartx}{-\mxmidheight}{-\mxpinendx}{-\mxmidheight}

		\linepoint{\mxpinstartx}{\mxmidheight}{\mxpinendx}{\mxmidheight}
		\linepoint{\mxpinstartx}{-\mxmidheight}{\mxpinendx}{-\mxmidheight}
		\pgfusepath{draw}


		\pinlabel{-\mxpinstartx}{\mxmidheight}{CLK}%
		\pinlabel{-\mxpinstartx}{0cm}{Data}%
		\pinlabel{-\mxpinstartx}{-\mxmidheight}{CS}%
		\pinlabel{\mxpinstartx}{\mxmidheight}{5V}%
		\pinlabel{\mxpinstartx}{-\mxmidheight}{GND}%

		\pgftext[at={\pgfpoint{0cm}{0.18cm}}]{\fontsize{10}{10}\textttbf{MAX}}
		\pgftext[at={\pgfpoint{0cm}{-0.18cm}}]{\fontsize{10}{10}\textttbf{7219}}
	}%
}%

\pgfdeclareshape{arduino}{%
	\anchor{center}{\pgfpointorigin}% within the node, (0,0) is the center
	\anchor{text}{
		\pgfpoint{-.5\wd\pgfnodeparttextbox}
		{-.5\ht\pgfnodeparttextbox}
	}%
	% Pin Assignments

	\pinasgn{IOREF}{-\pinendx}{3cm}%
	\pinasgn{RESET}{-\pinendx}{2.5cm}%
	\pinasgn{3V3}{-\pinendx}{2cm}%
	\pinasgn{5V}{-\pinendx}{1.5cm}%
	\pinasgn{GND_2}{-\pinendx}{1cm}%
	\pinasgn{GND_1}{-\pinendx}{.5cm}%
	\pinasgn{VIN}{-\pinendx}{-.5cm}%

	\pinasgn{A0}{-\pinendx}{-1cm}%
	\pinasgn{A1}{-\pinendx}{-1.5cm}%
	\pinasgn{A2}{-\pinendx}{-2cm}%
	\pinasgn{A3}{-\pinendx}{-2.5cm}%
	\pinasgn{A4}{-\pinendx}{-3cm}%
	\pinasgn{A5}{-\pinendx}{-3.5cm}%

	\pinasgn{AREF}{\pinendx}{3.5cm}%
	\pinasgn{D13}{\pinendx}{3cm}%
	\pinasgn{D12}{\pinendx}{2.5cm}%
	\pinasgn{D11}{\pinendx}{2cm}%
	\pinasgn{D10}{\pinendx}{1.5cm}%
	\pinasgn{D9}{\pinendx}{1cm}%
	\pinasgn{D8}{\pinendx}{0.5cm}%
	\pinasgn{D7}{\pinendx}{0cm}%
	\pinasgn{D6}{\pinendx}{-0.5cm}%
	\pinasgn{D5}{\pinendx}{-1cm}%
	\pinasgn{D4}{\pinendx}{-1.5cm}%
	\pinasgn{D3}{\pinendx}{-2cm}%
	\pinasgn{D2}{\pinendx}{-2.5cm}%
	\pinasgn{D1}{\pinendx}{-3cm}%
	\pinasgn{D0}{\pinendx}{-3.5cm}%

	\pinasgn{GND_D}{0cm}{-4.2cm-0.5cm}%

	% Border and Pin Names are drawn here
	\foregroundpath{
		%%% Container
		\pgfsetlinewidth{\boxsw}
		\pgfpathrectanglecorners
		{\pgfpoint{\ardwidth/2}{\ardheight/2-.2cm}}
		{\pgfpoint{-\ardwidth/2}{-\ardheight/2+.2cm}}
		\pgfusepath{draw}


		%%% pin array

		% power, io ref, rst
		\pgfsetlinewidth{\linesw}
		\pinarray{-\pinstartx}{-\pinendx}{0}{.5}{6}
		\linepoint{0cm}{-4.2cm}{0cm}{-4.7cm} % GND_D

		% Analog Input Pins
		\pinarray{-\pinstartx}{-\pinendx}{-3.5}{.5}{5}

		% Digital Input/Output Pins
		\pinarray{\pinstartx}{\pinendx}{-3.5}{.5}{14}
		\pgfusepath{draw}

		% Board Name
		\pgftext[at={\pgfpoint{0cm}{0.5cm}}]{\fontsize{14}{14} \textttbf{Arduino}}
		\pgftext[at={\pgfpoint{0}{0cm}}]{\fontsize{14}{14} \textttbf{UNO}}

		% power, io ref, reset pin labels
		\pinlabel{-\textmargin}{3cm}{IOREF}
		\pinlabel{-\textmargin}{2.5cm}{RESET}
		\pinlabel{-\textmargin}{2cm}{3V3}
		\pinlabel{-\textmargin}{1.5cm}{5V}
		\pinlabel{-\textmargin}{1cm}{GND}
		\pinlabel{-\textmargin}{.5cm}{GND}
		\pinlabel{-\textmargin}{0cm}{VIN}
		\pinlabel{0cm}{-4cm}{GND}

		% Analoge IO pin labels
		\foreach \i [evaluate=\i as \y using -1 - 0.5*\i] in {0,...,5} {
				\pinlabel{-\textmargin}{\y cm}{A\i}
			}

		% Digital IO pin labels
		\pinlabel{\textmargin}{3.5cm}{AREF}
		\pinlabel{\textmargin}{3cm}{D13}
		\pinlabel{\textmargin}{2.5cm}{D12}
		\pinlabel{\textmargin}{2cm}{\textasciitilde D11}
		\pinlabel{\textmargin}{1.5cm}{\textasciitilde D10}
		\pinlabel{\textmargin}{1cm}{\textasciitilde D9}
		\pinlabel{\textmargin}{.5cm}{D8}
		\pinlabel{\textmargin}{0cm}{D7}
		\pinlabel{\textmargin}{-.5cm}{\textasciitilde D6}
		\pinlabel{\textmargin}{-1cm}{\textasciitilde D5}
		\pinlabel{\textmargin}{-1.5cm}{D4}
		\pinlabel{\textmargin}{-2cm}{\textasciitilde D3}
		\pinlabel{\textmargin}{-2.5cm}{D2}
		\pinlabel{\textmargin}{-3cm}{\fixedsm $\blacktriangleright$\thinspace D1}
		\pinlabel{\textmargin}{-3.5cm}{\fixedsm $\blacktriangleleft$\thinspace D0}
	}%
}%
\usetikzlibrary{calc}%
\usetikzlibrary{intersections}%
\ctikzset{
	resistors/scale=0.6,
	capacitors/scale=0.6,
	diodes/scale=0.6,
	bipoles/thickness=1.2
}%

\tikzset{
	connect/.style args={(#1) to (#2) over (#3) by #4}{
			insert path={
					let \p1=($(#1)-(#3)$), \n1={veclen(\x1,\y1)},
					\n2={atan2(\y1,\x1)}, \n3={abs(#4)}, \n4={#4>0 ?180:-180}  in
					(#1) -- ($(#1)!\n1-\n3!(#3)$)
					arc (\n2:\n2+\n4:\n3) -- (#2)
				}
		},
}%
\newcommand{\ptrad}{1mm}%
\begin{circuitikz}[line width=\linesw]
	\draw
	node[arduino] (ard) at (0, 0){}
	(ard.A0)
	to [short, -*] ++(-3,0)
	to [R, l=5kΩ] ++(-3,0)
	to [C, l=100nF] ++(-.3,0)
	to [short, -o] ++(-.8,0)
	;

	\draw (ard.3V3)
	to [R, l=5kΩ] ++(-3,0)
	to [short, -*] ++(0,0) coordinate (itsc2)
	to [R, l_={100kΩ}] ++(0,-3.5)
	to [R, l_={100kΩ}] ++(0,-1.5)
	|- (ard.GND_D);

	\draw [name path=toAREF](itsc2)
	to ++(0, \ardheight/2-1cm)
	-| (ard.AREF)
	;

	\draw [name path=d6to5v] (ard.D6)
	to [short, -*] ++(.8cm, 0) coordinate (a)
	to [nopb] ++(2cm, 0)
	to ++(3.8cm, 0)
	to [short, -*] ++(0, 3cm)
	;

	\draw (a)
	to [R, l={10kΩ}] ++(0,-3)
	to ++(0, -1.2cm)
	to [short, -*] ++(0, 0)
	;

	\draw
	node[max7219] (mx) at (6cm, 2cm){}
	(mx.CLK)  -| ($(mx.CLK)! .5 !(ard.D13)$) |- (ard.D13)
	(ard.D11) -- (mx.Data)
	(ard.D10) -- (mx.CS)
	;

	\draw (ard.5V)
	let \p1 = (ard.5V) in
	to ++(-.5, 0)
	to ($(-.5cm, \ardabovemin) + (\x1, 0)$) coordinate (A);

	\path [name path=toMX5V]
	let
	\p1 = ($(mx.5V) + (.5cm, 0)$),
	\p2 = (A)
	in (A) -- (\x1, \y2) coordinate (B);

	\path [name path=MXGtoArG]
	let
	\p1 = (mx.GND),
	\p2 = (ard.GND_D)
	in (\p1) -- (\x1, \y2) coordinate (C);

	\path [name intersections={of=toAREF and toMX5V,by=interTop}];
	\path [name intersections={of=d6to5v and MXGtoArG,by=interRight}];
	\draw [connect=(A) to (B) over (interTop) by -4pt]
	-- ($(mx.5V) + (.5cm, 0)$) -- (mx.5V);
	\draw [connect=(mx.GND) to (C) over (interRight) by 4pt]
	-- (ard.GND_D);

	\fill (ard.GND_D) circle [radius=\ptrad];
	\draw (0, 8);
\end{circuitikz}
\end{document}