\documentclass{article}
\usepackage{amsmath}
\usepackage{amsfonts}

\begin{document}

Non-chromaticity of L\textsuperscript{th}

\section*{Temporal Coherence}
When there is no phase change at a point on the wave over a time. \(\langle \Delta \phi \rangle = 0\) (mean \(\Delta \phi\) of E-field)

\subsection*{Coherence Time}
Time over which a propagating wave remains coherent. \(\tau_c\) (major time interval over which the wave has definite phase relation). \\
For He-Ne:
\[
\tau_c = 2 \times 10^{-3} \approx 2 \text{ms}
\]
Na-lamp:
\[
\approx 10^{-15}
\]

\subsection*{Brightness/Intensity}
Laser produce highly intense beams, because more light energy is concentrated in a small region. Also laser light is coherent, so at a time many photons are in phase, they superimpose to produce a wave of larger amplitude. Hence resultant intensity (\(\alpha \text{Amp}^2\)) is very high.

\section*{Spontaneous \& Stimulated Emission}
\subsection*{Spontaneous Emission}
Atom initially at excited state makes transition voluntarily on its own, without aid of any external agency, to ground state and emits photon of energy \(h\nu_2 = E_2 - E_1\).

Different atoms of medium emits light at different times and different directions. Hence emitted photons are incoherent.

\subsection*{Stimulated Emission}
Here, photon having energy \(h\nu_2 = (E_2 - E_1)\) impinges (passes in the vicinity) on an atom present in its excited state and atom is stimulated to make transition to ground state and gives off a photon of energy \(h\nu_2\).

Emitted photon is in phase with the incident photon. These two travel in same direction and possess same frequency. They're coherent.

\subsection*{Einstein's Coefficients}
For a system containing atoms and radiation, ratio of atoms in ground state to atoms in excited state is:
\[
\frac{N_2}{N_1} = e^{\frac{E_1 - E_2}{kT}} = e^{\frac{-\Delta E}{kT}}
\]

\subsubsection*{If atom in ground state gets excited to higher state, absorption rate of radiation}
\[
B_{12} n(\nu) \rho_\nu \quad (\rho_\nu = \text{energy density of incident radiation})
\]
\begin{itemize}
    \item Rate of spontaneous emission: \(R_{21} = N_2 A_{21}\)
    \item Rate of absorption: \(R_{12} = N_1 B_{12} \rho_\nu\)
    \item Rate of stimulated emission: \(R'_{21} = N_2 B_{21} \rho_\nu\)
\end{itemize}

\end{document}
