\documentclass[12pt,a4paper]{article}
\usepackage[T1]{fontenc}
\usepackage[left=1.5cm, right=1cm, top=2cm, bottom=2cm]{geometry}
\usepackage{graphicx}
\usepackage{mathtools}
\usepackage{amssymb}
\usepackage{hyperref}
\usepackage{multicol}
\usepackage{minted}
\usepackage{tcolorbox}
\usepackage{framed}

\allowdisplaybreaks
\addtolength\jot{10pt}
\setlength{\parindent}{0pt}
\setlength{\belowdisplayskip}{30pt}
\setlength{\belowdisplayskip}{10pt}
\setlength{\fboxsep}{10pt}
\setlength{\columnsep}{30pt}

\renewcommand{\lg}{\log}
\renewcommand{\thesubsection}{\thesection.\alph{subsection}}
\renewcommand{\thesubsubsection}{\thesection.\alph{subsection}.\arabic{subsubsection}}

\newcommand{\MHz}{\text{MHz}}
\title{Week 4 Assignment}
\author{Roopesh O R}
\begin{document}
	\begin{center}
		{\LARGE Week 4 Assignment} \\[10pt]
		{Roopesh O R}
		\vspace{1cm}
	\end{center}
	\begin{multicols*}{2}
	
	\section{Question 1}
	Given specification:
	\begin{enumerate}
		\item The DC gain is 1.
		\item The gain at 1 MHz is at least 0.95.
		\item The gain at 2 MHz is at most $10^{-1}$
	\end{enumerate}
	
	\subsection{Finding filter parameters}
	\subsubsection{Minimum order needed}
	For specification 3: \\
	Let cutoff frequency be $f_c$, hence:
	
	
	\begin{flalign*}
		\left[\frac{1}{\sqrt{1 + \left(f/f_c\right) ^{2n}}}\right]_{f = 2\times10^6} &= 10^{-1} && \\
		\frac{1}{\sqrt{1 + \left(2\times10^6/f_c\right) ^{2n}}} &= 10^{-1} && \\	
		\left(2\times10^6/f_c\right) ^{2n} &= 10^2 - 1  && \\
		& = 99&& \\
		2n \left[\lg (2 \times 10^6) - \lg f_c\right] & = \lg 99 \tag{1}&&
	\end{flalign*}
	
	For specification 2: 
	\begin{flalign*}
		\left[\frac{1}{\sqrt{1 + \left(f/f_c\right) ^{2n}}}\right]_{f = 10^6} &= 0.95 && \\
		2n \left[\lg 10^6 - \lg f_c\right] & = \lg (1/0.95^2-1) \tag{2}&&
	\end{flalign*}
	
	Subtracting (2) from (1):
	
	\begin{flalign*}
		2n\lg 2 &= \lg 99 - \lg (1/0.95^2-1) && \\
		n &= \frac{\lg 99 - \lg (1/0.95^2-1)}{2\lg 2} = 4.9&&
	\end{flalign*}
	\setlength{\fboxsep}{10pt}
	\fbox{So Order of filter is $n=5$}
	
	\subsubsection{3dB bandwidth}
	
	Substituting n=5 in equation (1):
	
	\begin{flalign*}
		10 \left[\lg (2 \times 10^6) - \lg f_c\right] & = \lg 99 && \\
		\lg f_c = -\frac{\lg 99}{10} &+ \lg (2 \times 10^6) && \\
		\therefore f_c &= 1.263 \MHz &&
	\end{flalign*}
	\fbox{So 3dB bandwidth is 1.263 \MHz}
	
	\subsubsection{Magnitude response plot}
	\includegraphics*[width=\linewidth]{db.pdf}	
	Here orange line shows that passband criteria has met ($|H(j\omega)| \ge -0.22dB$).

\subsection{New order for tolerance in 3dB bandwidth}
If $f_c$ varies by $\pm 5$\% then the range of $f_c$ is from 1.199 MHz to 1.326 MHz

\textbf{For $f_c$ = 1.199 MHz:}

stop band:
\begin{flalign*}
	\frac{1}{\sqrt{1 + \left(\frac{2\times 10^6}{1.199\times 10^6}\right) ^{2n}}} &= 10^{-1} && \\
	n  &= \frac{\lg 99}{2\lg (2/1.199)}&& \\
	n &= 4.49
\end{flalign*}


pass band:
\begin{flalign*}
	\frac{1}{\sqrt{1 + \left(\frac{1\times 10^6}{1.199\times 10^6}\right) ^{2n}}} &= 0.95 && \\
	n  &= \frac{\lg (1/0.95^2 - 1)}{-2\lg (1.199)}&& \\
	\Aboxed{n &= 6.13}
\end{flalign*}


\textbf{For $f_c$ = 1.326 MHz:}

stop band:
\begin{flalign*}
	\frac{1}{\sqrt{1 + \left(\frac{2\times 10^6}{1.326\times 10^6}\right) ^{2n}}} &= 10^{-1} && \\
	n  &= \frac{\lg 99}{2\lg (2/1.326)}&& \\
	n &= 5.59
\end{flalign*}


pass band:
\begin{flalign*}
	\frac{1}{\sqrt{1 + \left(\frac{1\times 10^6}{1.326\times 10^6}\right) ^{2n}}} &= 0.95 && \\
	n  &= \frac{\lg (1/0.95^2 - 1)}{-2\lg (1.326)}&& \\
	n &= 3.94
\end{flalign*}
\begin{framed}
	Therefore order need to satisfy requirement for 5\% tolerance in $f_c$ is 7
\end{framed}
Magnitude response for $n=7$:

\includegraphics*[width=\linewidth]{n7.pdf}
Here we can see pass band and stop band is satisfied for given frequency deviation.

\includegraphics*[width=\linewidth]{n6.pdf}	
But if $n=6$, the pass band criteria is not satisfied for f = 1.199 MHz. Hence the minimum filter order is 7

\section{Question 2}
A high-Q bandpass filter the poles are located approximately at
$$s≈ -\frac{2Q}{\omega_0} \pm j\omega_0​$$
Hence the impulse response of the filter is approximately a damped oscillation with frequency $\omega_0$ :

\[h(t) \approx \frac{\omega_0}{Q} e^{-\frac{\omega_0}{2Q}t} \cos(\omega_0 t)\]
To maximize the output at any point in time, the input signal $v_i(t)$ should be such that convolution $h(t)*v_i(t)$ is maximum. For that to occur $v_i(t)$ should be a \textbf{square wave} whose sign at any instant is same as that of $h(t)$.

\includegraphics*[width=\linewidth]{2.pdf}

Hence the output at the peak becomes the integral of absolute of $h(t)$.

\begin{framed}
Hence the waveform $v_i(t)$ that maximizes the peak output is a \textbf{square wave with frequency $\omega_0$}.
\end{framed}


\subsection*{Peak value}

\begin{flalign*}
V_{\text{max}} &= \int_{0}^{\infty} |h(t)| dt && \\
 &= \int_{0}^{\infty} \left|\frac{\omega_0}{Q} e^{-\frac{\omega_0}{2Q}t} \cos(\omega_0 t)\right| dt &&
\end{flalign*}
Since the envelope $\exp\left(-\dfrac{\omega_0}{2Q}t\right)$ decays very slowly relative to the oscillation period (because $Q\gg 1$), we can approximate $∣\cos(\omega_0 ​t)∣$ by its average value over a period, which is $2/\pi$:

\begin{flalign*}
	V_{\text{max}} &\approx \frac{2\omega_0}{\pi Q}\int_{0}^{\infty} e^{-\frac{\omega_0}{2Q}t} dt && \\
	&= \frac{2\omega_0}{\pi Q} \frac{2Q}{\omega_0} && \\
	&= \frac{4}{\pi} \approx 1.273 &&
\end{flalign*}

\subsection*{Comparison with Sinusoidal Input}
If $v_i(t)$ is constrained to be sinusoidal with $|v_i| \le 1$:
$$v_i(t) = \cos(\omega_0 t)$$

The peak occurs at resonant frequency $\omega_0$, and magnitude is:
\begin{flalign*}
|H(j\omega_0)| &= \left|\frac{j\omega_0/(\omega_0 Q)}{(j\omega_0/\omega_0)^2 + j\omega_0/(\omega_0 Q) + 1}\right| && \\
&= \frac{j/Q}{-1 + j/Q + 1} = 1 && 
\end{flalign*}

\begin{framed}
Hence maximum peak voltage is 1 is $v_i$ is constrained to be sinusoidal and amplitude < 1, compared to 1.273 if $v_i$ is just amplitude constrained
\end{framed}

\section*{MATLAB codes}
\subsection*{1.a.3}
\begin{minted}{matlab}
f = .5e6:1e3:2.5e6;
n = 5;
fc = 1.263e6;
TF = 1./sqrt(1 + (f/fc).^(2*n));
plot(f, 10*log10(TF));

l1Mhz = 10*log10(0.95);
l2Mhz = 10*log10(0.1);

line([.5e6, 2.5e6], [l1Mhz, l1Mhz]);
line([.5e6, 2.5e6], [l2Mhz, l2Mhz]);
line([1e6, 1e6], [0, -15]);
line([2e6, 2e6], [0, -15]);
xlabel("f(MHz)");
ylabel("Gain (dB)")
\end{minted}

\subsection*{1.b}
\begin{minted}{matlab}
f = .5e6:1e3:2.5e6;
TF = @(fc, n) 1./sqrt(1 + (f/fc).^(2*n));
hold on;
n=6;
plot(f, TF(1.263e6, n));
plot(f, TF(1.199e6, n));
plot(f, TF(1.326e6, n));

line([0.5e6, 2.5e6], [0.95, 0.95]);
line([0.5e6, 2.5e6], [0.1, 0.1]);
line([1e6, 1e6], [1, 0]);
line([2e6, 2e6], [1, 0]);
hold off;
title("n="+n)
legend([ ...
"f_c = 1.263MHz", ...
"f_c = 1.199MHz", ...
"f_c = 1.326MHz" ...
]);
xlabel("f(MHz)"); ylabel("Gain")
\end{minted}

\subsection*{2}
\begin{minted}{matlab}
x = 0:.01:10;
w = 5;
y = 5*exp(-x/2).*cos(w*x);
plot(x, y);
hold on;
plot(x, sign(y));
legend(["h(t)", "v_i(t)"])
\end{minted}
\end{multicols*}
\end{document}