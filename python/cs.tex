\documentclass[11pt]{article}

    \usepackage[breakable]{tcolorbox}
    \usepackage{parskip} % Stop auto-indenting (to mimic markdown behaviour)
    

    % Basic figure setup, for now with no caption control since it's done
    % automatically by Pandoc (which extracts ![](path) syntax from Markdown).
    \usepackage{graphicx}
    % Keep aspect ratio if custom image width or height is specified
    \setkeys{Gin}{keepaspectratio}
    % Maintain compatibility with old templates. Remove in nbconvert 6.0
    \let\Oldincludegraphics\includegraphics
    % Ensure that by default, figures have no caption (until we provide a
    % proper Figure object with a Caption API and a way to capture that
    % in the conversion process - todo).
    \usepackage{caption}
    \DeclareCaptionFormat{nocaption}{}
    \captionsetup{format=nocaption,aboveskip=0pt,belowskip=0pt}

    \usepackage{float}
    \floatplacement{figure}{H} % forces figures to be placed at the correct location
    \usepackage{xcolor} % Allow colors to be defined
    \usepackage{enumerate} % Needed for markdown enumerations to work
    \usepackage{geometry} % Used to adjust the document margins
    \usepackage{amsmath} % Equations
    \usepackage{amssymb} % Equations
	\usepackage{svg}
    \usepackage{textcomp} % defines textquotesingle
    % Hack from http://tex.stackexchange.com/a/47451/13684:
    \AtBeginDocument{%
        \def\PYZsq{\textquotesingle}% Upright quotes in Pygmentized code
    }
    \usepackage{upquote} % Upright quotes for verbatim code
    \usepackage{eurosym} % defines \euro
	\usepackage{mathrsfs}
    \usepackage{iftex}
    \ifPDFTeX
        \usepackage[T1]{fontenc}
        \IfFileExists{alphabeta.sty}{
              \usepackage{alphabeta}
          }{
              \usepackage[mathletters]{ucs}
              \usepackage[utf8x]{inputenc}
          }
    \else
        \usepackage{fontspec}
        \usepackage{unicode-math}
    \fi

    \usepackage{fancyvrb} % verbatim replacement that allows latex
    \usepackage{grffile} % extends the file name processing of package graphics
                         % to support a larger range
    \makeatletter % fix for old versions of grffile with XeLaTeX
    \@ifpackagelater{grffile}{2019/11/01}
    {
      % Do nothing on new versions
    }
    {
      \def\Gread@@xetex#1{%
        \IfFileExists{"\Gin@base".bb}%
        {\Gread@eps{\Gin@base.bb}}%
        {\Gread@@xetex@aux#1}%
      }
    }
    \makeatother
    \usepackage[Export]{adjustbox} % Used to constrain images to a maximum size
    \adjustboxset{max size={0.9\linewidth}{0.9\paperheight}}

    % The hyperref package gives us a pdf with properly built
    % internal navigation ('pdf bookmarks' for the table of contents,
    % internal cross-reference links, web links for URLs, etc.)
    \usepackage{hyperref}
    % The default LaTeX title has an obnoxious amount of whitespace. By default,
    % titling removes some of it. It also provides customization options.
    \usepackage{titling}
    \usepackage{longtable} % longtable support required by pandoc >1.10
    \usepackage{booktabs}  % table support for pandoc > 1.12.2
    \usepackage{array}     % table support for pandoc >= 2.11.3
    \usepackage{calc}      % table minipage width calculation for pandoc >= 2.11.1
    \usepackage[inline]{enumitem} % IRkernel/repr support (it uses the enumerate* environment)
    \usepackage[normalem]{ulem} % ulem is needed to support strikethroughs (\sout)
                                % normalem makes italics be italics, not underlines
    \usepackage{soul}      % strikethrough (\st) support for pandoc >= 3.0.0
    \usepackage{mathrsfs}
    

    
    % Colors for the hyperref package
    \definecolor{urlcolor}{rgb}{0,.145,.698}
    \definecolor{linkcolor}{rgb}{.71,0.21,0.01}
    \definecolor{citecolor}{rgb}{.12,.54,.11}

    % ANSI colors
    \definecolor{ansi-black}{HTML}{3E424D}
    \definecolor{ansi-black-intense}{HTML}{282C36}
    \definecolor{ansi-red}{HTML}{E75C58}
    \definecolor{ansi-red-intense}{HTML}{B22B31}
    \definecolor{ansi-green}{HTML}{00A250}
    \definecolor{ansi-green-intense}{HTML}{007427}
    \definecolor{ansi-yellow}{HTML}{DDB62B}
    \definecolor{ansi-yellow-intense}{HTML}{B27D12}
    \definecolor{ansi-blue}{HTML}{208FFB}
    \definecolor{ansi-blue-intense}{HTML}{0065CA}
    \definecolor{ansi-magenta}{HTML}{D160C4}
    \definecolor{ansi-magenta-intense}{HTML}{A03196}
    \definecolor{ansi-cyan}{HTML}{60C6C8}
    \definecolor{ansi-cyan-intense}{HTML}{258F8F}
    \definecolor{ansi-white}{HTML}{C5C1B4}
    \definecolor{ansi-white-intense}{HTML}{A1A6B2}
    \definecolor{ansi-default-inverse-fg}{HTML}{FFFFFF}
    \definecolor{ansi-default-inverse-bg}{HTML}{000000}

    % common color for the border for error outputs.
    \definecolor{outerrorbackground}{HTML}{FFDFDF}

    % commands and environments needed by pandoc snippets
    % extracted from the output of `pandoc -s`
    \providecommand{\tightlist}{%
      \setlength{\itemsep}{0pt}\setlength{\parskip}{0pt}}
    \DefineVerbatimEnvironment{Highlighting}{Verbatim}{commandchars=\\\{\}}
    % Add ',fontsize=\small' for more characters per line
    \newenvironment{Shaded}{}{}
    \newcommand{\KeywordTok}[1]{\textcolor[rgb]{0.00,0.44,0.13}{\textbf{{#1}}}}
    \newcommand{\DataTypeTok}[1]{\textcolor[rgb]{0.56,0.13,0.00}{{#1}}}
    \newcommand{\DecValTok}[1]{\textcolor[rgb]{0.25,0.63,0.44}{{#1}}}
    \newcommand{\BaseNTok}[1]{\textcolor[rgb]{0.25,0.63,0.44}{{#1}}}
    \newcommand{\FloatTok}[1]{\textcolor[rgb]{0.25,0.63,0.44}{{#1}}}
    \newcommand{\CharTok}[1]{\textcolor[rgb]{0.25,0.44,0.63}{{#1}}}
    \newcommand{\StringTok}[1]{\textcolor[rgb]{0.25,0.44,0.63}{{#1}}}
    \newcommand{\CommentTok}[1]{\textcolor[rgb]{0.38,0.63,0.69}{\textit{{#1}}}}
    \newcommand{\OtherTok}[1]{\textcolor[rgb]{0.00,0.44,0.13}{{#1}}}
    \newcommand{\AlertTok}[1]{\textcolor[rgb]{1.00,0.00,0.00}{\textbf{{#1}}}}
    \newcommand{\FunctionTok}[1]{\textcolor[rgb]{0.02,0.16,0.49}{{#1}}}
    \newcommand{\RegionMarkerTok}[1]{{#1}}
    \newcommand{\ErrorTok}[1]{\textcolor[rgb]{1.00,0.00,0.00}{\textbf{{#1}}}}
    \newcommand{\NormalTok}[1]{{#1}}

    % Additional commands for more recent versions of Pandoc
    \newcommand{\ConstantTok}[1]{\textcolor[rgb]{0.53,0.00,0.00}{{#1}}}
    \newcommand{\SpecialCharTok}[1]{\textcolor[rgb]{0.25,0.44,0.63}{{#1}}}
    \newcommand{\VerbatimStringTok}[1]{\textcolor[rgb]{0.25,0.44,0.63}{{#1}}}
    \newcommand{\SpecialStringTok}[1]{\textcolor[rgb]{0.73,0.40,0.53}{{#1}}}
    \newcommand{\ImportTok}[1]{{#1}}
    \newcommand{\DocumentationTok}[1]{\textcolor[rgb]{0.73,0.13,0.13}{\textit{{#1}}}}
    \newcommand{\AnnotationTok}[1]{\textcolor[rgb]{0.38,0.63,0.69}{\textbf{\textit{{#1}}}}}
    \newcommand{\CommentVarTok}[1]{\textcolor[rgb]{0.38,0.63,0.69}{\textbf{\textit{{#1}}}}}
    \newcommand{\VariableTok}[1]{\textcolor[rgb]{0.10,0.09,0.49}{{#1}}}
    \newcommand{\ControlFlowTok}[1]{\textcolor[rgb]{0.00,0.44,0.13}{\textbf{{#1}}}}
    \newcommand{\OperatorTok}[1]{\textcolor[rgb]{0.40,0.40,0.40}{{#1}}}
    \newcommand{\BuiltInTok}[1]{{#1}}
    \newcommand{\ExtensionTok}[1]{{#1}}
    \newcommand{\PreprocessorTok}[1]{\textcolor[rgb]{0.74,0.48,0.00}{{#1}}}
    \newcommand{\AttributeTok}[1]{\textcolor[rgb]{0.49,0.56,0.16}{{#1}}}
    \newcommand{\InformationTok}[1]{\textcolor[rgb]{0.38,0.63,0.69}{\textbf{\textit{{#1}}}}}
    \newcommand{\WarningTok}[1]{\textcolor[rgb]{0.38,0.63,0.69}{\textbf{\textit{{#1}}}}}


    % Define a nice break command that doesn't care if a line doesn't already
    % exist.
    \def\br{\hspace*{\fill} \\* }
    % Math Jax compatibility definitions
    \def\gt{>}
    \def\lt{<}
    \let\Oldtex\TeX
    \let\Oldlatex\LaTeX
    \renewcommand{\TeX}{\textrm{\Oldtex}}
    \renewcommand{\LaTeX}{\textrm{\Oldlatex}}
    % Document parameters
    % Document title
    \title{Signal Classification System: Energy Vs. Power}
    
% Pygments definitions
\makeatletter
\def\PY@reset{\let\PY@it=\relax \let\PY@bf=\relax%
    \let\PY@ul=\relax \let\PY@tc=\relax%
    \let\PY@bc=\relax \let\PY@ff=\relax}
\def\PY@tok#1{\csname PY@tok@#1\endcsname}
\def\PY@toks#1+{\ifx\relax#1\empty\else%
    \PY@tok{#1}\expandafter\PY@toks\fi}
\def\PY@do#1{\PY@bc{\PY@tc{\PY@ul{%
    \PY@it{\PY@bf{\PY@ff{#1}}}}}}}
\def\PY#1#2{\PY@reset\PY@toks#1+\relax+\PY@do{#2}}

\@namedef{PY@tok@w}{\def\PY@tc##1{\textcolor[rgb]{0.73,0.73,0.73}{##1}}}
\@namedef{PY@tok@c}{\let\PY@it=\textit\def\PY@tc##1{\textcolor[rgb]{0.24,0.48,0.48}{##1}}}
\@namedef{PY@tok@cp}{\def\PY@tc##1{\textcolor[rgb]{0.61,0.40,0.00}{##1}}}
\@namedef{PY@tok@k}{\let\PY@bf=\textbf\def\PY@tc##1{\textcolor[rgb]{0.00,0.50,0.00}{##1}}}
\@namedef{PY@tok@kp}{\def\PY@tc##1{\textcolor[rgb]{0.00,0.50,0.00}{##1}}}
\@namedef{PY@tok@kt}{\def\PY@tc##1{\textcolor[rgb]{0.69,0.00,0.25}{##1}}}
\@namedef{PY@tok@o}{\def\PY@tc##1{\textcolor[rgb]{0.40,0.40,0.40}{##1}}}
\@namedef{PY@tok@ow}{\let\PY@bf=\textbf\def\PY@tc##1{\textcolor[rgb]{0.67,0.13,1.00}{##1}}}
\@namedef{PY@tok@nb}{\def\PY@tc##1{\textcolor[rgb]{0.00,0.50,0.00}{##1}}}
\@namedef{PY@tok@nf}{\def\PY@tc##1{\textcolor[rgb]{0.00,0.00,1.00}{##1}}}
\@namedef{PY@tok@nc}{\let\PY@bf=\textbf\def\PY@tc##1{\textcolor[rgb]{0.00,0.00,1.00}{##1}}}
\@namedef{PY@tok@nn}{\let\PY@bf=\textbf\def\PY@tc##1{\textcolor[rgb]{0.00,0.00,1.00}{##1}}}
\@namedef{PY@tok@ne}{\let\PY@bf=\textbf\def\PY@tc##1{\textcolor[rgb]{0.80,0.25,0.22}{##1}}}
\@namedef{PY@tok@nv}{\def\PY@tc##1{\textcolor[rgb]{0.10,0.09,0.49}{##1}}}
\@namedef{PY@tok@no}{\def\PY@tc##1{\textcolor[rgb]{0.53,0.00,0.00}{##1}}}
\@namedef{PY@tok@nl}{\def\PY@tc##1{\textcolor[rgb]{0.46,0.46,0.00}{##1}}}
\@namedef{PY@tok@ni}{\let\PY@bf=\textbf\def\PY@tc##1{\textcolor[rgb]{0.44,0.44,0.44}{##1}}}
\@namedef{PY@tok@na}{\def\PY@tc##1{\textcolor[rgb]{0.41,0.47,0.13}{##1}}}
\@namedef{PY@tok@nt}{\let\PY@bf=\textbf\def\PY@tc##1{\textcolor[rgb]{0.00,0.50,0.00}{##1}}}
\@namedef{PY@tok@nd}{\def\PY@tc##1{\textcolor[rgb]{0.67,0.13,1.00}{##1}}}
\@namedef{PY@tok@s}{\def\PY@tc##1{\textcolor[rgb]{0.73,0.13,0.13}{##1}}}
\@namedef{PY@tok@sd}{\let\PY@it=\textit\def\PY@tc##1{\textcolor[rgb]{0.73,0.13,0.13}{##1}}}
\@namedef{PY@tok@si}{\let\PY@bf=\textbf\def\PY@tc##1{\textcolor[rgb]{0.64,0.35,0.47}{##1}}}
\@namedef{PY@tok@se}{\let\PY@bf=\textbf\def\PY@tc##1{\textcolor[rgb]{0.67,0.36,0.12}{##1}}}
\@namedef{PY@tok@sr}{\def\PY@tc##1{\textcolor[rgb]{0.64,0.35,0.47}{##1}}}
\@namedef{PY@tok@ss}{\def\PY@tc##1{\textcolor[rgb]{0.10,0.09,0.49}{##1}}}
\@namedef{PY@tok@sx}{\def\PY@tc##1{\textcolor[rgb]{0.00,0.50,0.00}{##1}}}
\@namedef{PY@tok@m}{\def\PY@tc##1{\textcolor[rgb]{0.40,0.40,0.40}{##1}}}
\@namedef{PY@tok@gh}{\let\PY@bf=\textbf\def\PY@tc##1{\textcolor[rgb]{0.00,0.00,0.50}{##1}}}
\@namedef{PY@tok@gu}{\let\PY@bf=\textbf\def\PY@tc##1{\textcolor[rgb]{0.50,0.00,0.50}{##1}}}
\@namedef{PY@tok@gd}{\def\PY@tc##1{\textcolor[rgb]{0.63,0.00,0.00}{##1}}}
\@namedef{PY@tok@gi}{\def\PY@tc##1{\textcolor[rgb]{0.00,0.52,0.00}{##1}}}
\@namedef{PY@tok@gr}{\def\PY@tc##1{\textcolor[rgb]{0.89,0.00,0.00}{##1}}}
\@namedef{PY@tok@ge}{\let\PY@it=\textit}
\@namedef{PY@tok@gs}{\let\PY@bf=\textbf}
\@namedef{PY@tok@ges}{\let\PY@bf=\textbf\let\PY@it=\textit}
\@namedef{PY@tok@gp}{\let\PY@bf=\textbf\def\PY@tc##1{\textcolor[rgb]{0.00,0.00,0.50}{##1}}}
\@namedef{PY@tok@go}{\def\PY@tc##1{\textcolor[rgb]{0.44,0.44,0.44}{##1}}}
\@namedef{PY@tok@gt}{\def\PY@tc##1{\textcolor[rgb]{0.00,0.27,0.87}{##1}}}
\@namedef{PY@tok@err}{\def\PY@bc##1{{\setlength{\fboxsep}{\string -\fboxrule}\fcolorbox[rgb]{1.00,0.00,0.00}{1,1,1}{\strut ##1}}}}
\@namedef{PY@tok@kc}{\let\PY@bf=\textbf\def\PY@tc##1{\textcolor[rgb]{0.00,0.50,0.00}{##1}}}
\@namedef{PY@tok@kd}{\let\PY@bf=\textbf\def\PY@tc##1{\textcolor[rgb]{0.00,0.50,0.00}{##1}}}
\@namedef{PY@tok@kn}{\let\PY@bf=\textbf\def\PY@tc##1{\textcolor[rgb]{0.00,0.50,0.00}{##1}}}
\@namedef{PY@tok@kr}{\let\PY@bf=\textbf\def\PY@tc##1{\textcolor[rgb]{0.00,0.50,0.00}{##1}}}
\@namedef{PY@tok@bp}{\def\PY@tc##1{\textcolor[rgb]{0.00,0.50,0.00}{##1}}}
\@namedef{PY@tok@fm}{\def\PY@tc##1{\textcolor[rgb]{0.00,0.00,1.00}{##1}}}
\@namedef{PY@tok@vc}{\def\PY@tc##1{\textcolor[rgb]{0.10,0.09,0.49}{##1}}}
\@namedef{PY@tok@vg}{\def\PY@tc##1{\textcolor[rgb]{0.10,0.09,0.49}{##1}}}
\@namedef{PY@tok@vi}{\def\PY@tc##1{\textcolor[rgb]{0.10,0.09,0.49}{##1}}}
\@namedef{PY@tok@vm}{\def\PY@tc##1{\textcolor[rgb]{0.10,0.09,0.49}{##1}}}
\@namedef{PY@tok@sa}{\def\PY@tc##1{\textcolor[rgb]{0.73,0.13,0.13}{##1}}}
\@namedef{PY@tok@sb}{\def\PY@tc##1{\textcolor[rgb]{0.73,0.13,0.13}{##1}}}
\@namedef{PY@tok@sc}{\def\PY@tc##1{\textcolor[rgb]{0.73,0.13,0.13}{##1}}}
\@namedef{PY@tok@dl}{\def\PY@tc##1{\textcolor[rgb]{0.73,0.13,0.13}{##1}}}
\@namedef{PY@tok@s2}{\def\PY@tc##1{\textcolor[rgb]{0.73,0.13,0.13}{##1}}}
\@namedef{PY@tok@sh}{\def\PY@tc##1{\textcolor[rgb]{0.73,0.13,0.13}{##1}}}
\@namedef{PY@tok@s1}{\def\PY@tc##1{\textcolor[rgb]{0.73,0.13,0.13}{##1}}}
\@namedef{PY@tok@mb}{\def\PY@tc##1{\textcolor[rgb]{0.40,0.40,0.40}{##1}}}
\@namedef{PY@tok@mf}{\def\PY@tc##1{\textcolor[rgb]{0.40,0.40,0.40}{##1}}}
\@namedef{PY@tok@mh}{\def\PY@tc##1{\textcolor[rgb]{0.40,0.40,0.40}{##1}}}
\@namedef{PY@tok@mi}{\def\PY@tc##1{\textcolor[rgb]{0.40,0.40,0.40}{##1}}}
\@namedef{PY@tok@il}{\def\PY@tc##1{\textcolor[rgb]{0.40,0.40,0.40}{##1}}}
\@namedef{PY@tok@mo}{\def\PY@tc##1{\textcolor[rgb]{0.40,0.40,0.40}{##1}}}
\@namedef{PY@tok@ch}{\let\PY@it=\textit\def\PY@tc##1{\textcolor[rgb]{0.24,0.48,0.48}{##1}}}
\@namedef{PY@tok@cm}{\let\PY@it=\textit\def\PY@tc##1{\textcolor[rgb]{0.24,0.48,0.48}{##1}}}
\@namedef{PY@tok@cpf}{\let\PY@it=\textit\def\PY@tc##1{\textcolor[rgb]{0.24,0.48,0.48}{##1}}}
\@namedef{PY@tok@c1}{\let\PY@it=\textit\def\PY@tc##1{\textcolor[rgb]{0.24,0.48,0.48}{##1}}}
\@namedef{PY@tok@cs}{\let\PY@it=\textit\def\PY@tc##1{\textcolor[rgb]{0.24,0.48,0.48}{##1}}}

\def\PYZbs{\char`\\}
\def\PYZus{\char`\_}
\def\PYZob{\char`\{}
\def\PYZcb{\char`\}}
\def\PYZca{\char`\^}
\def\PYZam{\char`\&}
\def\PYZlt{\char`\<}
\def\PYZgt{\char`\>}
\def\PYZsh{\char`\#}
\def\PYZpc{\char`\%}
\def\PYZdl{\char`\$}
\def\PYZhy{\char`\-}
\def\PYZsq{\char`\'}
\def\PYZdq{\char`\"}
\def\PYZti{\char`\~}
% for compatibility with earlier versions
\def\PYZat{@}
\def\PYZlb{[}
\def\PYZrb{]}
\makeatother


    % For linebreaks inside Verbatim environment from package fancyvrb.
    \makeatletter
        \newbox\Wrappedcontinuationbox
        \newbox\Wrappedvisiblespacebox
        \newcommand*\Wrappedvisiblespace {\textcolor{red}{\textvisiblespace}}
        \newcommand*\Wrappedcontinuationsymbol {\textcolor{red}{\llap{\tiny$\m@th\hookrightarrow$}}}
        \newcommand*\Wrappedcontinuationindent {3ex }
        \newcommand*\Wrappedafterbreak {\kern\Wrappedcontinuationindent\copy\Wrappedcontinuationbox}
        % Take advantage of the already applied Pygments mark-up to insert
        % potential linebreaks for TeX processing.
        %        {, <, #, %, $, ' and ": go to next line.
        %        _, }, ^, &, >, - and ~: stay at end of broken line.
        % Use of \textquotesingle for straight quote.
        \newcommand*\Wrappedbreaksatspecials {%
            \def\PYGZus{\discretionary{\char`\_}{\Wrappedafterbreak}{\char`\_}}%
            \def\PYGZob{\discretionary{}{\Wrappedafterbreak\char`\{}{\char`\{}}%
            \def\PYGZcb{\discretionary{\char`\}}{\Wrappedafterbreak}{\char`\}}}%
            \def\PYGZca{\discretionary{\char`\^}{\Wrappedafterbreak}{\char`\^}}%
            \def\PYGZam{\discretionary{\char`\&}{\Wrappedafterbreak}{\char`\&}}%
            \def\PYGZlt{\discretionary{}{\Wrappedafterbreak\char`\<}{\char`\<}}%
            \def\PYGZgt{\discretionary{\char`\>}{\Wrappedafterbreak}{\char`\>}}%
            \def\PYGZsh{\discretionary{}{\Wrappedafterbreak\char`\#}{\char`\#}}%
            \def\PYGZpc{\discretionary{}{\Wrappedafterbreak\char`\%}{\char`\%}}%
            \def\PYGZdl{\discretionary{}{\Wrappedafterbreak\char`\$}{\char`\$}}%
            \def\PYGZhy{\discretionary{\char`\-}{\Wrappedafterbreak}{\char`\-}}%
            \def\PYGZsq{\discretionary{}{\Wrappedafterbreak\textquotesingle}{\textquotesingle}}%
            \def\PYGZdq{\discretionary{}{\Wrappedafterbreak\char`\"}{\char`\"}}%
            \def\PYGZti{\discretionary{\char`\~}{\Wrappedafterbreak}{\char`\~}}%
        }
        % Some characters . , ; ? ! / are not pygmentized.
        % This macro makes them "active" and they will insert potential linebreaks
        \newcommand*\Wrappedbreaksatpunct {%
            \lccode`\~`\.\lowercase{\def~}{\discretionary{\hbox{\char`\.}}{\Wrappedafterbreak}{\hbox{\char`\.}}}%
            \lccode`\~`\,\lowercase{\def~}{\discretionary{\hbox{\char`\,}}{\Wrappedafterbreak}{\hbox{\char`\,}}}%
            \lccode`\~`\;\lowercase{\def~}{\discretionary{\hbox{\char`\;}}{\Wrappedafterbreak}{\hbox{\char`\;}}}%
            \lccode`\~`\:\lowercase{\def~}{\discretionary{\hbox{\char`\:}}{\Wrappedafterbreak}{\hbox{\char`\:}}}%
            \lccode`\~`\?\lowercase{\def~}{\discretionary{\hbox{\char`\?}}{\Wrappedafterbreak}{\hbox{\char`\?}}}%
            \lccode`\~`\!\lowercase{\def~}{\discretionary{\hbox{\char`\!}}{\Wrappedafterbreak}{\hbox{\char`\!}}}%
            \lccode`\~`\/\lowercase{\def~}{\discretionary{\hbox{\char`\/}}{\Wrappedafterbreak}{\hbox{\char`\/}}}%
            \catcode`\.\active
            \catcode`\,\active
            \catcode`\;\active
            \catcode`\:\active
            \catcode`\?\active
            \catcode`\!\active
            \catcode`\/\active
            \lccode`\~`\~
        }
    \makeatother

    \let\OriginalVerbatim=\Verbatim
    \makeatletter
    \renewcommand{\Verbatim}[1][1]{%
        %\parskip\z@skip
        \sbox\Wrappedcontinuationbox {\Wrappedcontinuationsymbol}%
        \sbox\Wrappedvisiblespacebox {\FV@SetupFont\Wrappedvisiblespace}%
        \def\FancyVerbFormatLine ##1{\hsize\linewidth
            \vtop{\raggedright\hyphenpenalty\z@\exhyphenpenalty\z@
                \doublehyphendemerits\z@\finalhyphendemerits\z@
                \strut ##1\strut}%
        }%
        % If the linebreak is at a space, the latter will be displayed as visible
        % space at end of first line, and a continuation symbol starts next line.
        % Stretch/shrink are however usually zero for typewriter font.
        \def\FV@Space {%
            \nobreak\hskip\z@ plus\fontdimen3\font minus\fontdimen4\font
            \discretionary{\copy\Wrappedvisiblespacebox}{\Wrappedafterbreak}
            {\kern\fontdimen2\font}%
        }%

        % Allow breaks at special characters using \PYG... macros.
        \Wrappedbreaksatspecials
        % Breaks at punctuation characters . , ; ? ! and / need catcode=\active
        \OriginalVerbatim[#1,codes*=\Wrappedbreaksatpunct]%
    }
    \makeatother

    % Exact colors from NB
    \definecolor{incolor}{HTML}{303F9F}
    \definecolor{outcolor}{HTML}{D84315}
    \definecolor{cellborder}{HTML}{CFCFCF}
    \definecolor{cellbackground}{HTML}{FAFAFA}

    % prompt
    \makeatletter
    \newcommand{\boxspacing}{\kern\kvtcb@left@rule\kern\kvtcb@boxsep}
    \makeatother
    \newcommand{\prompt}[4]{
        {\ttfamily\llap{{\color{#2}[#3]:\hspace{3pt}#4}}\vspace{-\baselineskip}}
    }
    

    
    % Prevent overflowing lines due to hard-to-break entities
    \sloppy
    % Setup hyperref package
    \hypersetup{
      breaklinks=true,  % so long urls are correctly broken across lines
      colorlinks=true,
      urlcolor=urlcolor,
      linkcolor=linkcolor,
      citecolor=citecolor,
      }
    % Slightly bigger margins than the latex defaults
    
    
    

\begin{document}
	
\begin{titlepage}
	\newgeometry{top=2.5in, bottom=1in, left=1in, right=1in}
	\centering

	{\Large \textbf{Python Project:} \par}
	\vspace{.3cm}
	{\huge \textbf{Signal Classification System:\\ Energy Vs. Power} \par}

	\vspace{1.5cm}
	
	\includesvg[inkscapearea=drawing, width=1.8in]{cusat.svg}\par
	
	\vspace{1.5cm}
	
	{\large Submitted by \par}
	{\large \textbf{\underline {Group - 5}}\par}
	{\large \textbf{Roopesh O R}\par}
	{\large \textbf{Roshna Palatty Santhosh}\par}
	{\large \textbf{Merella Jobi}\par}
	% \vspace{.1cm}

	% {\textbf{Division of Electronics Engineering} \par}
	% {\textbf{School of Engineering} \par}
	% {\textbf{Cochin University of Science and Technology} \par}
	% {\textbf{Kochi - 682022} \par}

	\vfill

\end{titlepage}

\newgeometry{top=.8in, bottom=.9in, left=.9in, right=.9in}

    \maketitle
    
    \section*{Aim}

The aim is to classify a given real-valued signal into energy signal
or power signal, or neither of them, using numeric integration, and also
to find their energy and power. The signal may be given as a Python
function, and the test is done by calling the classifier function.

\section*{Tools used}

\begin{itemize}
\tightlist
\item
  Jupyter notebook
\item
  \texttt{math} module
\item
  \texttt{inspect} module (for testing purpose)
\end{itemize}

\section*{Method of classification}

The given signal's energy (\(E\)) and power (\(P\)) are computed using
definite integrals: \[ E = \int_{-L} ^ L f^2(x) dt \]
\[P = \frac{1}{2L}\int_{-L} ^ L f^2(x) dx\]

with a large upper limit of \(L\)

The integration is computed numerically using the Trapezoidal rule
(https://en.wikipedia.org/wiki/Trapezoidal\_rule\#Uniform\_grid).

To classify the signal into either category, the integral has to be
evaluated and checked if it converges to a non-zero value. The signal is
energy signal if \(0 < E < \infty\) and it is power signal if
\(0 < P < \infty\)

To test for the convergence of integral, integrals for energy and power
are computed with various increasing upper limits \(L\)
(\(L = 10^2, 10^3, \cdots 10^6\)). From this, a sequence containing
differences between each successive limit is obtained and checked if it
tends to decrease. Additionally, to account for any rounding error which
may result in an oscillatory pattern in the sequence of differences, it
is also checked whether each difference is within the average of the
whole difference sequence.

{ \textbf {\Large Numeric integration }}

In the program, numeric integration is performed with the help of a
wrapper function \(e(x)\). This wrapper function evaluates the signal
function at both positive and negative input values and also checks for
any overflows or zero divisions and handles them appropriately:

\[ e(x) = 
\begin{cases}
    \infty &, \text{if } f^2(x) \text{ or } f^2(-x) \text{ overflows or } \text{ it results in zero division}\\
    f(x) ^ 2 dx + f(-x) ^ 2 dx &, \text{otherwise}
\end{cases}
\] Here, stepsize \(dx\) is choosen to 0.1

The summation is done by:
\[ E = \frac{e(0) + e(L)}{2} + \sum_{dx \le x \le L-dx} e(x) \]

\[ P = \frac{e(0) + e(L)}{2T} + \sum_{dx \le x \le L-dx} \frac{e(x)}{T} \]

Where \(T = 2L\)

For readability, in the program \texttt{end} is used to denote \(L\) and
\texttt{duration} is used to denote \(T\)

    

    \section*{Source code}

    \begin{tcolorbox}[breakable, size=fbox, boxrule=1pt, pad at break*=1mm,colback=cellbackground, colframe=cellborder]
\prompt{In}{incolor}{1}{\boxspacing}
\begin{Verbatim}[commandchars=\\\{\}]
\PY{k+kn}{from} \PY{n+nn}{math} \PY{k+kn}{import} \PY{n}{inf}
\PY{k}{def} \PY{n+nf}{compute\PYZus{}E\PYZus{}P}\PY{p}{(}\PY{n}{f}\PY{p}{,} \PY{n}{end}\PY{p}{,} \PY{n}{dx}\PY{p}{)}\PY{p}{:}
\PY{+w}{    }\PY{l+s+sd}{\PYZdq{}\PYZdq{}\PYZdq{}}
\PY{l+s+sd}{    Computes energy and power of given signal f and integrates}
\PY{l+s+sd}{    it from \PYZhy{}end to end with step size dx}
\PY{l+s+sd}{    \PYZdq{}\PYZdq{}\PYZdq{}}
    \PY{n}{duration}  \PY{o}{=} \PY{n}{end} \PY{o}{*} \PY{l+m+mi}{2}
    \PY{n}{power\PYZus{}overflown} \PY{o}{=} \PY{k+kc}{False}
    
    \PY{k}{def} \PY{n+nf}{e}\PY{p}{(}\PY{n}{x}\PY{p}{)}\PY{p}{:}
\PY{+w}{        }\PY{l+s+sd}{\PYZdq{}\PYZdq{}\PYZdq{} wrapper function to compute integrants.}
\PY{l+s+sd}{        It also check overlfow and zero division}
\PY{l+s+sd}{        \PYZdq{}\PYZdq{}\PYZdq{}}
        \PY{k}{try}\PY{p}{:} \PY{k}{return} \PY{n}{f}\PY{p}{(}\PY{n}{x}\PY{p}{)} \PY{o}{*}\PY{o}{*} \PY{l+m+mi}{2} \PY{o}{*} \PY{n}{dx} \PY{o}{+} \PY{n}{f}\PY{p}{(}\PY{o}{\PYZhy{}}\PY{n}{x}\PY{p}{)} \PY{o}{*}\PY{o}{*} \PY{l+m+mi}{2} \PY{o}{*} \PY{n}{dx}
        \PY{k}{except} \PY{p}{(}\PY{n+ne}{OverflowError}\PY{p}{,} \PY{n+ne}{ZeroDivisionError}\PY{p}{)}\PY{p}{:}
            \PY{k}{return} \PY{n}{inf}
    
    \PY{c+c1}{\PYZsh{} using trapezoidal rule to integerate}
    
    \PY{c+c1}{\PYZsh{} add initial parts}
    \PY{c+c1}{\PYZsh{} e(0) is divided by 2 to remove duplicate summation}
    \PY{n}{E} \PY{o}{=} \PY{p}{(}\PY{n}{e}\PY{p}{(}\PY{l+m+mi}{0}\PY{p}{)} \PY{o}{+} \PY{n}{e}\PY{p}{(}\PY{n}{end}\PY{p}{)}\PY{p}{)} \PY{o}{/} \PY{l+m+mi}{2}
    \PY{n}{P} \PY{o}{=} \PY{n}{E} \PY{o}{/} \PY{n}{duration}
    
    \PY{n}{x} \PY{o}{=} \PY{n}{dx}
    \PY{k}{while} \PY{p}{(}
        \PY{n}{x} \PY{o}{\PYZlt{}} \PY{n}{end} \PY{o+ow}{and}
        \PY{n}{P} \PY{o}{!=} \PY{n}{inf} \PY{c+c1}{\PYZsh{} check if power has overflown, if so break the loop}
    \PY{p}{)}\PY{p}{:}
        \PY{n}{i} \PY{o}{=} \PY{n}{e}\PY{p}{(}\PY{n}{x}\PY{p}{)}
        \PY{n}{E} \PY{o}{+}\PY{o}{=} \PY{n}{i}
        \PY{n}{P} \PY{o}{+}\PY{o}{=} \PY{n}{i} \PY{o}{/} \PY{n}{duration}
        
        \PY{n}{x} \PY{o}{+}\PY{o}{=} \PY{n}{dx}
    
    \PY{k}{return} \PY{p}{(}\PY{n}{E}\PY{p}{,} \PY{n}{P}\PY{p}{)}

\PY{k}{def} \PY{n+nf}{is\PYZus{}converging\PYZus{}nonzero}\PY{p}{(}\PY{n}{sequence}\PY{p}{)}\PY{p}{:}
\PY{+w}{    }\PY{l+s+sd}{\PYZdq{}\PYZdq{}\PYZdq{}}
\PY{l+s+sd}{        returns true if sequence tend to approach a}
\PY{l+s+sd}{        value that is not zero or infinity}
\PY{l+s+sd}{    \PYZdq{}\PYZdq{}\PYZdq{}}
    
    \PY{c+c1}{\PYZsh{} Check if sequence contains zero or infinity}
    \PY{k}{if} \PY{l+m+mi}{0} \PY{o+ow}{in} \PY{n}{sequence} \PY{o+ow}{or} \PY{n}{inf} \PY{o+ow}{in} \PY{n}{sequence}\PY{p}{:} \PY{k}{return} \PY{k+kc}{False}
\PY{+w}{    }
\PY{+w}{    }\PY{l+s+sd}{\PYZdq{}\PYZdq{}\PYZdq{}}
\PY{l+s+sd}{    Check if function stabilizes to some value}
\PY{l+s+sd}{    This is done by finding absolute differences of successive}
\PY{l+s+sd}{    elements and checking whether each difference is smaller than}
\PY{l+s+sd}{    the previous difference.}
\PY{l+s+sd}{    \PYZdq{}\PYZdq{}\PYZdq{}}
    \PY{n}{diffs} \PY{o}{=} \PY{p}{[}
        \PY{n+nb}{abs}\PY{p}{(}\PY{n}{sequence}\PY{p}{[}\PY{n}{i}\PY{o}{\PYZhy{}}\PY{l+m+mi}{1}\PY{p}{]} \PY{o}{\PYZhy{}} \PY{n}{sequence}\PY{p}{[}\PY{n}{i}\PY{p}{]}\PY{p}{)} \PY{k}{for} \PY{n}{i} \PY{o+ow}{in} \PY{n+nb}{range}\PY{p}{(}\PY{l+m+mi}{1}\PY{p}{,} \PY{n+nb}{len}\PY{p}{(}\PY{n}{sequence}\PY{p}{)}\PY{p}{)}
    \PY{p}{]}
    \PY{c+c1}{\PYZsh{} and their average}
    \PY{n}{avg\PYZus{}diff} \PY{o}{=} \PY{n+nb}{sum}\PY{p}{(}\PY{n}{diffs}\PY{p}{)} \PY{o}{/} \PY{n+nb}{len}\PY{p}{(}\PY{n}{diffs}\PY{p}{)} 
\PY{+w}{    }
\PY{+w}{    }\PY{l+s+sd}{\PYZdq{}\PYZdq{}\PYZdq{}}
\PY{l+s+sd}{    Here the sequence is tested for instability.}
\PY{l+s+sd}{    The sequence considered unstable if any of difference is larger}
\PY{l+s+sd}{    than the previous one and the difference is larger than the}
\PY{l+s+sd}{    average of all differences.}
\PY{l+s+sd}{    }
\PY{l+s+sd}{    The condition is checked to ensure that sequence is not}
\PY{l+s+sd}{    marked unstable due to some roundoff error in summation.}
\PY{l+s+sd}{    \PYZdq{}\PYZdq{}\PYZdq{}}
    \PY{k}{for} \PY{n}{i} \PY{o+ow}{in} \PY{n+nb}{range}\PY{p}{(}\PY{l+m+mi}{1}\PY{p}{,} \PY{n+nb}{len}\PY{p}{(}\PY{n}{diffs}\PY{p}{)}\PY{p}{)}\PY{p}{:}
        \PY{k}{if} \PY{p}{(}\PY{n}{diffs}\PY{p}{[}\PY{n}{i}\PY{o}{\PYZhy{}}\PY{l+m+mi}{1}\PY{p}{]} \PY{o}{\PYZlt{}} \PY{n}{diffs}\PY{p}{[}\PY{n}{i}\PY{p}{]} \PY{o+ow}{and} \PY{n}{diffs}\PY{p}{[}\PY{n}{i}\PY{p}{]} \PY{o}{\PYZgt{}} \PY{n}{avg\PYZus{}diff}\PY{p}{)}\PY{p}{:}
            \PY{k}{return} \PY{k+kc}{False}
    \PY{k}{return} \PY{k+kc}{True}

\PY{k}{def} \PY{n+nf}{classify\PYZus{}signal}\PY{p}{(}\PY{n}{f}\PY{p}{)}\PY{p}{:}
\PY{+w}{    }\PY{l+s+sd}{\PYZdq{}\PYZdq{}\PYZdq{}}
\PY{l+s+sd}{    Function that does the final classification}
\PY{l+s+sd}{    Classification is done by finding integrals from \PYZhy{}10\PYZca{}2 to 10\PYZca{}2,}
\PY{l+s+sd}{    \PYZhy{}10\PYZca{}3 to 10\PYZca{}3, upto \PYZhy{}10\PYZca{}6 to 10\PYZca{}6 and checking if they}
\PY{l+s+sd}{    converge to some value (not equal to 0)}
\PY{l+s+sd}{    \PYZdq{}\PYZdq{}\PYZdq{}}
    \PY{n}{Energies} \PY{o}{=} \PY{p}{[}\PY{p}{]}
    \PY{n}{Powers} \PY{o}{=} \PY{p}{[}\PY{p}{]}
    \PY{k}{for} \PY{n}{i} \PY{o+ow}{in} \PY{n+nb}{range}\PY{p}{(}\PY{l+m+mi}{2}\PY{p}{,} \PY{l+m+mi}{7}\PY{p}{)}\PY{p}{:}
        \PY{n}{EP} \PY{o}{=} \PY{n}{compute\PYZus{}E\PYZus{}P}\PY{p}{(}\PY{n}{f}\PY{p}{,} \PY{l+m+mi}{10}\PY{o}{*}\PY{o}{*}\PY{n}{i}\PY{p}{,} \PY{l+m+mf}{0.1}\PY{p}{)}
        \PY{n}{Energies}\PY{o}{.}\PY{n}{append}\PY{p}{(}\PY{n}{EP}\PY{p}{[}\PY{l+m+mi}{0}\PY{p}{]}\PY{p}{)}
        \PY{n}{Powers}\PY{o}{.}\PY{n}{append}\PY{p}{(}\PY{n}{EP}\PY{p}{[}\PY{l+m+mi}{1}\PY{p}{]}\PY{p}{)}

    \PY{k}{if} \PY{n}{is\PYZus{}converging\PYZus{}nonzero}\PY{p}{(}\PY{n}{Energies}\PY{p}{)}\PY{p}{:}
        \PY{n+nb}{print}\PY{p}{(}\PY{l+s+s2}{\PYZdq{}}\PY{l+s+s2}{Energy signal, energy:}\PY{l+s+s2}{\PYZdq{}}\PY{p}{,} \PY{n}{Energies}\PY{p}{[}\PY{o}{\PYZhy{}}\PY{l+m+mi}{1}\PY{p}{]}\PY{p}{)}
    \PY{k}{elif} \PY{n}{is\PYZus{}converging\PYZus{}nonzero}\PY{p}{(}\PY{n}{Powers}\PY{p}{)}\PY{p}{:}
        \PY{n+nb}{print}\PY{p}{(}\PY{l+s+s2}{\PYZdq{}}\PY{l+s+s2}{Power signal, power:}\PY{l+s+s2}{\PYZdq{}}\PY{p}{,} \PY{n}{Powers}\PY{p}{[}\PY{o}{\PYZhy{}}\PY{l+m+mi}{1}\PY{p}{]}\PY{p}{)}
    \PY{k}{else}\PY{p}{:}
        \PY{n+nb}{print}\PY{p}{(}\PY{l+s+s2}{\PYZdq{}}\PY{l+s+s2}{Neither energy nor power signal}\PY{l+s+s2}{\PYZdq{}}\PY{p}{)}
\end{Verbatim}
\end{tcolorbox}

    

    \section*{Testing against some signals}

    \begin{tcolorbox}[breakable, size=fbox, boxrule=1pt, pad at break*=1mm,colback=cellbackground, colframe=cellborder]
\prompt{In}{incolor}{2}{\boxspacing}
\begin{Verbatim}[commandchars=\\\{\}]
\PY{k+kn}{from} \PY{n+nn}{math} \PY{k+kn}{import} \PY{n}{sin}\PY{p}{,} \PY{n}{e}
\PY{k+kn}{from} \PY{n+nn}{inspect} \PY{k+kn}{import} \PY{n}{getsource}

\PY{k}{def} \PY{n+nf}{f1}\PY{p}{(}\PY{n}{x}\PY{p}{)}\PY{p}{:} \PY{k}{return} \PY{n}{sin}\PY{p}{(}\PY{n}{x}\PY{p}{)} \PY{o}{*} \PY{n}{e}\PY{o}{*}\PY{o}{*}\PY{n}{x}
\PY{k}{def} \PY{n+nf}{f2}\PY{p}{(}\PY{n}{x}\PY{p}{)}\PY{p}{:} \PY{k}{return} \PY{l+m+mi}{1}\PY{o}{/}\PY{n}{x}
\PY{k}{def} \PY{n+nf}{exp}\PY{p}{(}\PY{n}{x}\PY{p}{)}\PY{p}{:} \PY{k}{return} \PY{n}{e}\PY{o}{*}\PY{o}{*}\PY{n}{x}
\PY{k}{def} \PY{n+nf}{unit\PYZus{}step}\PY{p}{(}\PY{n}{x}\PY{p}{)}\PY{p}{:} \PY{k}{return} \PY{l+m+mi}{1} \PY{k}{if} \PY{n}{x} \PY{o}{\PYZgt{}}\PY{o}{=} \PY{l+m+mi}{0} \PY{k}{else} \PY{l+m+mi}{0}
\PY{k}{def} \PY{n+nf}{rect}\PY{p}{(}\PY{n}{x}\PY{p}{)}\PY{p}{:} \PY{k}{return} \PY{l+m+mi}{1} \PY{k}{if} \PY{n}{x}\PY{o}{\PYZgt{}}\PY{o}{=}\PY{l+m+mi}{0} \PY{o+ow}{and} \PY{n}{x} \PY{o}{\PYZlt{}}\PY{o}{=} \PY{l+m+mi}{5} \PY{k}{else} \PY{l+m+mi}{0}
\PY{k}{def} \PY{n+nf}{linear}\PY{p}{(}\PY{n}{x}\PY{p}{)}\PY{p}{:} \PY{k}{return} \PY{n}{x}
\PY{k}{def} \PY{n+nf}{gaussian}\PY{p}{(}\PY{n}{x}\PY{p}{)}\PY{p}{:} \PY{k}{return} \PY{n}{e}\PY{o}{*}\PY{o}{*}\PY{p}{(}\PY{o}{\PYZhy{}}\PY{n}{x}\PY{o}{*}\PY{o}{*}\PY{l+m+mi}{2}\PY{p}{)}
\PY{k}{def} \PY{n+nf}{sigmoid}\PY{p}{(}\PY{n}{x}\PY{p}{)}\PY{p}{:} \PY{k}{return} \PY{l+m+mi}{1}\PY{o}{/}\PY{p}{(}\PY{l+m+mi}{1}\PY{o}{+}\PY{n}{e}\PY{o}{*}\PY{o}{*}\PY{p}{(}\PY{o}{\PYZhy{}}\PY{n}{x}\PY{p}{)}\PY{p}{)}
\PY{k}{def} \PY{n+nf}{sqrt}\PY{p}{(}\PY{n}{x}\PY{p}{)}\PY{p}{:} \PY{k}{return} \PY{n}{x} \PY{o}{*}\PY{o}{*} \PY{l+m+mf}{0.5}
\PY{k}{def} \PY{n+nf}{sine}\PY{p}{(}\PY{n}{x}\PY{p}{)}\PY{p}{:} \PY{k}{return} \PY{n}{sin}\PY{p}{(}\PY{n}{x}\PY{p}{)}


\PY{k}{def} \PY{n+nf}{print\PYZus{}function}\PY{p}{(}\PY{n}{f}\PY{p}{)}\PY{p}{:}
\PY{+w}{    }\PY{l+s+sd}{\PYZdq{}\PYZdq{}\PYZdq{}}
\PY{l+s+sd}{    Prints the expression inside a given function}
\PY{l+s+sd}{    \PYZdq{}\PYZdq{}\PYZdq{}}
    \PY{n+nb}{print}\PY{p}{(}
        \PY{l+s+s2}{\PYZdq{}}\PY{l+s+s2}{f(x) =}\PY{l+s+s2}{\PYZdq{}}\PY{p}{,}
        \PY{n}{getsource}\PY{p}{(}\PY{n}{f}\PY{p}{)}\PY{o}{.}\PY{n}{split}\PY{p}{(}\PY{l+s+s2}{\PYZdq{}}\PY{l+s+s2}{return}\PY{l+s+s2}{\PYZdq{}}\PY{p}{)}\PY{p}{[}\PY{l+m+mi}{1}\PY{p}{]}\PY{o}{.}\PY{n}{strip}\PY{p}{(}\PY{p}{)}
    \PY{p}{)}

\PY{n}{fx} \PY{o}{=} \PY{p}{[}
    \PY{n}{f1}\PY{p}{,}
    \PY{n}{f2}\PY{p}{,}
    \PY{n}{exp}\PY{p}{,}
    \PY{n}{unit\PYZus{}step}\PY{p}{,}
    \PY{n}{linear}\PY{p}{,}
    \PY{n}{sqrt}\PY{p}{,}
    \PY{n}{sine}\PY{p}{,}
    \PY{n}{gaussian}\PY{p}{,}
    \PY{n}{sigmoid}\PY{p}{,}
    \PY{n}{rect}
\PY{p}{]}

\PY{k}{for} \PY{n}{f} \PY{o+ow}{in} \PY{n}{fx}\PY{p}{:}
    \PY{n}{print\PYZus{}function}\PY{p}{(}\PY{n}{f}\PY{p}{)}
    \PY{n}{classify\PYZus{}signal}\PY{p}{(}\PY{n}{f}\PY{p}{)}
    \PY{n+nb}{print}\PY{p}{(}\PY{l+s+s2}{\PYZdq{}}\PY{l+s+se}{\PYZbs{}n}\PY{l+s+s2}{\PYZdq{}}\PY{p}{)}
\end{Verbatim}
\end{tcolorbox}

\section*{Output}

\begin{Verbatim}[commandchars=\\\{\}]
f(x) = sin(x) * e**x
Neither energy nor power signal


f(x) = 1/x
Neither energy nor power signal


f(x) = e**x
Neither energy nor power signal


f(x) = 1 if x >= 0 else 0
Power signal, power: 0.500000074875085


f(x) = x
Neither energy nor power signal


f(x) = x ** 0.5
Neither energy nor power signal


f(x) = sin(x)
Power signal, power: 0.5000001756971644


f(x) = e**(-x**2)
Energy signal, energy: 1.2533141373155003


f(x) = 1/(1+e**(-x))
Neither energy nor power signal


f(x) = 1 if x>=0 and x <= 5 else 0
Energy signal, energy: 5.099999999999998\end{Verbatim}
    
\end{document}
