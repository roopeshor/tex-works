\documentclass[12pt]{article}
\usepackage[margin=2cm]{geometry}
\usepackage{amsmath}
\usepackage{xcolor}
\usepackage{hyperref}
\begin{document}
	\newcommand{\ssq}{\ensuremath{\sqrt{2}}}
	\newcommand{\sect}[1]{\section*{\centering Module #1}
		\vspace*{-.5cm}
		\rule{\textwidth}{.5pt}}
	\title{\vspace*{-1.5cm} DSP Questions}
	\date{}
	\maketitle
\vspace*{-2cm}
\begin{center}
	Compilation of previous year questions and internal papers
\end{center}

\sect{1}
\begin{enumerate}
\item Compare DFT and FFT Algorithms. How many complex additions and multiplications are needed to compute a 64 pt DFT when using DFT equations and FFT? Also comment on the computational advantage of the FFT algorithm over the direct method.
\item Define Discrete Fourier Transform (DFT) \hfill [\textit{2025 sem}]
\item State and prove the time shifting property of DTFT, DFT
\item Find the output of an LTI system if the input $x(n) = \{2,2,1\}$ and the impulse response $h(n) = \{3,2\}$ using DFT
\item Find the four point circular convolution of the sequences $x_1(n)=\{2, 1, 2, 1\}$ and $x_2(n) = \{1, 2, 3, 4\}$.
\item If $x(n)= \{1, 2, 3, 4\}$ find the DFT.
\item How will you obtain linear convolution  from circular convolution? For $x(n)=\{1,2,3\}$ and $h(n)=\{-1,-2\}$, obtain linear convolution $x(n) * h(n)$ using circular convolution.
\item Find the DFT of the sequence $\{1, 2, 3, 4, 4, 3, 2, 1\}$ using DIT FFT algorithm.
\item Find the circular convolution of $x_1(n)=\{1,2,1,3\}$ and $x_2(n)  = \{1,-1,1\}$.
\item State and prove any 3 properties of DFT.
\item Compute the 4 point DFT of $x[n]=\{1,1,0,0\}$.
\item What is inplace computation?
\item Compare DIT and DIF FFT algorithms.
\item Find circular convolution of the sequences $x_1(n) = \{1, 2, 0,-1\}$ and $x_2(n) = \{2,1\}$
\item An 8 point sequence is given by $x(n) = \{1,-1,-1, -1,1,1,1,-1\}$, compute the 8 point DFT of $x(n)$ using radix-2 DIT-FFT.
\item Find the inverse DFT of $X(k) =\{7,-\ssq -j\ssq, -j,\ssq - j\ssq ,1, \ssq +j \ssq , j, -\ssq + j\ssq \}$ using FFT,
\item Using DIT FFT, compute the 8-point DFT of $x(n) = \{1,0,1,0,1,0,1,0\}$. \hfill [\textit{2025 sem}]
\item Find IDFT of sequence $X(k) = \{10, -2 + 2j, -2, -2-2j\}$ using DIT algorithm. \\
$\rightarrow$ \textit{\color{teal}{This can be done by a 4 point IFFT, and you would get $x(n) = \{1,2,3,4\}$}}
\item Find the DTFT of the sequence $h(n) = \{4,2,3,2,4\}$ and plot the magnitude response.
\item The first 5 points of an 8 pt. DFT is given as $X(k) = \{20, -5.828 - j2.414, 0, -0.172-j0.414, 0,\cdots \}$. Find the corresponding $x(n)$. Use DIT algorithm.
\item find DTFT of $x[n] = a^n u(n); |a| < 1$
\item Using the properties of DFT, compute the circular convolution of $x_1(n) = \{1,2,1,2\}$ and $x_2(n) = \{1,2,3,4\}$. \hfill [\textit{2025 sem}]
\item Compute the 4-point DFT of $x(n) = \{1,2,2,1\}$. \hfill [\textit{2025 sem}]
\item \textbf{\textit{Block convolution and DIF-FFT was not asked in any previous year questions. But do study it}}\end{enumerate}

\sect{2}
\begin{enumerate}
\item Define phase delay and group delay of FIR filters.
\item Discuss the finite word length effects in FIR filters.
\item Discuss the effect of coefficient quantization in FIR filters.
\item What are Gibbs Oscillations (Gibb’s phenomenon)? How can they be overcome them?
\item What are the disadvantages of Fourier series method? [\textit{CUCEK 2025 internal}]
\item What are the desirable characteristics of window functions? [\textit{CUCEK 2025 internal}]
\item Realize the system with impulse response $h(n) = \{4,2,3,2,4\}$ in Direct form and Linear Phase form. Calculate the phase delay and group delay.
\item Give the Hamming window function and plot its spectrum.
\item Compare the characteristics of Hamming and Blackman windows.\hfill [\textit{2025 sem}]
\item Give the equations for the N point Hamming and Hanning, Rectangular, Bartlett window functions. Compare them in terms of main lobe width and side lobe level. [\textit{2024 supplimentary exam}]
\item Explain the procedure for designing FIR filters using windows.
\item Explain the frequency sampling method used for the design of FIR filters. Discuss the principle of sampling the desired frequency response and how it determines the filter coefficients. \hfill [\textit{2025 sem}]
\item Briefly explain the different types of windows used in FIR filter design.
\item Design an FIR low-pass filter of length $N=7$ using a Hamming window with cutoff frequency, $\omega_c = 0.4\pi$. \hfill [\textit{2025 sem}]
\item Design an FIR HPF filter using Bartlett window. The cut off frequency $\omega _c=50\pi$. $\omega_s = 200\pi$. Assume $N=9$
\item Realize an FIR filter with $h[n] = \{1,0.5,0.25,0.5,1\}$ using minimum number of multipliers.
\item Determine a direct form realization of the FIR filter with the following filter function using minimum number of multipliers. $h(n) =\{1, 2, 3, 4, 3, 2, 1\}$
\item Get the filter coefficients for the following FIR filter using Fourier series truncation method. Assume $N = 7$.

\begin{equation*}
	H_d(e^{j\omega}) =
	\begin{cases}
		1 &,  0 \le |\omega| < \frac{\pi}{2} \\
		0 &,  \frac{\pi}{2} \le \omega \le \pi
	\end{cases}
\end{equation*}
\item Obtain the linear phase realization of 
$$y(n) = x(n) + 2x(n - 1) - 0.5x(n - 2) + 3x(n - 3) - 0.5x(n-4)+ 2x(n-5)+ x(n-6)$$
\item Obtain the cascade realization with minimum number of multipliers for the system function \hfill [\textit{2025 sem}]
$$H(z) = \left(\frac{1}{2} + z^{-1} + \frac{1}{2}z^{-2}\right)\left(1+\frac{1}{3}z+^{-1}+z^{-2}\right)$$

\item Design a filter with desired frequency response using a Hamming window for N=7.
\begin{equation*}
	H_d(e^{j\omega}) =
	\begin{cases}
		e^{-j3\omega} &,  -\frac{\pi}{4} \le \omega \le \frac{\pi}{4} \\
		0 &,  \frac{\pi}{4} < |\omega| \le \pi
	\end{cases}
\end{equation*}
\item Design an ideal high pass filter with a frequency response
\begin{equation*}
	H_d(e^{j\omega}) =
	\begin{cases}
		1 &,  \textrm{for }\frac{\pi}{4} \le |\omega| \le \pi \\
		0 &,  \textrm{for } 0 \le |\omega| \le \frac{\pi}{4}
	\end{cases}
\end{equation*}
using Hamming window. Find the values of $h(n)$ for $N = 11$. Find $H(z)$. Realize the filter using a suitable method.
\item Using frequency sampling method, designa bandpass filter with the following specifications:
\begin{itemize}
	\item sampling frequency = 8000Hz
	\item cut-off frequencies = 1000Hz, 3000Hz
\end{itemize}
Determine filter coefficient for $N=7$ \hfill [\textit{CUCEK 2025 internal}]

\href{https://drive.google.com/file/d/1OMfmG-yjCWkEvjYwHO8KlNEm-NfIYwSq/view}{\color{blue}{Solution uploaded here}}
\end{enumerate}

\sect{3}
\begin{enumerate}
	\item Describe the characteristics of a Butterworth filter
	\item All stable analog filters are transformed to stable digital filters using Impulse Invariance Technique - Prove. What are the drawbacks?
	\item Write short notes on finite word length effects in IIR digital filters.
	\item What is frequency warping? Discuss how it can be eliminated?
	\item What is the fundamental principle of impulse invariant method? \hfill [\textit{CUCEK 2025 internal}]
	\item What are the properties of the impulse invariant transformation? Explain.\hfill [\textit{2025 sem}]
	\item Compare IIR and FIR filters
	\item Distinguish between Butterworth and Chebyshev filters.
	\item Explain limit cycle behaviors in signal processing, and their types.
	\item  Illustrate the bilinear transformation method of obtaining digital filter from analog filter.
	\item Find the poles of an analog Chebyshev LP filter whose pass band ripple is 2 dB at 20 r/sec and the stop band attenuation is 15 dB at 40 r/s
	\item Design a first order low pass digital Butterworth filter with a cut off frequency of 500 Hz. The sampling frequency is 10 KHz. Use bilinear transformation technique.
	\item Design a digital Butterworth LP filter with the following	specifications, use Impulse invariance transformation.
	$$
	\begin{aligned}
		0.7071  \le & |H(e^{j\omega})| \le 1,	\quad & 0 \le |\omega| \le 0.2\pi \\
		& |H(e^{j\omega})| \le 0.17783, \quad  & 0.4\pi \le |\omega| \le \pi
	\end{aligned}
	$$
	[Analog filter design + transformation to digital (8+2 marks)]
	\item Design a digital low-pass Butterworth filter using the Bilinear Transformation (BLT) method to satisfy the following specifications: \hfill [\textit{2025 sem}]
	$$
	\begin{aligned}
		0.9 \le & |H(e^{j\omega})| \le 1,	\quad & 0 \le |\omega| \le 0.4\pi \\
		& |H(e^{j\omega})| \le 0.15, \quad  & 0.6\pi \le |\omega| \le \pi
	\end{aligned}
	$$
	Take $T = 1$s. Realize the filter.
	
	\item Obtain the canonical form, Cascade and parallel realizations of the IIR filter:
	$$y(n) = -0.1 y(n - 1) + 0.72y(n - 2) + 0.7x(n) - 0.252 x(n - 2)$$
	\item Obtain the direct form I, direct form II and cascade realization for the system having the difference equation:
	$$y(n) + 0.1 y(n-1)- 0.2 y(n- 2) = 3x(n) +3.6x(n-1) + 0.6x(n-2)$$
	\item Draw the Direct Form I, Direct Form II, Parallel, and Cascade realizations of the digital filter with the following difference equation \hfill [\textit{2025 sem}]
	$$y[n]-\frac{3}{4}y[n-1] + \frac{1}{8}y[n-2] = x[n] + \frac{1}{2}x[n-1] + \frac{1}{4}x[n-2]$$
	\item  Realize the given IIR filter in Direct Form I, Direct Form II, cascade and parallel forms:
	$$H(z) = \frac{1-z^{-1}}{\left(1-\dfrac{1}{2}z^{-1}\right)\left(1-\dfrac{1}{4}z^{-1}\right)}$$
\end{enumerate}

\sect{4}
\begin{enumerate}
	\item What is pipelining? What are the different stages in pipelining?
	\item Discuss the special instructions used in DSP processors. [\textit{2023 sem exam}]
	\item Describe Harvard architecture.
	\item Explain the operation of a MAC unit in a DSP processor.
	\item Describe how MAC operations are performed in a single instruction cycle in TMS320C54X. Explain its architecture with diagram.\hfill [\textit{2025 sem}]
	\item List the advantages of floating point processors.
	\item Discuss how, DSP processors are more advantageous than microcontrollers?
	\item Differentiate between fixed and floating point processors.
	\item Discuss the different addressing modes of a TMS320C54X processor with examples.
	\item Draw and explain the architecture of the TMS320C54x processor [\textit{2022 special Supplementary exam}]
	\item Draw and explain architecture of the TMS320C67X \hfill [\textit{2022 special Supplementary exam}]
	\item Explain the different addressing modes in TMS320C67X processors.
\end{enumerate}
\end{document}
