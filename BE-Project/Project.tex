\documentclass[12pt, a4paper]{article}
\usepackage{geometry}
\usepackage[utf8]{inputenc}
\usepackage{amsmath}
\usepackage{amsfonts}
\usepackage{mathrsfs}
\usepackage{fancybox,graphicx}
\usepackage{color}
\usepackage[colorlinks]{hyperref}
\usepackage{accents}
\usepackage{cite}
\usepackage{svg}
\usepackage{fancyhdr}
\usepackage{minted}

\pagestyle{fancy}
\fancyhead[C]{}
\fancyfoot[C]{} % quick fix
\fancyfoot[R]{\thepage}
\addtolength{\headwidth}{\marginparsep}
\addtolength{\headwidth}{\marginparwidth}
\renewcommand{\headrulewidth}{0pt}
\addtolength{\topmargin}{-4.0pt}
\setlength{\headheight}{14.49998pt}

\begin{document}
\begin{titlepage}
	\newgeometry{top=1.7in, bottom=1in, left=1in, right=1in}
	\centering
	{\Huge\bfseries Visualisation of Audio using Dot Matrix Display \par}
	\vspace{1.7cm}
	{\Large \textbf{Submitted by} \par}
	\vspace{1cm}
	{\large Roopesh O R\par}
	{\large Prabhath C S\par}
	{\large Pranav Praveen\par}
	\vspace{2cm}

	\includesvg[inkscapearea=drawing, width=150pt]{cusat.svg}\par
	\vspace{.3cm}
	{\large \bfseries Division of Electronics Engineering \par}
	{\large \bfseries School of Engineering \par}
	{\large \bfseries Cochin University of Science and Technology \par}
	{\large \bfseries Kochi - 682022 \par}
	\vspace{.5cm}
	{\large \bfseries June 2024 \par}

	\vfill

\end{titlepage}

\newgeometry{top=3cm, bottom=3cm, left=2cm, right=2cm}

\newcommand{\usection}[1]{
	\section*{\center \Huge #1}
	\addcontentsline{toc}{section}{\protect\numberline{}#1}
}
\newcommand{\usubsection}[1]{
	\section*{\LARGE #1}
	\addcontentsline{toc}{subsection}{\protect\numberline{}#1}
}

\vspace*{2cm}
\usection{Abstract}
\vspace{.5cm}
From simple music player applications to concerts, audio visualization
enhances the ambience and overall experience of the enjoyer. In this
project we try to replicate a simpler version of such systems using
Arduino UNO development board, 32*8 led dot matrix display, and few small
components. After some preprocessing, an audio signal of our interest is
fed to one of the analogue pin of arduino. This signal is processed 
using "ArduinoFFT" library which splits audio into discrete chunks and
performs FFT on it. Resulting frequency spectrum is then again processed
and scaled down to match the width and height of led display and is
diaplayed on it. In the end, we try to add more customizations such as
different "display modes", etc.


\pagebreak

\usection{Implementation}
\par
\vspace*{1cm}
\usubsection{Introduction}
dfdsfsas

\vspace*{1cm}
\usubsection{Block Diagram}
dggds

\newpage
\usubsection{Arduino UNO}
\vspace{1cm}
\begin{center}
	\includesvg[
	inkscapearea=drawing,
	width=5in,
	angle=-90]{arduino-uno.svg}\par
\end{center}


\vspace*{1cm}
\usubsection{Hardware}



\vspace*{1cm}
\usubsection{Program}

\begin{minted}[]{c}
void setup () {
    pinMode(12, LOW);
}

void loop () {

}
\end{minted}

\vspace*{1cm}
\usubsection{Conclusion}
dfakf

\end{document}