\documentclass[11pt, a4paper]{article}
\usepackage[margin={1.5cm,2cm}]{geometry}
\usepackage{float}
\usepackage{titlesec}
\usepackage{enumitem}
\usepackage{graphicx}
\usepackage{subcaption}
\usepackage{multicol}
\usepackage[T1]{fontenc}

\titleformat*{\section}{\fontsize{14}{14}\selectfont\normalfont\bfseries}
\titleformat*{\subsection}{\fontsize{12}{12}\selectfont\normalfont\bfseries}
\setlist[enumerate]{itemsep=5pt}
\setlength{\columnsep}{1cm}

\renewcommand\thesubsection{\arabic{subsection}.\hspace*{-1em} }
\begin{document}
\pagenumbering{gobble}
\begin{center}
	\fontsize{16}{16}
	\selectfont
	\textbf{Experiment 1 : E-Plane Structure Using HFSS}
\end{center}
\vspace*{.75cm}
\fontfamily{ppl}\selectfont
\begin{multicols}{2}
\section*{Aim}
To design and model an E-plane structure using
ANSYS HFSS by creating and modifying 3D
geometry.
\section*{Software Required}
 ANSYS Electronics Desktop (HFSS)
\section*{Theory}
An E-Plane Tee junction is formed by
attaching a simple waveguide to the broader
dimension of a rectangular waveguide,
which already has two ports. The arms of
rectangular waveguides make two ports
called collinear ports i.e., Port1 and Port2,
while the new one, Port3 is called as Side
arm or E-arm. This E-plane Tee is also
called as Series Tee.\\[0pt]

As the axis of the side arm is parallel to the
electric field, this junction is called E-Plane
Tee junction. This is also called
as Voltage or Series junction. The ports 1
and 2 are 180$^\circ$ out of phase with each
other. The cross sectional view of the E
plane Tee is given in figure below:\\[.5cm]
\includegraphics[width=\linewidth]{1.jpg}

\vspace*{2cm}
\section*{Procedure}
\renewcommand{\labelenumii}{\alph{enumii}.}
\begin{enumerate}[leftmargin=*, itemsep=15pt]
	\item \textbf{Creation of the Base Box}
	\begin{enumerate}
		\item Open \textbf{HFSS} and start a new 3D model.
		\item Go to \textbf{Draw $\rightarrow$ Box} to create a rectangular box.
		\item Enter the following dimensions:
		\begin{itemize}
			\item \textbf{X-dimension}: 22.86 mm
			\item \textbf{Y-dimension}: 10.16 mm
			\item \textbf{Z-dimension}: 60 mm
		\end{itemize}
		\item Specify the position of the box as:
		\begin{itemize}
			\item \textbf{X = -11.43 mm}
			\item \textbf{Y = -5.08 mm}
			\item \textbf{Z = 0 mm}
		\end{itemize}
		\item Click \textbf{OK} to create the box.
		\item Press \textbf{Ctrl + D} to zoom out and view the model properly.
	\end{enumerate}
	\item \textbf{Duplicating the Geometry Around the X-Axis}
	\begin{enumerate}
		\item Right-click the created box.
		\item Select \textbf{Edit $\rightarrow$ Duplicate $\rightarrow$ Around Axis}
		\item In the Duplicate Around Axis window:
		\begin{itemize}
			\item Select \textbf{Axis = X-axis}
			\item Enter \textbf{Rotation Angle = +90$^\circ$}
		\end{itemize}
		\item Click OK to generate the first rotated copy.
		\item Again right-click on the original box.
		\item Select \textbf{Edit $\rightarrow$ Duplicate $\rightarrow$ Around Axis}
		\item In the dialog box:
		\begin{itemize}
			\item Select \textbf{Axis = X-axis}
			\item Enter \textbf{Rotation Angle = -90$^\circ$}
		\end{itemize}
		\item Click \textbf{OK} to generate the second rotated copy.
	\end{enumerate}
	\item \textbf{Uniting the Model}
		\begin{enumerate}
			\item Select \textbf{all three boxes} created.
			\item Right-click on the selection.
			\item Choose \textbf{Edit $\rightarrow$ Boolean $\rightarrow$ Unite} to merge all the shapes into a single solid model.
		\end{enumerate}
	\item \textbf{Assigning Wave Ports}
		\begin{enumerate}
			\item Select \textbf{one face} at the end of the plane (the open end).
			\item Right-click on the selected face and choose: \textbf{Excitations $\rightarrow$ Wave Port $\rightarrow$ Next}
			\item In the dialog box, change \textbf{Name} if required $\rightarrow$ click \textbf{New Line}.
			\item Move the cursor on the face; when the cursor becomes a \textbf{triangle}, click and drag to draw the \textbf{integration line} from the center of the inner line to the top inner line.
			\item Click \textbf{Next $\rightarrow$ Finish} to complete the wave port assignment.
			\item Select the \textbf{other end face} and repeat the same procedure to assign the second wave port.
		\end{enumerate}
	\item \textbf{Assigning Perfect E Boundary}
		\begin{enumerate}
			\item Select \textbf{ all the faces of the plane except the two end faces} where the wave ports were assigned.
			\item Right-click and choose: \textbf{Assign Boundary $\rightarrow$ Perfect E}
			\item Click \textbf{OK}.
		\end{enumerate}
	\item \textbf{Adding Solution Setup}
		\begin{enumerate}
		\item Go to \textbf{Analysis $\rightarrow$ Add Solution Setup}.
			\item Enter the \textbf{Solution Frequency = 10 GHz}.
			\item Set \textbf{Maximum Number of Passes = 6}.
			\item Click \textbf{OK}.
		\end{enumerate}
	\item \textbf{Adding Frequency Sweep}
		\begin{enumerate}
		\item Go to \textbf{Analysis $\rightarrow$ Setup}.
			\item Click on \textbf{Setup 1} $\rightarrow$ choose \textbf{Add Frequency Sweep}.
			\item Enter the sweep limits:
			\begin{itemize}
				\item \textbf{Start Frequency = 5 GHz}
				\item \textbf{Stop Frequency = 15 GHz}
			\end{itemize}
			\item Click \textbf{OK}.
			\item Right-click on \textbf{Analysis} and select \textbf{Analyze All}.
			\item Save the project. The simulation will now run.
		\end{enumerate}
	\item \textbf{Plotting Electric Field Vectors}
		\begin{enumerate}
		\item Select the plane and right-click.
			\item Choose \textbf{Plot Fields $\rightarrow$ E $\rightarrow$ Vector E}.
			\item A dialog box appears $\rightarrow$ click \textbf{Done}.
			\item Electric field vectors will be displayed.
			\item Animate the field distribution if required.
		\end{enumerate}
	\item \textbf{Plotting Magnetic Field (One Side of Plane)}
		\begin{enumerate}
		\item Select one side of the plane and right-click.
			\item Choose \textbf{Plot Fields $\rightarrow$ H $\rightarrow$ Mag H}.
			\item Click \textbf{Animate} to observe the magnetic field variation.
		\end{enumerate}
	\item \textbf{Plotting Magnetic Field on Remaining Faces}
		\begin{enumerate}
		\item Select all the other faces and right-click.
			\item Choose \textbf{Plot Fields $\rightarrow$ H $\rightarrow$ Mag H}.
			\item When the dialog box appears, click \textbf{Done}.
			\item Animate if required.
		\end{enumerate}
\end{enumerate}
\section*{Result}
The E-plane structure was successfully created,
simulated, and field distributions were observed in
HFSS
\end{multicols}
\end{document}