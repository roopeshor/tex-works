\documentclass[11pt, a4paper]{article}
\usepackage[margin={1.8cm,1.8cm}]{geometry}
\usepackage[utf8]{inputenc}
\usepackage{float}
\usepackage{titlesec}
\usepackage{enumitem}
\usepackage{amsmath}
\usepackage{graphicx}
\usepackage{tikz}
\usepackage{subcaption}
\usepackage[T1]{fontenc}
\usepackage{soulutf8} 
\titleformat*{\section}{\fontsize{14}{14}\selectfont\normalfont\bfseries}

\titleformat{\subsection}{\fontsize{12}{12}\selectfont\normalfont\bfseries}{}{0em}{\ul}

\setlist[enumerate]{itemsep=0pt}
\renewcommand\thesubsection{\arabic{subsection}.\hspace*{-1em} }
\begin{document}
\pagenumbering{gobble}
\begin{center}
\fontsize{16}{16}
\selectfont
\textbf{Experiment 6 : TE$_{10}$ mode in Rectangular Waveguide}
\end{center}
\fontfamily{ppl}\selectfont
\section*{Aim}
To determine the frequency \& wavelength in a rectangular waveguide working on TE$_{10}$ mode.
\section*{Instruments/Equipments}
\begin{enumerate}
\item Klystron Power Supply
\item Klystron tube with Klystron mounts
\item Isolator
\item Variable attenuator
\item Frequency meter
\item Slotted section
\item Tunable probe
\item oscilloscope
\item BNC cable
\end{enumerate}

\section*{Theory}
Mode represents in wave guides as either TE$_{mn}$/ TM$_{mn}$, Where
TE-Transverse electric, TM-Transverse magnetic. $m$ - Number of half wavelength variation in
broader direction. $n$ - Number of half wavelength variation in shorter direction.
$$\frac{\lambda_g}{2} = d_1 - d_2$$
Where $d_1$ and $d_2$ are the distance between two successive minima/maxima. It
is having highest cut off frequency hence dominant mode. For dominant TE$_{10}$ mode
in rectangular wave guide $\lambda_0, \lambda_g, \lambda_c$ are related as below.
$$\frac{1}{\lambda_0^2} = \frac{1}{\lambda_g^2} + \frac{1}{\lambda_c^2} $$

Where $\lambda_0$ is free space wavelength, $\lambda_g$ is guide wavelength, $\lambda_c$
is cutoff wavelength. For TE$_{10}$ mode
$$
\lambda_c=\frac{2a}{m}
$$
Where m = 1 in TE$_{10}$ mode and $a$ is inner broad dimension of waveguide.
The wavelength of the signal in an unbounded medium (air or vacuum), calculated
as
$$\lambda_0 = c/f$$
Where $c = 3\times10^8$ m/s is velocity of light and $f$ is frequency.
For propagation to occur, the operating free space wavelength must be less than
the cutoff wavelength ($\lambda_0 < \lambda_c$)

\section*{Procedure}
\begin{enumerate}
\item Set up the components and equipments as shown in figure.
\usetikzlibrary {graphs,shapes.misc}

\begin{tikzpicture}[very thick, nsd/.style={
 rectangle, minimum width=21mm, minimum height=13mm, very thick, draw=black, align=center
}]
\graph [grow right sep=7mm] {
"Klystron \\ Power supply" [nsd, yshift=10mm, xshift=-3mm];
"Klystron \\ Power supply" -- "Klystron \\ Mount"[nsd]
-- Isolator[nsd]
-- "Variable \\ Attenuator"[nsd]
-- "Frequency \\ Meter"[nsd]
-- "Slotted line" [nsd];
"Slotted line" -- "Tunable \\ Probe" [nsd, xshift=114mm, yshift=30mm]
-- DSO[nsd, xshift=114mm, yshift=30mm]
};
\end{tikzpicture}

\item Set Mode selector switch to FM-Mode position with FM amplitude and FM frequency knob at mid
position. Keep beam voltage control knob minimum and reflector voltage knob
to Maximum.
\item Fan should be kept infront of klystron
\item Switch on Fan and the klystron power supply and oscilloscope. Adjust the repeller voltage until a square wave on a DSO is obtained .Record the parameters beam voltage,beam current,repeller voltage correctly using
the Mode Select switch on the Klystron Power Supply
\end{enumerate}

\subsection*{Frequency Measurement}
Direct Method (Frequency Meter)
\begin{enumerate}
	\item Slowly rotate the knob of the Direct Reading Frequency Meter.
\item Observe the DSO square waveform for a sharp dip in the reading, which occurs at resonance.
\item The frequency indicated on the meter dial at this dip is the operating frequency ($f_0$)
\end{enumerate}
\subsection*{Guide Wavelength ($\lambda_g$) Measurement}
This method uses a slotted line terminated with a open, Short Circuit and loaded to produce standing waves.
\begin{enumerate}
	\item Move the control knob of the slotted line slowly from a minimum position. Observe the output on the Oscilloscope and note the distance at the point where a peak is achieved.
\item Record this position as $d_1$
\item Move to the next adjacent minimum position and record it as $d_2$. Repeat this procedure
\item Calculate $\lambda_g$: The distance between two successive minima is half the guide wavelength: \\[-1em]
$$\lambda_g = 2|d_1 - d_2|$$
\end{enumerate}

\subsection*{Indirect Method (Calculation)}
\begin{enumerate}
	\item Measure Waveguide Dimension ($a$): Use the broader inner dimension of the waveguide (typically 2.286cm for X-band).
\item Calculate Cutoff Wavelength ($\lambda_c$): For the dominant
TE$_{10}$ mode, $m=1, \lambda_c = 2a/m$
\item Find Free Space Wavelength ($\lambda_0$): Use the relation: $1/ \lambda_0^2 = 1/ \lambda_g^2 + 1/ \lambda_c^2 $
\item Determine Frequency ($f$): Use $f = c/ \lambda_0$ , where $c \approx 3 \times 10^8$m/s.
\item Finally, compare the calculated frequency from the slotted line method with the direct reading from
the frequency meter to verify accuracy 
$$\frac{\text{Obtained value} (f) - \text{Observed value}(f_0)}{\text{Obtained value} (f)}\times 100$$
\end{enumerate}

\section*{Result Analysis}
\begin{enumerate}
	\item Measure the frequency obtained by frequency meter.
	\item Calculate the guide wavelength as twice the distance between two successive minimum positions obtained as above: $ \lambda_g = 2(d_1-d_2)$
\item Measure the wave-guide inner broad dimension $a$ which will be around 22.86 mm for X band: 
$\lambda_c = 2a$
\item Calculate the frequency by following equation:\\[-.5em]
$$f = \frac{c}{\lambda_0} = c \sqrt{\frac{1}{\lambda_g^2} + \frac{1}{\lambda_c^2}}$$
\item Calculate \% error in frequency in open, shorted and loaded condition
\end{enumerate}

\end{document}
