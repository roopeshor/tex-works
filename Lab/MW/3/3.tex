\documentclass[11pt, a4paper]{article}
\usepackage[margin={1.5cm,2cm}]{geometry}
\usepackage{float}
\usepackage{titlesec}
\usepackage{enumitem}
\usepackage{graphicx}
\usepackage{subcaption}
\usepackage{multicol}
\usepackage[T1]{fontenc}

\titleformat*{\section}{\fontsize{14}{14}\selectfont\normalfont\bfseries}
\titleformat*{\subsection}{\fontsize{12}{12}\selectfont\normalfont\bfseries}
\setlist[enumerate]{itemsep=5pt}
\setlength{\columnsep}{1cm}

\renewcommand\thesubsection{\arabic{subsection}.\hspace*{-1em} }
\begin{document}
\pagenumbering{gobble}
\begin{center}
	\fontsize{16}{16}
	\selectfont
	\textbf{Experiment No 2: H-Plane Structure Using HFSS}
\end{center}
\vspace*{.75cm}
\fontfamily{ppl}\selectfont
\begin{multicols}{2}
\section*{Aim}
To design and model an H-plane structure using
ANSYS HFSS by creating and modifying 3D
geometry
\section*{Software Required}
 ANSYS Electronics Desktop (HFSS)
\section*{Theory}
An H-plane tee junction is formed by attaching a
simply waveguide to a rectangular waveguide
which already has two parts. The arms of
rectangular waveguide make two parts called
collinear parts i.e., part1 and part2 while the new
part is called as side arm or H-arm. This H-plane
tee is also called Shunt tee.As the axis of the side
arm is parallel to the magnetic field, this function
is called H-plane tee junction. This is also called as
current junction or the magnetic field divide itself
into arms.\\[.5cm]
\includegraphics[width=\linewidth]{3.jpg}

\vspace*{2cm}
\section*{Procedure}
\renewcommand{\labelenumii}{\alph{enumii}.}
\begin{enumerate}[leftmargin=*, itemsep=15pt]
	\item \textbf{Create Base Box}
	\begin{enumerate}
		\item Open \textbf{HFSS} and start a new 3D model.
		\item Go to \textbf{Draw $\rightarrow$ Box} to create a rectangular box.
		\item Enter the following dimensions:
		\begin{itemize}
			\item \textbf{X-dimension}: 22.86 mm
			\item \textbf{Y-dimension}: 10.16 mm
			\item \textbf{Z-dimension}: 60 mm
		\end{itemize}
		\item Specify the position of the box as:
		\begin{itemize}
			\item \textbf{X = -11.43 mm}
			\item \textbf{Y = -5.08 mm}
			\item \textbf{Z = 0 mm}
		\end{itemize}
		\item Click \textbf{OK} to create the box.
		\item Press \textbf{Ctrl + D} to zoom out and view the model properly.
	\end{enumerate}
	\item \textbf{Rotate/Duplicate for H-Plane Structure}
	\begin{enumerate}
		\item Right-click the created box $\rightarrow$ \textbf{Edit $\rightarrow$ Duplicate $\rightarrow$ Around Axis}
		\item select Y-axis and set rotation = +90$^\circ$ $\rightarrow$ Click OK.
		\item Repeat: Duplicate around Y-axis with -90$^\circ$ $\rightarrow$ Click OK.
	\end{enumerate}
	\item \textbf{Uniting the Model}
		\begin{enumerate}
			\item Select \textbf{all three boxes} created.
			\item Right-click on the selection.
			\item Choose \textbf{Edit $\rightarrow$ Boolean $\rightarrow$ Unite}
		\end{enumerate}
	\item \textbf{Assigning Wave Ports}
		\begin{enumerate}
			\item Select \textbf{one face} at the open end of the plane.
			\item Right-click on the selected face and choose: \textbf{Excitations $\rightarrow$ Wave Port $\rightarrow$ Next}
			\item In the dialog box, change \textbf{Name} if required $\rightarrow$ click \textbf{New Line}.
			\item Click New Line and drag to draw the \textbf{integration line}.
			\item Click \textbf{Next $\rightarrow$ Finish}.
			\item Repeat to assign the second wave port.
		\end{enumerate}
	\item \textbf{Assigning Perfect E Boundary}
		\begin{enumerate}
			\item Select all the faces of the plane except the two end faces.
			\item Right-click and choose: Assign Boundary $\rightarrow$ Perfect E $\rightarrow$ OK.
		\end{enumerate}
	\item \textbf{Creating Solution Setup}
		\begin{enumerate}
            \item Go to \textbf{Analysis $\rightarrow$ Add Solution Setup}.
			\item set Solution Frequency = 10 GHz.
			\item Set Maximum Number of Passes = 6 $\rightarrow$ OK.
		\end{enumerate}
	\item \textbf{Adding Frequency Sweep}
		\begin{enumerate}
            \item Go to Analysis $\rightarrow$ Setup 1 $\rightarrow$ Add Frequency Sweep.
			\item Start = 5 GHz, Stop = 15 GHz
			\item Click \textbf{OK}.
			\item Right-click Analysis $\rightarrow$ Analyze All.
		\end{enumerate}
	\item \textbf{Plotting E Field Vectors}
		\begin{enumerate}
		\item Select the plane $\rightarrow$ right-click$\rightarrow$ Plot Fields $\rightarrow$ E $\rightarrow$ Vector E.
			\item A dialog box appears $\rightarrow$ Done.
			\item Animate the field distribution if required.
		\end{enumerate}
	\item \textbf{Plotting Magnetic Field (Mag H)}
		\begin{enumerate}
            \item Select one side internal face $\rightarrow$ right-click $\rightarrow$ Plot Fields $\rightarrow$ H $\rightarrow$ Mag H.
			\item Click Animate to observe the magnetic field variation.
		\end{enumerate}
	\item \textbf{Plotting Magnetic Field on Remaining Faces}
		\begin{enumerate}
		\item Select all the other faces and right-click $\rightarrow$ Plot Fields $\rightarrow$ H $\rightarrow$ Mag H}.
        \item Click \textbf{Done}.
		\end{enumerate}
\end{enumerate}
\section*{Result}
The H-plane structure was successfully created,
simulated, and field distributions were observed in
HFSS
\end{multicols}
\end{document}
