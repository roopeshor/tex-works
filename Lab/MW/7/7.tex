\documentclass[11pt, a4paper]{article}
\usepackage[margin={2cm,2cm}]{geometry}
\usepackage{float}
\usepackage{titlesec}
\usepackage{enumitem}
\usepackage{graphicx}
\usepackage{subcaption}
\usepackage{multicol}
\usepackage[T1]{fontenc}

\titleformat*{\section}{\fontsize{14}{14}\selectfont\normalfont\bfseries}
\titleformat*{\subsection}{\fontsize{12}{12}\selectfont\normalfont\bfseries}
\setlist[enumerate]{itemsep=5pt}
\setlength{\columnsep}{1cm}

\renewcommand\thesubsection{\arabic{subsection}.\hspace*{-1em} }
\begin{document}
\pagenumbering{gobble}
\begin{center}
	\fontsize{16}{16}
	\selectfont
	\textbf{Experiment 7 : Dipole Antenna Design Using HFSS}
\end{center}
\raggedcolumns

\vspace*{.75cm}
\fontfamily{ppl}\selectfont

\begin{multicols}{2}
\section*{Aim}
To design and simulate a dipole antenna using ANSYS HFSS and to study its S-parameter characteristics, radiation pattern, gain, and electric field distribution at an operating frequency of 5 GHz
\section*{Software Required}
 ANSYS Electronics Desktop (HFSS)
\section*{Theory}
A dipole antenna is one of the simplest and most 
widely used antennas in wireless communication 
systems. It consists of two equal-length conductive 
arms placed collinearly and fed at the center by a 
balanced source. The most common form is the 
half-wave dipole, whose total length is 
approximately $\lambda/2$, where $\lambda$ is the wavelength of 
operation.

When an alternating current is applied at the feed 
point, currents flow in opposite directions in the 
two arms of the dipole. These time-varying currents 
produce electromagnetic radiation. The maximum 
current occurs at the center of the dipole and
gradually decreases to zero at the ends, while the
voltage is minimum at the center and maximum at
the ends.

The dipole antenna radiates maximum power in the
plane perpendicular to its axis and minimum power
along the axis of the antenna. Hence, its radiation pattern is bidirectional and resembles a figure-of-eight in the E-plane. The radiation is linearly polarized, and the polarization depends on the orientation of the dipole.

\columnbreak
\section*{Procedure}
\begin{enumerate}[leftmargin=*, itemsep=15pt]
% 1
\item \textbf{Open HFSS}
\begin{itemize}
	\item Open ANSYS HFSS
	\item New Project $\rightarrow$ \textbf{Insert HFSS Design}
\end{itemize}

%2
\item \textbf{Draw Dipole (Copper Cylinder)}
\begin{itemize}
	\item Draw $\rightarrow$ \textbf{Cylinder}
	\item Material: \textbf{Copper}
	\item Axis: \textbf{Z}
	\item Radius: \textbf{1}
	\item Height: \textbf{12}
	\item Position: \textbf{(0, 0, 0.5)}
	\item No. of segments: \textbf{0}
	\item Fit All (Ctrl + D)
\end{itemize}

% 3
\item \textbf{Duplicate for Second Arm}
\begin{itemize}
	\item Duplicate the cylinder
	\item Rotate about X-axis = $180^\circ$
	\item Ok
\end{itemize}

%4 
\item \textbf{Dipole Formation}
	\begin{itemize}
		\item Select both cylinders
		\item Ensure \textbf{double arm dipole}
		\item \textbf{Unite}
		\item Change \textbf{color}
		\item Set \textbf{Transparency = 0.6}
	\end{itemize}

% 5
\item \textbf{Feed Creation}
\begin{itemize}
	\item Select \textbf{YZ Plane}
	\item Draw $\rightarrow$ \textbf{Rectangle}
	\item Zoom in and rotate
	\item Rectangle name: \textbf{Rectangle1 (Feed)}
	\textbf{Rectangle Properties:}
	\begin{itemize}
		\item Position: \textbf{(0, -1, -0.5)}
		\item Axis: \textbf{X}
		\item Y-size: \textbf{2}
		\item Z-size: \textbf{1}
	\end{itemize}
\end{itemize}
(feed be placed at the center of the dipole)

%6
\item \textbf{Assign Excitation}
\begin{itemize}
	\item Select \textbf{Rectangle1}
	\item \textbf{Excitations $\rightarrow$ Assign $\rightarrow$ Lumped Port}
	\item Click \textbf{New Line}
	\item Assign reference conductor
	\item OK
\end{itemize}
(Dipole feed gap is very small and not a  transmission line-hence lumped port is used)

%7
\item \textbf{Radiation Box}
\begin{itemize}
	\item Select \textbf{XY Plane}
	\item Create $\rightarrow$ \textbf{Box}
	\item \textbf{Dimensions}
	\begin{itemize}
		\item X = 4
		\item Y = 4
		\item Z = 40
	\end{itemize}
	\item \textbf{Position}: (-20, -20, -20)
	\item Set \textbf{Transparency = 0.8}
	\item Assign \textbf{Radiation Boundary}
	\end{itemize}

%8
\item \textbf{Analysis Setup}
\begin{itemize}
	\item Analysis $\rightarrow$ \textbf{Add Solution Setup}
	\item \textbf{Frequency = 5 GHz}
\end{itemize}

%9
\item \textbf{Frequency Sweep}
\begin{itemize}
	\item Sweep Type: \textbf{Fast}
	\item Start: \textbf{5 GHz}
	\item Stop: \textbf{10 GHz}
	\item Points: \textbf{451}
\end{itemize}

%10
\item \textbf{Far Field Setup}
\begin{itemize}
	\item Results $\rightarrow$ \textbf{Create Far Fields Report}
	\item Select \textbf{Radiation}
	\item Insert \textbf{Far Field Setup}
	\item Magnitude \& Phase
	\item Angular Values:
	\begin{itemize}[leftmargin=*]
		\item Phi $(\phi): -180^\circ$ to $+180^\circ$, Step = $2^\circ$
 		\item Theta $(\theta): -180^\circ$ to $+180^\circ$, Step = $2^\circ$
	\end{itemize}
\end{itemize}

\columnbreak
% 11
\item \textbf{Visibility OFF}
\begin{itemize}
	\item Turn axis / grid \textbf{visibility OFF}
	\item Used for clear plots
\end{itemize}

% 12
\item \textbf{Field Overlay}
\begin{itemize}
	\item Results $\rightarrow$ \textbf{Field Overlay}
	\item Plot Fields $\rightarrow$ \textbf{E}
	\item Shows electric field distribution
\end{itemize}

% 
\item \textbf{Validation \& Analysis}
\begin{itemize}
	\item HFSS $\rightarrow$ \textbf{Validate Design}
	\item Click \textbf{Analyze All}
\end{itemize}

% 
\item \textbf{S-Parameter Plot}
\begin{itemize}
	\item Results $\rightarrow$ \textbf{Create Modal Solution Data Report}
	\item Rectangular Plot
	\item \textbf{S(1,1) in dB}
	\item Add marker if required
\end{itemize}

% 
\item \textbf{3D Far Field Plot}
\begin{itemize}
	\item Results $\rightarrow$ \textbf{Create Far Field Report}
	\item \textbf{3D Polar Plot}
	\item Quantity: \textbf{Gain (Total)}
	\item Units: \textbf{dB}
\end{itemize}

% 
\item \textbf{Radiation Pattern}
\begin{itemize}
	\item Results $\rightarrow$ \textbf{Create Far Field }Report
	\item \textbf{Radiation Pattern}
	\item Gain (Total) in dB
\end{itemize}

% 
\item \textbf{E-Field Animation}
\begin{itemize}
	\item Results $\rightarrow$ \textbf{Animate Fields}
	\item Select \textbf{Mag\_E}
	\item Observe radiation behavior
\end{itemize}

\end{enumerate}
\section*{Result}
A dipole antenna was designed and simulated in HFSS at 5 GHz. The S$_{11}$ plot shows good impedance matching, and the radiation pattern obtained is bidirectional, confirming proper dipole antenna operation.
\end{multicols}
\end{document}