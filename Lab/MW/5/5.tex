\documentclass[11pt, a4paper]{article}
\usepackage[margin={2cm,2cm}]{geometry}
\usepackage[utf8]{inputenc}
\usepackage{float}
\usepackage{titlesec}
\usepackage{enumitem}
\usepackage{graphicx}
\usepackage{subcaption}
\usepackage[T1]{fontenc}
\titleformat*{\section}{\fontsize{14}{14}\selectfont\normalfont\bfseries}
\titleformat*{\subsection}{\fontsize{12}{12}\selectfont\normalfont\bfseries}
\setlist[enumerate]{itemsep=0pt}
\renewcommand\thesubsection{\arabic{subsection}.\hspace*{-1em} }
\begin{document}
\pagenumbering{gobble}
\begin{center}
\fontsize{16}{16}
\selectfont
\textbf{Experiment 5 : Study Of Microwave Components}
\end{center}
\fontfamily{ppl}\selectfont
\section*{Aim}
To sdetermine the frequency \& wavelength in a rectangular waveguide working on TE$_{10}$ mode.
\section*{Instruments/Equipments}
\begin{enumerate}
\item Klystron Power Supply
\item Klystron tube with Klystron mounts
\item Isolator
\item Variable attenuator
\item Frequency meter
\item Slotted section
\item Tunable probe
\item oscilloscope
\item BNC cable
\end{enumerate}

\section*{Theory}
Mode represents in wave guides as either TE$_{mn}$/ TM$_{mn}$, Where
TE-Transverse electric, TM-Transverse magnetic. $m$ - Number of half wave length variation in
broader direction. $n$ - Number of half wave length variation in shorter direction.
$$\frac{\lambda_g}{2} = d_1 - d_2$$
Where $d_1$ and $d_2$ are the distance between two successive minima/maxima. It
is having highest cut off frequency hence dominant mode. For dominant TE$_{10}$ mode
in rectangular wave guide $\lambda_0, \lambda_g, \lambda_c$ are related as below.
$$\frac{1}{\lambda_0^2} = \frac{1}{\lambda_g^2} + \frac{1}{\lambda_c^2} $$

Where $\lambda_0$ is free space wave length,$\lambda_g$ is guide wave length,
$\lambda_c$
is cutoff wave length.
For TE$_{10}$ mode
$$
\lambda_c=\frac{2a}{m}
$$
Where m = 1 in TE$_{10}$ mode and $a$ is inner broad dimension of waveguide.
The wavelength of the signal in an unbounded medium (air or vacuum), calculated
as
$$\lambda_0 = c/f$$
Where $c = 3\times10^8$ m/s is velocity of light and $f$ is frequency.
For propagation to occur, the operating free space wavelength must be less than
the cutoff wavelength ($\lambda_0 < \lambda_c$)

\section*{Procedure}
\begin{enumerate}
\item Set up the components and equipments as shown in figure.
\begin{figure}
\centering
\includegraphics[width=.9\linewidth]{1.png}
\end{figure}

\item Set Mode selector switch to FM-Mode position with FM amplitude and FM frequency knob at mid
position. Keep beam voltage control knob fully anticlockwise(minimum) and reflector voltage knob
to fully clockwise(Maximum).
\item Fan should be kept infront of klystron
\item Switch on Fan
\item Switch On the klystron power supply and oscilloscope. SAdjust the repeller voltage until a square wave on a DSO.Record the parameters beam voltage,beam current,repeller voltage correctly using
the Mode Select switch on the Klystron Power Supply
\end{enumerate}
\end{document}