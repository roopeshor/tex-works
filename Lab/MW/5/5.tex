\documentclass[11pt, a4paper]{article}
\usepackage[vmargin={2cm,4.7cm}, hmargin={2cm,2.4cm}]{geometry}
\usepackage[utf8]{inputenc}
\usepackage{float}
\usepackage{titlesec}
\usepackage{enumitem}
\usepackage{amsmath}
\usepackage{graphicx}
\usepackage{tikz}
\usepackage{subcaption}
\usepackage[T1]{fontenc}
\usepackage{soulutf8} 
\titleformat*{\section}{\fontsize{14}{14}\selectfont\normalfont\bfseries}

\titleformat{\subsection}{\fontsize{12}{12}\selectfont\normalfont\bfseries}{}{0em}{\ul}

\setlist[enumerate]{itemsep=0pt}
\renewcommand\thesubsection{\arabic{subsection}.\hspace*{-1em} }
\begin{document}
\pagenumbering{gobble}
\begin{center}
\fontsize{16}{16}
\selectfont
\textbf{Experiment 5(a) : E-Plane Tee}
\end{center}
\fontfamily{ppl}\selectfont
\section*{Aim}
To study the characteristics of an E-plane tee junction and determine its isolation.
\section*{Apparatus required}
\begin{enumerate}
\item Klystron Power Supply
\item Klystron tube with Klystron mounts
\item Isolator
\item Variable attenuator
\item Frequency meter
\item Detector mount
\item DSO
\item E-plane tee junction
\item Matched load
\item BNC cable
\end{enumerate}

\section*{Theory}
An E-plane tee is a T-shaped waveguide junction in which the side arm is parallel to the electric field (E-field) of the main waveguide.
\begin{itemize}
\item  When power is fed into the side arm, it divides equally between the two collinear arms with $180^\circ$ phase difference.
\item  When equal and opposite waves are applied to the collinear arms, the resultant field at the side arm is zero.
\item Hence, the side arm acts as a difference port.
\end{itemize}

Insertion loss is the reduction in signal power caused by inserting a device (such as a filter, attenuator, or waveguide junction) into a transmission line or microwave system.
When a component is introduced between a source and a load, part of the input power is
\begin{itemize}
\item Absorbed,
\item Reflected, or
\item Dissipated as heat
\end{itemize}
This results in a decrease in the output power, which is termed insertion loss.

$$\text{Insersion Loss(dB)} = 10\log_{10} \left(\frac{P_{\text{in}}}{P_\text{out}}\right) $$

\pagebreak

Since power is proportional to the square of voltage:

$$\text{Insersion Loss(dB)} = 20\log_{10} \left(\frac{V_1}{V_2}\right) $$

Where:
\begin{itemize}
\item $V_1$ = Input voltage
\item $V_2$ = Output voltage
\end{itemize}

\section*{Procedure}
\begin{enumerate}
\item Set up the components and equipments as shown in figure.

\includegraphics[width=\linewidth]{a.pdf}

\item Set Mode selector switch to AM-Mode position with AM amplitude and AM frequency knob at mid
position. Keep beam voltage control knob fully anticlockwise (minimum) and reflector voltage knob
to fully clockwise (maximum).
\item Place the cooling fan in front of the klystron tube and switch ON the fan to avoid overheating
\item Switch 'On' the klystron power supply and oscilloscope. Adjust the repeller voltage until a square wave on a DSO
\item Measure the input voltage by disconnecting the E plane tee and Connect the detector mount directly to the output of the microwave bench. Note the voltage indicated on the DSO. This voltage is taken as the \textbf{input reference voltage $V_1$}
\item Connect the E-plane tee to the setup.
\item Connect the detector mount to one port and terminate the other port with a matched load.
\item Measure the output voltage on the DSO . This voltage is the output voltage $V_2$.and calculate the isolation in dB using 
$$\text{Isolation (dB)} = 20\log_{10} \left(\frac{V_1}{V_2}\right) $$
\item Interchange the position of the matched load and detector mount. Note the change in DSO reading and determine isolation.
\end{enumerate}

\section*{Result}
The characteristics of the E-plane tee were studied and its isolation was determined
\end{document}
