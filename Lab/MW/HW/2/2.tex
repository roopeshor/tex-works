\documentclass[11pt, a4paper]{article}
\usepackage[margin={2cm,2cm}]{geometry}
\usepackage[utf8]{inputenc}
\usepackage{float}
\usepackage{titlesec}
\usepackage{enumitem}
\usepackage{graphicx}
\usepackage{subcaption}
\usepackage[T1]{fontenc}
\titleformat*{\section}{\fontsize{14}{14}\selectfont\normalfont\bfseries}
\titleformat*{\subsection}{\fontsize{12}{12}\selectfont\normalfont\bfseries}
\setlist[enumerate]{itemsep=0pt}
\renewcommand\thesubsection{\arabic{subsection}.\hspace*{-1em} }
\begin{document}
\pagenumbering{gobble}
\begin{center}
	\fontsize{16}{16}
	\selectfont
	\textbf{Experiment 2 : Study Of Microwave Components}
\end{center}
\fontfamily{ppl}\selectfont
\section*{Aim}
To study various microwave components used in X-band waveguide systems and understand their construction,
working principles, and applications.
\section*{Instruments/Equipments}
\begin{enumerate}
	\item Reflex klystron power supply
	\item Gunn power supply
	\item Slotted Line section
	\item Microwave Test bench based on klystron and Gunn diode
\end{enumerate}
\section*{Components}
Reflex Klystron tube, Klystron mount, Isolator , Variable Attenuator, Frequency meter, Detector mount, E plane,
H plane, Magic tee, Directional Coupler, Matched termination, PIN Modulator, Gunn diode, different types of
antennas available.
\section*{Theory}
Microwave components are devices used to generate, guide, control, measure, and radiate microwave energy
(frequency > 1 GHz).In X-band systems (8.2-12.4 GHz), waveguides and solid-state sources like Gunn diodes
or reflex-klystrons are used.
Each microwave component performs a specific function, such as:
\begin{enumerate}
\item Generation (Gunn Oscillator / Klystron)
\item Guiding (Rectangular Waveguide)
\item Power measurement (Detector mounts, Slotted line)
\item Power division (E-plane/H-plane/Magic Tee)
\item Isolation (Circulators, Isolators)
\item Matching (Slide Screw Tuners, Movable short)
\item Radiation (Horn Antenna)
\end{enumerate}
\section*{Components With Explanation}
\subsection{Waveguides}
Waveguides are rectangular or circular shaped metallic tubes used to guide electromagnetic waves. In a
rectangular waveguide, the electric and magnetic fields are distributed in a pattern across the cross-section of the
waveguide. Each waveguide has a definite cut-off frequency below which no energy is transmitted.
The dominant mode of a rectangular waveguide is given by:
$$f_c = \frac{1}{2\sqrt{\mu\epsilon}}
\sqrt{\left(\frac{m}{a}\right)^2 + \left(\frac{n}{b}\right)^2}$$
where $m$ and $n$ correspond to modes and $a$ and $b$ are waveguide dimensions.
\subsection{Waveguide Tees}
Waveguide tees are used to connect waveguides to branch out the microwave signals. Commonly used microwave
junctions are E-plane tee, H-plane tee, magic tee, hybrid ring, directional coupler, and circulator.
\subsection{E-Plane Tee}
E-plane tee is a T-shaped waveguide in which the side arm is parallel to the electric field of the main arm. A
rectangular slot is made on the broader dimension of a waveguide and the side arm is attached as shown in the
figure. Port 1 and port 2 are collinear ports and port 3 is the side arm. E-arm is also called series arm. Equal and
opposite fields at the two ends of the collinear arms are obtained.
\subsection{H-Plane Tee}
The side arm of H-plane tee is parallel to the magnetic field of the main arm. All the three arms are of equal
dimensions. When power is fed through the main arm, it splits into two equal parts at the two side arms. The
waves
 leaving
 the
 two
 side
 arms
 have
 equal
 magnitudes
 and
 are
 in
 phase.
When power is fed through one of the side arms, it divides into two waves entering the main arm and the other
side arm. The outputs will be additive in magnitude and phase.
\subsection{Magic Tee}
A magic tee (hybrid tee) is a combination of E-plane tee and H-plane tee. It has a long waveguide in which two
side arms are attached as shown in the figure. Port 1 and port 2 are collinear arms, port 3 is the H-arm (parallel
arm), and port 4 is the E-arm (series arm).
This four-port hybrid tee junction combines the properties of E-plane and H-plane tees. When power is fed
through port 3 (H-arm), the field divides equally between port 1 and port 2 and the outputs are in phase.
When power is fed through port 4 (E-arm), the field divides equally between port 1 and port 2 but the outputs are
180° out of phase.
If equal amplitude and phase signals are applied at the two collinear arms, the energy appears only at the H-arm,
while no output appears at the E-arm. The energy applied to the parallel arm gets divided equally between port 1
and port 2.
\subsection{Directional Coupler}
A directional coupler is a hybrid waveguide which couples power in an auxiliary waveguide in one direction. It
is a four-port device with one port terminated in a matched load so that there is no power reflected.
Directivity of the directional coupler is defined as the ratio of the power coupled in the auxiliary arm to the power
coupled in the isolated arm. Ideally, directivity is infinite since the power in the uncoupled auxiliary arm is zero.
The coupling is expressed in decibels (dB) as the ratio of power coupled in the auxiliary arm.
\subsection{Slotted Waveguide Section}
The slotted waveguide section is used to study the standing wave in the waveguide. A narrow longitudinal slot
is cut along the centre of the broad wall of the waveguide. The slot provides smooth impedance transformation.
The slot is long enough to cover several wavelengths.
A probe is inserted in the holder and movable carriage on the section. A scale is attached to the slotted section to
measure the distance the probe is moved.
\subsection{Frequency Meter}
A frequency meter is used to measure the frequency of the microwave signal in the waveguide. It consists of a
calibrated tunable cavity. The calibration of frequency is done based on the resonance condition.
Frequency Meter (Resonant Type)
In one type of frequency meter, the frequency can be read directly from the scale. In another type, the frequency
is determined from the chart supplied by the manufacturer.
\subsection{Attenuators}
Attenuators are used to provide suitable power level and to isolate different parts of a microwave circuit. They
are calibrated to read the attenuation level in decibels (dB). A variable attenuator in a microwave test bench is
an important component used to adjust the power level of a signal in a controlled manner. It allows precise control
of signal attenuation, which is critical for testing and calibration in microwave systems.A variable attenuator
typically consists of resistive elements or a combination of passive components that can be adjusted to change
the level of attenuation in the signal path
\subsection{Isolator}
An isolator is a device used to isolate the microwave source from the rest of the circuit. It allows transmission of
waves only in one direction. This prevents reflections from reaching the source and protects the microwave
generator.
\subsection{Circulator}
A circulator is a multi-port junction in which power may flow from one port to the next port only in one direction.
A common application of circulators is in isolation of transmitter and receiver. Four-port circulators are
commonly used.
\subsection{Vswr Meter}
A VSWR meter is an indicating voltmeter calibrated in dB. The square wave modulated output is needed to use
the VSWR meter as the measuring device.
Klystron Power Supply
A klystron power supply is an equipment used to energize the klystron tube. It provides heater voltage and
stabilized bias voltages to the electrodes of the tube. Since the variation in the bias voltage will affect the
frequency of operation.
\subsection{Tunable Probe}
A tunable probe consists of a probe tip, crystal detector, and an adjustable penetration depth mechanism.
It is used to pick up the electric field inside a waveguide and converts the RF (microwave) signal into a DC
voltage, which can be measured using a VSWR meter or microammeter.
\subsection{Detector Mount}
A detector mount consists of a crystal diode placed inside a waveguide along with a tuning plunger.
It is used to detect microwave power by converting the microwave signal into a proportional DC output, which
can be measured using a VSWR meter or microammeter.
\subsection{Klystron/Gunn Oscillator \& Mount}
A klystron tube or Gunn diode is mounted inside a tuned cavity to generate microwave oscillations.
A movable tuning plunger is provided to adjust the operating frequency.
It generates continuous wave (CW) microwave signals and serves as the basic microwave source for X-band
experiments.
\subsection{Matched Termination}
The matched termination is a termination or a load for a microwave setup. Standing waves occur when ever a
load doesn't completely absorb the power reaching it. In the microwave measurements, whether for power or
component characteristics e.g. VSWR of slotted section, it is terminated for minimum reflection. The match
termination serves this purpose. The closed end has a resistive element of suitable electrical and mechanical
characteristics for providing near perfect load.
\subsection{Pyramidal Horn Antenna}
A pyramidal horn antenna is used to convert guided electromagnetic waves from a waveguide into radiated waves
infreespace. It is commonly used for gain measurements, radiation pattern measurements, and other microwave
radiation experiments due to its high directivity and low reflection
\subsection{Gunn Oscillators}
Gunn Oscillators are solid state microwave energy generators. These consist of waveguide cavity flanged on one
end and micrometer driven plunger fitted on the other end. A Gunn-diode is mounted inside the Wave guide with
BNC (F) connector for DC bias. Each Gunn oscillator is supplied with calibration certificate giving frequency vs
micrometer reading.
\subsection{Pin Modulators}
Pin modulators are designed to modulate the continuous wave output of Gunn Oscillators. It is operated by the
square pulses derived from the UHF (F) connector of the Gunn power supply. These consist of a pin diode
mounted inside a section of Wave guide flanged on its both end. A fixed attenuation vane is mounted inside at
the input to protect the oscillator.
\subsection{Gunn Power Supply}
Provides regulated DC bias and modulation pulses for Gunn diode \& PIN modulator.
\vspace*{.5cm}
\section*{Precautions}
Beam voltage should not exceed 250V , Fan should be operated to reduce the heat for reflex
klystron. Carefully follow the specifications of the device and equipment.
\vspace*{.5cm}
\section*{Result}
Various microwave components including waveguides, attenuators, directional couplers, tees, tuners,
terminations,
 horn
 antennas,
 Gunn
 oscillators
 were
 successfully
 studied.

 \begin{figure}[htbp]
	\centering
	\newcommand{\llwd}{0.25\linewidth}
	\setlength{\tabcolsep}{20pt}
	\begin{tabular}{ccc}
\begin{minipage}{\llwd}
\centering
\includegraphics[width=\linewidth]{01.jpg}
Slotted Section
\end{minipage} &

\begin{minipage}{\llwd}
\centering
\includegraphics[width=\linewidth]{02.jpg}
Mounting Base
\end{minipage} &

\begin{minipage}{\llwd}
\centering
\includegraphics[width=\linewidth]{03.jpg}
Matched load
\end{minipage} \\[3.4cm]

\begin{minipage}{\llwd}
\centering
\includegraphics[width=\linewidth]{04.jpg}
Horn antenna
\end{minipage} &

\begin{minipage}{\llwd}
\centering
\includegraphics[width=\linewidth]{05.jpg}
Detector Mount
\end{minipage} &

\begin{minipage}{\llwd}
\centering
\includegraphics[width=\linewidth]{06.jpg}
Klystron Power Supply
\end{minipage} \\[2.6cm]

\begin{minipage}{\llwd}
\centering
\includegraphics[width=\linewidth]{07.jpg}
Microwave test bench \\ -Gunn
\end{minipage} &

\begin{minipage}{\llwd}
\centering
\includegraphics[width=\linewidth]{08.jpg}
Microwave test bench \\ -Klystron
\end{minipage} &

\begin{minipage}{\llwd}
\centering
\includegraphics[width=\linewidth]{09.jpg}
Directional coupler
\end{minipage} \\[4em]

\begin{minipage}{\llwd}
\centering
\includegraphics[width=\linewidth]{10.jpg}
Isolator
\end{minipage} &

\begin{minipage}{\llwd}
\centering
\includegraphics[width=\linewidth]{11.jpg}
E Plane
\end{minipage} &

\begin{minipage}{\llwd}
\centering
\includegraphics[width=\linewidth]{12.jpg}
H Plane
\end{minipage} \\[1em]

\begin{minipage}{\llwd}
\centering
\includegraphics[width=\linewidth]{13.jpg}
Magic Tee
\end{minipage} & &

\end{tabular}

\end{figure}

 \end{document}
