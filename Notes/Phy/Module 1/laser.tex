\topic{Laser}
LASER stands for Light Amplification by Stimulated Emission of Radiation. Differing from conventional sources, lasers produce highly
directional, monochromatic, coherent and stimulated radiation. Lasing has been extended from microwave(Maser) to $\gamma$-rays (Graser)

\subtopic {Characteristics}

\begin{enumerate}
	\item \textbf{Monuchromaticity}: The line width of laser beams are extremely narrow. (1 in $10^{15}$, in conventional light source, it is 1 in $10^6$). Degree of non-monochromaticly: $\xi=\Delta\lambda/\lambda$ = $\Delta\nu/\nu$($\Delta\lambda\text{ or }\Delta\nu$ is variation in wavelength or frequency of light)
	\item \textbf{Directionality}: Laser can travel very long distances without divergence.
	      \eqnleft{\text{Divergence: } \Delta \theta=\frac{r_2-r_1}{D_2-D_1}\text{ or }\frac{D^2}{\lambda}\text{ (Reliegh's range)}}
	      For laser: 0.01 milliradian\\
	      for search light 0.5 radian
	      $>\Rightarrow$ divergence
	\item \term{Coherence}\\[2pt]
	      \textbf{Spatial coherence}: If a wave maintains a constant phase difference or in phase at two different points on
	      the wave over a time $t$, then the wave is said to have spatial coherence.

	      \textbf{Temporal coherence}: If there is no change in phase over a time $t$ at a point on the wave, then it is said to be coherent temporally
	      during that time. (these reference points are taken on electric field)
	\item \textbf{Brightness/intensity}\\
	      Laser produce highly intense beams, because more light energy is concentrated in a small region. Also laser light is coherent, so at a time many photons are in phase. They superimpose to produce a wave of larger amplitude. Hence resultant intensity ($\propto$amplitude$^2$) is very high.
\end{enumerate}

\noparindent

\termlist{
	\term{Coherence length ($L_c$)} \\[2pt]
	Propagation distance over which a coherent wave maintains a specified degree of coherence or phase difference. For He-Ne $L_c=$ 600km
	\eqnleft{\text{Non-chromaticity } \propto \frac{1}{L_c}}

	\term{Coherence time($\tau_c$)} \\[2pt]
	Time over which a propagating wave remains coherent or the maximum time interval for which the wave have definite phase relation. \\
	For He-Ne: $\tau_c=2 \times 10^{-3} \uunit{s}$\\
	Sodium lamp: $10^{-10} \uunit{s}$
}