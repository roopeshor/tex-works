Numerical Aperture is defined as sine of largest angle that a incident ray should have for it to undergo TIR in core

\begin{minipage}[t][][b]{.4\textwidth}%
	By snell's law:

	$n_a\sin (\alpha)=n_1\sin (\alpha)$

	For TIR, angle of incidence	in the medium must be $>\theta_c$.

	$\alpha$ at $\theta_{c}$ is called acceptance angle. $\left(\alpha_{m}\right)$

	\begin{flalign*}
		n_a \sin \left(\alpha_{m}\right) & =n_1 \sin \left(90-\theta_{c}\right)                 &  & \\
		                                 & =n_{1} \cos (\theta)                                 &  & \\
		                                 & \equiv n_{1} \sqrt{1-\sin ^{2}(\theta_c)}            &  & \\
		                                 & =n_{1} \sqrt{1-\left(\frac{n_{2}}{n_{1}}\right)^{2}} &  & \\
		                                 & =\sqrt{n_{1}^{2}-n_{2}^{2}}                          &  &
	\end{flalign*}

	$\boxed{\therefore NA = n_{a} \cdot \sin \left(\alpha_{m}\right) =\sqrt{n_{1}^{2}-n_{2}^{2}}\ }$
\end{minipage}
\hfill%
\begin{minipage}[t][8cm][b]{.6\textwidth}%
	\resizebox{\textwidth}{!}{
		\begin{tikzpicture}[every node/.style={sloped,allow upside down}]
	\coordinate (S) at (2, 0);
	\coordinate (S') at (2, 10);
	\coordinate (O) at (5, 5);
	\coordinate (L) at (0, 5);

	% cable
	\draw[fill=gray!30, fill opacity=.5] (5, 2) rectangle ++(8, 6);
	\draw[fill=blue!50, fill opacity=.3] (5, 3.5) rectangle ++(8, 3);
	
	% cone base
	\filldraw [green!10, fill opacity=.5]
	(S')
	.. controls ++(1.1, -2) and ++(1.1, 2) .. (S)
	.. controls ++(-1.1, 2) and ++(-1.1, -2) ..
	(S');

	% axis
	\draw[dashed, thick] (L) -- (13, 5);
	
	% cone LS
	\filldraw [green!10, fill opacity=.7] (O) -- (S') .. controls ++(1.1, -2) and ++(1.1, 2) .. (S) -- cycle;
	\draw [dashed] (O) -- (S');
	\draw [green, dashed]
	(S')
	.. controls ++(1.1, -2) and ++(1.1, 2) .. (S)
	.. controls ++(-1.1, 2) and ++(-1.1, -2) ..
	(S');

	% ray
	\draw[line width=1pt, red]
	(S) -- node{\midarrow{1pt}{.5}}
	(O) -- node{\midarrow{1pt}{.5}}
	++(3, 1.5) coordinate(A) -- node{\midarrow{1pt}{.7}} (13, 4);

	% labels
	\draw (A) -- ++(0, -1.5) coordinate(P);
	\draw (P) rectangle ++(-.2, .2);
	\pic[blue, draw=blue, thick, angle eccentricity=1.5,angle radius=.5cm, "$\alpha$"] {angle=L--O--S};
	\pic[draw=black, thick, angle eccentricity=1.5,angle radius=.5cm, "$\theta_c$"] {angle=O--A--P};
	\node at (12.5, 7.5) {$n_2$};
	\node at (12.5, 6) {$n_1$};
	\node at (12.5, 8.4) {$n_a$};
	\node at (12, 3.8) {Core};
	\node at (12, 2.2) {Cladding};
	\draw[->] (1, 2) node[align=center, anchor=north east] {Acceptance\\ angle} -- ($(O) - (.9, .6)$);
	\draw[->, green!50!black] (4.5, 9) node[anchor=west] {Acceptance cone} -- (3, 7.7);
\end{tikzpicture}
	}
\end{minipage}
