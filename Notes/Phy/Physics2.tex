\documentclass[12pt, a4paper]{article}

\usepackage[utf8]{inputenc}
\usepackage[a4paper, top=2cm, bottom=3cm, left=2cm, right=2cm]{geometry}
\usepackage[export]{adjustbox}
\usepackage{graphicx}
\graphicspath{ {./images/} }
\usepackage{mathtools}
\usepackage{amsmath}
\usepackage{amsfonts}
\usepackage{amssymb}
\usepackage[version=4]{mhchem}
\usepackage{stmaryrd}
\usepackage[fallback]{xeCJK}
\usepackage{polyglossia}
\usepackage{fontspec}
\usepackage{ucharclasses}
\usepackage{fancyhdr}
\usepackage{wrapfig}
\usepackage{subcaption}
\usepackage{relsize}
\usepackage{framed}
\usepackage{changepage}
\usepackage{tabularray}
\setCJKmainfont{Noto Serif CJK TC}

\setmainlanguage{english}
\setotherlanguages{norwegian, arabic}
\newfontfamily\arabicfont{Noto Naskh Arabic}
% \newfontfamily\lgcfont{CMU Serif}

\pagestyle{fancy}
\fancyhead[C]{}
\fancyfoot[C]{\medskip\thepage}
\renewcommand{\footrulewidth}{.4pt}
\renewcommand{\headrulewidth}{0pt}

\setlength{\headheight}{14.49998pt}
\addtolength{\topmargin}{-2.49998pt}

\newcommand{\figwidth}{8cm}
\newcommand{\floatfigwidth}{5cm}
\newcommand{\uprimary}[1]{
	\section*{\center \Huge \underline{#1}}
	\addcontentsline{toc}{section}{\protect\numberline{}#1}
}
\newcommand{\usecondary}[1]{
	\section*{\center \LARGE \underline{#1}}
	\addcontentsline{toc}{section}{\protect\numberline{}#1}
}
\newcommand{\usection}[1]{
	\section*{\LARGE #1}
	\addcontentsline{toc}{subsection}{\protect\numberline{}#1}
}
\newcommand{\usubsection}[1]{
	\section*{\Large #1}
	\addcontentsline{toc}{subsection}{\protect\numberline{}#1}
}
\newcommand{\ussubsection}[1]{
	\section*{\large #1}
	\addcontentsline{toc}{subsection}{\protect\numberline{}#1}
}
\newcommand{\ans}{\bigskip\underline{\textbf{Answer}}}
\newcommand{\rfloatingimg}[1]{
	\begin{wrapfigure}{r}{\floatfigwidth}
		\includegraphics[max width=\floatfigwidth]{#1}
	\end{wrapfigure}
}
\newcommand\longUparrow{\mathrel{\scalebox{1}[2]{$\uparrow$}}}

\newcommand{\doparindent}{\setlength\parindent{.5cm}}
\newcommand{\noparindent}{\setlength\parindent{0pt}}
\newcommand{\ques}[1]{\noparindent\textbf{#1}\doparindent}

\newcommand{\indentbox}[2]{
	\begin{adjustwidth}{#1}{0pt}
		#2
	\end{adjustwidth}

}
\newcommand{\qa}[3]{
	\noparindent
	\textbf{#1 #2}
	\indentbox{.76cm}{
		\ans
		
		#3
	}
	\bigskip
}
\newcommand{\noskipqa}[2]{
	\noparindent
	\textbf{#1}
	\indentbox{.76cm}{
		\ans
		#2
	}
}

\newcommand{\eqnleft}[1]{
	\begin{flalign*}
		 & #1 &  &
	\end{flalign*}
}
\DeclareRobustCommand{\rchi}{{\mathpalette\irchi\relax}}
\newcommand{\irchi}[2]{\raisebox{\depth}{$#1\chi$}}
\newcommand{\term}[1]{\underline{\textbf{#1}}}
\newcommand{\amstr}{\mathring{\textrm{A}}}
\newcommand{\h}{6.626 \times 10^{-39}}
\newcommand{\lc}{3 \times 10^{8}}
\newcommand{\unit}[1]{\mathrm{~#1}}
\newcommand{\fullwidthimg}[1]{
	\begin{center}
		\includegraphics[max width=\textwidth]{#1}
	\end{center}
}
\newcommand{\uheading}[2]{
	\uprimary{Module - #1}
	\vspace{-.7cm}
	\usecondary{#2}
}

\DefTblrTemplate{caption-tag}{default}{}
\DefTblrTemplate{caption-sep}{default}{}
\DefTblrTemplate{caption-text}{default}{}
\DefTblrTemplate{contfoot-text}{default}{}
\DefTblrTemplate{conthead-text}{default}{}
\begin{document}
\uheading{2}{Crystallography}


\usection{Bragg's X-ray spectrometer}

\fullwidthimg{braggs}

\ussubsection{Components}
\begin{itemize}
	\item Source of X-rays
	\item Crystal held on a circular table which is graduated and provided with vernier.
	\item A detector (Ionisation chamber)
\end{itemize}

\ussubsection{Operation}
\begin{enumerate}
	\item X-ray from X-ray tube, limited by two narrow slits: S$_1$ and S$_2$ are allowed to fall upon crystel.
	\item The crystal is mound on the circular table which can rotate above a vertical und its position can be determined by vernier.
	\item The table is provided with a radial arm which carries an ionisation chamber.
	\item The ionisation chamber is connected to a electrometer to measure the ionisation current. Hence we can measure the intensity of X-ray beam diffracted in the direction of ionisation chamber.
	\item S$_3$ is another slit to limit the width of diffracted beam.
\end{enumerate}

To begin with, the glancing angle $\theta$ for the incident beam is kept very small, the ionisation chamber is adjusted to recieve the reflected beam till the rate of deflection is maximum. The glancing angle and intensity of diffracted beam are detected. The glancing angle is increased in equal steps by rotating the crystal table. The ionization current is noted for different glancing angles. The graph of ionisation current against glancing angle obtained is called X-ray spectrum. The prominent peaks, A$_1$, A$_2$ \& A$_3$ refer to X-rays of wavelength $\lambda$. The glancing angles $\theta_1, \theta_2, \theta_3$ Corresponding to peaks A$_1$, A$_2$ \& A$_3$ which are obtained from the graph. It is found that $\sin \left(\theta_1\right): \sin \left(\theta_2\right): \sin \left(\theta_2\right)$ is $1: 2: 3$ that is A$_1$, A$_2$ \& A$_3$ refer to first, second \& third order reflections of same wavelength, similarly B$_1$, B$_2$ \& B$_3$ are such peaks for 1$^{st}$, 2$^{nd}$ and 3$^{rd}$ order for other wavelength. So Bragg experimentally verified 
$$2 d \sin (\theta) = n \lambda$$


\noparindent
\ques{2. Monochromatic X-rays of wavelength 1.5$\amstr$, are incident on a crystal having interplar ar spacing 1.6$\amstr$. Find highest order for which bragg's reflection maximum can be seen.}

\ans
\begin{flalign*}
	& n\lambda = 2d\sin(\theta) &&\\
	& \therefore n= 2\frac{d}{\lambda}\sin(\theta) = 2 \times \frac{1.6}{1.5} \sin (\theta) &&\\
	& \text {For $n_{max}$}, \theta=90^{\circ} &&\\
	& \therefore n_{\text {max }} = 2 \times \frac{1.6}{1.5} = 2.13 &&
\end{flalign*}

For integer values (n = 2), $\theta=69^{\circ}$


\ques{3. A beam of X-rays of wavelength 0.071n m is diffracted by (110) plane with lattice constant, 0.28 nm. Find glancing angle for $2^{\text {nd }}$ order diffraction}

\ans
\begin{flalign*}
& 2 \lambda = 2 d \sin (\theta), d=0.28 / \sqrt{3}=0.198 \mathrm{nm} &&\\
& \theta = \sin^{-1}(\lambda / d)=\sin^{-1}(0.071 / 0.198) &&\\
& \theta = 21.01^\circ
\end{flalign*}

\ques{4. Calculate glancing angle at (110) plane of a cubic crystal having axial length 0.26 nm corresponding to $2^{\text {nd }}$ order diffraction maxima, for x-ray of wavelength 0.065 nm}

\ans

Axial length = $a$
\begin{flalign*}
2 \lambda &=2d\cdot\sin(\theta) &&\\
\theta &=\sin^{-1}\left(\frac{0.065}{0.185}\right)=20.69^\circ &&
\end{flalign*}

\ques{5. Bragg's angle for reflection from (111) plane in FCC crystal is $19.2^\circ$ for an X-ray of wavelength 1.54 $\amstr$}

\ans
(Assume that $n=1$, if its not given in the question)
\begin{flalign*}
n\cdot\lambda &=2d\sin(\theta) &&\\
1.54 \amstr &=2d\sin\left(19.2^{\circ}\right) &&\\
&\Rightarrow d=2.341&&
\end{flalign*}

$\therefore$ Cube length a=$2.391\sqrt{3}=4.05 \amstr$

\end{document}