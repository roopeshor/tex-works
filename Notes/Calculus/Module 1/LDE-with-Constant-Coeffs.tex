\topic{Linear D.E with Constant Coeffis}

It is eqn of form.

$$
	a_{0} \cdot \frac{d^{n} y}{d x^{n}}+a_{1} \frac{d^{n-1} y}{d x^{n-1}}+\cdots+a_{n-1} \frac{d y}{d x}+a_{n} y=\phi(x)
$$

where $a_{i} \in \mathbb{R}$,

If $\phi(x)=0$ : to solve this we have to change the equation to symbolic form. is $\left(a_{0} D^{n}+D a, D^{n-1}+\cdots\right) s=0 \$$ ItS Auxillary equation is: $\left(a_{\theta} m^{n}+a_{1} m^{n-1}+\cdots\right)$ ky $=0$

From the auxillory equation we get the roots, $m_{1}, m_{2}, \cdots$ Now we procede by following rales. (which depends on nature of roots.

\begin{center}
	\begin{tabular}{|l|l|}
		\hline
		\multicolumn{1}{|c|}{Roots}         & complimentary $f_{x}$                                              \\
		\hline
		1 Roots are Rd equal                & $\left(c_{1}+c_{2} x\right) e^{m_{1} x}$                           \\
		$m_{1}=m_{2}$                       & $c_{1} e^{m_{1} x}+c_{2} e^{m_{2} x}$                              \\
		$m_{1} \neq m_{2}$                  &                                                                    \\
		2) $m_{1}=m_{2}=m_{3}$              & $\left(c_{1}+c_{2} x+c_{3} x^2 \right) e^{m_{1} x}$                \\
		3) $m_{1} \neq m_{2} \neq m_{3}$.   & $\left(c_{1} e^{m_{1} x}+c_{2} e^{m_{2} x}+c_{3} e_{3} x\right.$   \\
		$4) m_{1}=m_{2} \neq m_{3}$         & $\left(c_{1}+c_{2} 2 e^{m_{1} x}+c_{3} e^{m_{3} x}\right.$         \\
		4) $\mathbb{I}: \alpha \pm i \beta$ & $e^{\alpha x}\left(c_{1} \cos(\beta x)+c_{2} \sin(\beta x)\right)$ \\
		\hline
	\end{tabular}
\end{center}

\begin{itemize}
	\item From the nature of roots, we get complimentary function, Hence the Solution is:
\end{itemize}

$$
	y=C \cdot F
$$

\begin{enumerate}
	\item Solve $\frac{d^{2} y}{d x}+\frac{\int}{d x}+6 y=0$
\end{enumerate}

Symbolic form: $\left(D^{2}+5 D+6\right) y=0 \Rightarrow(D+3)(D+2)=0$

$\therefore$ roots are: $m=-3,-2$

Real \& distinct.

$\therefore$ complimentary function is : $c_{1} \cdot e^{m \cdot x}+c_{2} \cdot e^{m_{2} x}$

$$
	=c_{1} e^{-2 x}+c_{2} e^{-3 x}
$$

$\therefore$ Solution is: $\quad y=c_{1} e^{-2 x}+c_{2} e^{-3 x}$

2 Solve $\left(D^{3}+1\right) y=0$

$\rightarrow D^{3}=-1 \quad \Rightarrow \quad$ roots ar: $,-1, \frac{1}{2} \pm \frac{\sqrt{2}}{2} i$

usang: $(a+b)\left(a^{2}-a b+b^{2}\right)$

\begin{flalign*}
	 & \rightarrow(D+1)\left(D^{2}-D+1\right)=0                         \\
	 & \Rightarrow \quad D+1=0 \Rightarrow \text{ root }=-1             \\
	 & D^{2}-D+1=0 \Rightarrow \text{ root }=\frac{1 \pm \sqrt{3} i}{2}
\end{flalign*}

$C F: \quad e^{1 / 2 x}\left(c_{1} \cos \left(\frac{\sqrt{3}}{2} x\right)+c_{2} \sin \left(\frac{\sqrt{3}}{2} x\right)\right)+c_{3} \cdot e^{-x}$\\
To find particular seder integral $\frac{\phi(e)}{5}$

Case

$$
	I: \phi(x)=e^{a x}, \text{ put } D=a
$$

e.g: $\frac{d^{2} y}{d x}-13 \frac{d y}{d x}+12 y=e^{-2 x}$

\begin{flalign*}
	 & \rightarrow \underbrace{\left(D^{2}-12 D+12\right) y}_{=0 \rightarrow \text{ roots }}=e^{-2 x} \\
	 & \therefore C \cdot F=c_{1} e^{x}+c_{2} e^{12 x}
\end{flalign*}

$\therefore$ Porticulor integral $: P I=\frac{e^{-2 x}}{D^{2}-13 D+12} \quad, D=-2$,

$$
	\Rightarrow \frac{e^{-2 x}}{4+26+12}=\frac{e^{-2 x}}{42}
$$

$\therefore$ Solution: $y=C F+P I$

$$
	\begin{gathered}
		\quad=c_{1} e^{x}+c_{2} e^{12 x}+\frac{e^{-2 x}}{42} \\
		6 D^{2} y-D_{y}-2 y=e^{4 x} \quad, \quad 6 \\
		\therefore \text{ Auy-fx }=6 D^{2}-D-2, \text{ rooks }=\frac{+1 \pm \sqrt{1+4 \times 6 \times 2}}{12}=\frac{1 \pm 7}{12} \Rightarrow \frac{2}{3}-\frac{1}{2} \\
		\therefore C F=c_{1} e^{2 / 3 x}+c_{2} e^{1 / 2 x} \\
		P I=\frac{e^{4 x}}{6 D^{2}-D-2}=\frac{e^{4 x}}{6 \times 16-4-2}=\frac{e^{4 x}}{90}
	\end{gathered}
$$

$\therefore$ Solution:

\begin{flalign*}
	 & y=C F+P F                                                           \\
	 & =c_{1} e^{\frac{2}{3} x}+c_{2} e^{\frac{1}{2} x}+\frac{e^{4 x}}{90} \\
	 & y=c_{1}{\sqrt[3]{e^{x}}}^{2}+c_{2} \sqrt{e^{x}}+e^{4 x} / 90
\end{flalign*}

Particular Integral

case $2: \phi(x)=\cos(a x)$ or $\sin(a x)$, put $D^{2}=a-a^{2}$ ?. Solve $\left(0^{2}+4\right) y=\cos(3 x)$

Aux. $f_{x}=D^{2}+4$, roots $= \pm 2 i$

$\therefore C F=$ $e^{o x}\left(c_{1} \cdot \sin(2 x)+c_{2} \cdot \cos(2 x)\right)=c_{1} \cdot \sin(2 x)+c_{2} \cdot \cos(22)$

$P I=\frac{\cos(3 x)}{D^{2}+4}=\frac{\cos(3 x)}{-9+4}=\frac{\cos(3 x)}{-5}$

$\therefore y=C F+P I$

$=c_{1} \sin(2 x)+c_{2} \cdot \cos(2 x)-\frac{\cos(3 x)}{5}$

II? $\left(D^{2}-3 D+2\right) y=\sin(3 x)$

Aux. $f_{x}: D^{2}-3 D+2 \rightarrow$ routs $: 1,2$

$\therefore C F=c_{1} e^{x}+c_{2} e^{2 x}$

$D^{2}=-9$

$P I=\frac{\sin(3 x)}{D^{2}-3 D+2}=$

\begin{flalign*}
	 & =-\frac{\sin(3 x)}{67+3 D}=\frac{-\sin(52)}{3 D+7}                                             \\
	 & =\frac{-\sin(3 x)(30-7)}{9 D^{2}-49}                                                           \\
	 & =\sin(3 x) \cdot(3 D-7)                                                                        \\
	 & =+81+49                                                                                        \\
	 & =\frac{\sin(3 x)(3 D-7)}{130}                                                                  \\
	 & =\frac{1}{130}\left(\frac{3 D \cdot \sin(3 x)}{\frac{d \sin(x)}{d x}}-7 \cdot \sin(3 x)\right) \\
	 & =\frac{1}{130}(9 \cos(3 x)-7 \sin(30))
\end{flalign*}

$\therefore$ Solution: $c_{1} e^{x}+c_{2} e^{2 x}+\frac{1}{130}(9 \cos(3 x)-7 \sin(3 x))$

$\left(D^{2}-2 D-8\right) y=4 \cos(2 x)+e^{4 x}$

Aus. $f_{n}=D^{2}-2 D-8 \Rightarrow(D-4)(D+2) \Rightarrow$ roots $24,-2$,

$\therefore C \cdot F=c_{1} e^{4 x}+c_{2} e^{-2 x}$

\begin{flalign*}
	 & P I_{1}=\frac{4 \cos(2 x)}{D^{2}-2 D-8} \quad D^{2}=-4                              \\
	 & =\frac{4 \cos(2 x)}{-4-2 D-8}=-\frac{4 \cos(2 x)}{-2 D+12}                          \\
	 & \Rightarrow \frac{-2 \cos(2 x)(D-6)}{(D+6)(D-6)}=\frac{-2 \cos(2 x)(D-6)}{D^{2}-36} \\
	 & \begin{aligned}
		    & =\frac{2 \cos(2 x)(D-6)}{40} \\
		    & =\frac{\cos(2 x)(D-6)}{20}
	   \end{aligned}                                                     \\
	 & =\frac{D \cdot \cos(2 x)-6 \cos(2 x)}{20}                                           \\
	 & =-\frac{\sin(2 x)+3 \cos(2 x)}{10}
\end{flalign*}

$$
	\begin{array}{rlrl}
		P I_{2} & =\frac{e^{4 x}}{D^{2}-2 D-8}                                                          & D=4 \quad *: 1 f f(x)=e^{a x}                                                   \\
		        & =\frac{e^{4 x}}{16-8-8} \quad \frac{1}{f(a)}=0, \frac{1}{f(D)} e^{a x}=\frac{x}{\phi}                                                                                   \\
		        & =\frac{e^{4 x}}{0}                                                                    & \frac{1}{0} \quad f(x)=\cos(a x) \& \frac{1}{f\left(\left(^{2}\right)\right.}=0
	\end{array}
$$

$$
	\begin{array}{rlrl}
		P I_{2}     & =\frac{e^{4 x}}{(D-4)(D+2)}                & \text{ * If } f(x)=\sin(a x) d \frac{1}{f\left(D^{2}\right)}=0 \\
		            & =\frac{1}{D-4} \times \frac{e^{4 x}}{D+2}  & \frac{1}{f(D)} \sin(a x)=\frac{-x \cdot \cos(a x)}{29}         \\
		\Rightarrow & =\frac{x e^{4 x}}{4+2}=\frac{x e^{4 x}}{6}
	\end{array}
$$

$\therefore$ Solution is: $C F+P I_{1}+P I_{2}$

$$
	=c_{1} e^{4 x}+c_{2} e^{-2 x}-\frac{\sin(2 x)+3 \cos(2 x)}{10}+\frac{x e^{4 x}}{6}
$$

\begin{enumerate}
	\setcounter{enumi}{1}
	\item $\left(D^{2}-9\right) y=1+5 e^{4 x}+2 e^{3 x}$
\end{enumerate}

A)

\begin{flalign*}
	\text{ Aus } F_{n} & =D^{2}-9=3,-3                                                                                      \\
	C F                & =c_{1} e^{3 x}+c_{2} e^{-3 x}                                                                      \\
	P I_{1}            & =\frac{e^{0 x}}{D^{2}-9}=D^{2}=0                                                                   \\
	                   & =\frac{1}{-9} \quad D=4                                                                            \\
	P I_{2}            & =\frac{5 e^{4 x}}{D^{2}-9} \quad \frac{5 e^{4 x}}{7}=\frac{1}{(D-3)} \cdot \frac{2 e^{3 x}}{(D+3)} \\
	                   & =\frac{2 e^{3 x}}{D^{2}-9}=\frac{2 x e^{3 x}}{6}=\frac{x e^{3 x}}{3}
\end{flalign*}

$\therefore$ Solution: $c_{1} e^{3 x}+c_{2} e^{-3 x}-\frac{1}{9}+\frac{5 e^{4 x}}{7}+\frac{x e^{3 x}}{3}$\\
3. $\left(0^{2}+16\right) y=\cos(4 x)$

Case 3: $\phi(x)=x^{m}$

To find PI. $\frac{1}{f(D)} \phi(x)$, take $[f(D)]^{-1} \phi(x)$

$\rightarrow$ expand binomially, neglecting higher powers of $D$, (upto $m^{\text{th }}$ power)

\begin{flalign*}
	 & (1+x)^{n}=1+n x+\frac{n(n-1)}{2!} x^2 +\frac{n(n-1)(n-2)}{3!} x^{3}+\cdots \\
	 & (1+x)^{-1}=1-x+x^2 -x^{3}+x^{4}+\cdots                                     \\
	 & (1+x)^{-2}=1-2 x+3 x^2 4-4 x^{3}+\cdots
\end{flalign*}

\begin{enumerate}
	\item $\left(D^{2}+D+1\right) y=x^2 $
\end{enumerate}

Aux. $f$ n $=D^{2}+D+1$, roots: $\quad \frac{-1 \pm \sqrt{-3}}{2}=\frac{-1 \pm i \sqrt{3}}{2}$

\begin{flalign*}
	E F=                       & e^{-\frac{1}{2} x}\left(C_{1} \cos(\sqrt{3} x)+C_{2} \sin(x \sqrt{3})\right)                                           \\
	P I=\frac{x^2 }{D^{2}+D+1} & =\left(1+\left(D+D^{2}\right)\right)^{-1} x^2                                                                          \\
	                           & =\left(1-\left(D+D^{2}\right)+\left(D+D^{2}\right)^{2}-\left(D+D^{2}\right)^{3}+\cdots\right) x^2                      \\
	                           & =\left[1-D-D^{2}+D^{2}+2 D^{3}+D^{4}\right] x^2                                                                        \\
	                           & =x^2 -D\left(x^2 \right)-D^{2}\left(x^2 \right)+D^{2}\left(x^2 \right)+2 D^{3}\left(x^2 \right)+D^{4}\left(x^2 \right) \\
	                           & =x^2 -D\left(x^2 \right)+2 D^{3}\left(x^2 \right)+D^{4}\left(x^2 \right)                                               \\
	                           & =x^2 -2 x+0+0=x^2 -2 x
\end{flalign*}

$\therefore$ Solution: $y=c F+P I=e^{-\frac{1}{2} x}\left(c_{1} \cos(x \sqrt{3})+c_{2} \sin(x \sqrt{3})\right)+x^2 \rightarrow-p$

\begin{flalign*}
	 & \left(D^{2}+2 D+1\right) y=2 x+x^2             \\
	 & \quad-1                                        \\
	 & \quad \therefore F=c_{1} e^{-x}+c_{2} e^{-x} x
\end{flalign*}

\begin{flalign*}
	 & P I_{1}=\frac{2 x}{D^{2}+2 D+1}=\left(D^{2}+2 D+1\right)^{-1}(2 x) \\
	 & =(D+1)^{-2}(2 x)                                                   \\
	 & =\left(1-2 D+3 D^{2}\right) 2 x                                    \\
	 & =1-2 D(2 x)+3 D^{2}(2 x)                                           \\
	 & =2 x-4+6=2 x-4                                                     \\
	 & P I_{2}=\frac{x^2 }{(D+1)^{2}}=(D+1)^{-2}\left(x^2 \right)         \\
	 & =\left(1-2 D+3 D^{2}\right) x^2                                    \\
	 & =x^2 -2 D\left(x^2 \right)+3 D^{2}\left(x^2 \right)                \\
	 & =x^2 -4 x+6=x^2 -4 x+6
\end{flalign*}

\begin{flalign*}
	\therefore \text{ Solution }=y & =c_{1} e^{-x}+x^2 \nrightarrow-2 x+2+c_{1} e^{-x} \\
	                               & =e^{-x}\left(c_{1}+c_{2} x\right)+x^2 -2 x+2
\end{flalign*}

$$
	\begin{array}{rlr}
		\left(2 D^{2}-5 D+3\right) y & =\cos(3 x) \cos(2 x)                                                                                                              &              \\
		                             & =\frac{1}{2}(\cos(5 x)-\cos(x))                                                                                                   & C_{H} C_{B}  \\
		2 D^{2}-5 D+3                & =\frac{5 \pm \sqrt{25-24}}{4} \Rightarrow \frac{3}{2}, 1                                                                          & S_{1} S_{2}= \\
		\therefore C F               & =C_{1} e^{3 / 2 x}+C_{2} e^{x}                                                                                                    & S_{1} C_{2}= \\
		P I_{1}                      & =\frac{1}{2} \frac{\cos(5 x)}{2 D^{2}-5 D+3}                                                                                      & C_{1}=       \\
		                             & =\frac{1}{2} \frac{\cos(5 x)}{10-5 D+3}=-\frac{1}{2} \frac{\cos(5 x)}{5 D-13}=\frac{1}{2} \frac{\cos(5 x)(5 D+13)}{25 D^{2}-16 q}
	\end{array}
$$

$$
	C_{A} C_{B}=\frac{1}{2}[C(A+B)+C(A-B)]
$$

\begin{flalign*}
	 & =-\frac{1}{2} \frac{\cos(5 x)(5 D+13)}{125-169}=\frac{\cos(58)(50+13)}{88} \\
	 & =\frac{5 D(\cos(5 x))+13 \cos(5 x)}{88}                                    \\
	 & =\frac{13 \cos(5 x)-25 \sin(5 x)}{88}                                      \\
	 & P I_{2}=\frac{1}{2} \frac{\cos(x)}{2 D^{2}-50+3} \xrightarrow[Z]{ }
\end{flalign*}

$\therefore$ solution: $C_{1} e^{3 / 2 x}+c_{2} e^{x}+\frac{1}{5668}(47 \cos(5 x)+25 \sin(5 x))$

$\left(D^{2}-4 D+3\right) y=\sin(3 x) \cos(2 x)$

A $2^{\text{nd }} O . D E$. has complimentary fo $\$$ particular integral compl.fn is of form: $A e^{m} x+B e^{m_{2} x}+B C e^{m} x \ldots$ where $m_{1}, m_{2}, m_{3}$ are roots of auzillory $f_{n}$. for imaginary roots:

let $a_{5 b i}$ be the root,

$\therefore$ Solution is: $A e^{(a+b i) x}+B e^{(a-b i) x}$

\begin{flalign*}
	 & \Rightarrow A e^{a x} e^{b i x}+B e^{a x} e^{-b i x}       \\
	 & =e^{a x}\left(A e^{b i x}+B e^{-b i x}\right)              \\
	 & =e^{a x}(A(\cos(b x)+i \sin(b x))+B(\cos(b x)-i \sin(b x)) \\
	 & =e^{a x}((A+B)(\cos(b x)+(A-B) i \sin(b x))
\end{flalign*}

\begin{flalign*}
	 & \begin{array}{l}
		   \rightarrow e^{a x}(c \cdot \cos(b x) \\
		   \sin(3 x) \cos(2 x)
	   \end{array}                                                                                  \\
	 & =\frac{1}{2} \sin(5 x)+\frac{1}{2} \sin ^{\sin }(x)                                                                    \\
	 & \text{ Aux. } f_{n}=D^{2}-4 D+3, \text{ roots }=1,3                                                                    \\
	 & =(D-1)(D-3)                                                                                                            \\
	 & \therefore C F=c_{1} e^{x}+c_{2} e^{3 x}                                                                               \\
	 & P I_{1}=\frac{\sin(5 x)}{2\left(D^{2}-4 D+3\right)} \quad, D^{2}=5                                                     \\
	 & \Rightarrow \frac{\sin(5 x)}{13-8 D} \Rightarrow \frac{\sin(5 x)(13+8 D)}{169-64 D^{2}}=-\frac{\sin(5 x)(13+8 D)}{151} \\
	 & =-\frac{13 \sin(5 x)-40 \cos(5 x)}{151}                                                                                \\
	 & P I_{2}=\frac{\sin(x)}{2\left(D^{2}-4 D+3\right)} \Rightarrow C D^{2}=1                                                \\
	 & \Rightarrow \quad \frac{\sin(x)}{5-4 D} \Rightarrow \frac{\sin^{2}(x)(5+4 D)}{25-160^{2}}=\frac{\sin(x)+4 \cos(x)}{9}
\end{flalign*}

$\therefore$ Solution: $y=C F+P I_{1}+P I_{2}$

$$
	=c_{1} e^{x}+c_{2} e^{3 x}-\frac{13 \sin(5 x)+40 \cos(5 x)}{91}+\frac{5 \cos(x)-4 \sin(x)}{9}
$$

Case $\forall$ T $-e^{a x} f(x)$

$\lambda\left(D^{2}+30+2\right) y=e^{2 x} \sin(x)$

C.F in $y=c_{1} e^{-1}+c_{2} e^{-2 x}$

$$
	P . I,=\frac{e^{2 x} \sin(x)}{D^{2}+3 x+2}=e^{2 x} \cdot \frac{\sin(x)}{D^{2}+3 D+2} \stackrel{D=D+a}{\sin(x)} \cdot e^{2 x}
$$

\begin{flalign*}
	 & \frac{\sin(x)}{D^{2}+4 D+4+3 D+6+2} e^{23}                                                                                \\
	 & =\frac{\overline{\sin(x)}}{D^{2}+7 x+12} e^{2 x}                                                                          \\
	 & D^{2}=-(1)                                                                                                                \\
	 & \rightarrow \frac{\sin(x)}{70+11} e^{221}                                                                                 \\
	 & \Rightarrow \frac{\sin(x)(7 D \bar{a} 11)}{\cos D^{2}-121} e^{22}                                                         \\
	 & \Rightarrow \frac{\sin(x)(70-11)}{-170}\left(e^{2 x}\right) \Rightarrow \frac{7 \cos(x)-11 \cos \sin(x)}{9} \cdot e^{2 x}
\end{flalign*}

-Solution.

$$
	y=c_{1} e^{-x}+c_{2} e^{-2 x}+\frac{11 \sin(x)-7 \cos(x)}{170} e^{2 x}
$$

Ho

$$
	\left\{\begin{array}{l}
		\left(D^{2}+4 D+5\right) y=12 e^{-12} \cdot \cos(x) \\
		\left(D^{2}-2 D+1\right) y=x e^{2 x}
	\end{array}\right.
$$

e.

$$
	\therefore C F=c_{1} e^{-x}+c_{2} x e^{x}
$$

\begin{flalign*}
	 & \left.P I=e^{2 x} \cdot \frac{x}{D^{2}-2 D+1}=D^{2}\right\} \\
	 & y e^{2 x}-\left(D^{2}-2 D+1\right) x
\end{flalign*}

\begin{flalign*}
	 & \Rightarrow e^{2 x} \cdot\left(D^{2}-2 D+1\right)^{21} x \\
	 & \Rightarrow e^{2 x} \cdot(D-1)^{-2} \cdot x              \\
	 & \Rightarrow e^{2 x} \cdot\left(1+2 D-3 D^{2}\right) x    \\
	 & \Rightarrow e^{2 x} \cdot(x+2)
\end{flalign*}

\begin{flalign*}
	\Rightarrow & \frac{x}{(D-1+2)^{2}} \cdot e^{2 x}        \\
	\Rightarrow & (D+1)^{-2} x \cdot e^{2 x}                 \\
	            & \left(1+2 D+3 D^{2}\right) x \cdot e^{2 x} \\
	\Rightarrow & (x-2) e^{2 x}
\end{flalign*}

$\therefore$ Solution: $y=c_{1} e^{x}+c_{2} e^{x}+(x-2) e^{2 x}$


\begin{align*}
	 & \left(D^{3}-3 D_{x}^{2}+3 D-3\right) y=x^2 e^{x}                                                                             \\
	 & \Rightarrow(D-1)^{3} \Rightarrow D=1                                                                                         \\
	 & \therefore \quad c=c_{1} e^{x}+c_{2} e^{x} \cdot x+c_{2} x^2 e^{x}                                                           \\
	 & P I=e^{x} \cdot \frac{x^2 }{(Q-1)^{3}}, \quad D \rightarrow D+1                                                              \\
	 & \Rightarrow e^{x} \frac{x^2 }{D^{3}} \Rightarrow e^{x} \cdot\left(D^{-3}\right) x^2                                          \\
	 & e^{x} \cdot\left(D^{-2}\right) \frac{x^{3}}{3}=e^{x} \cdot\left(D^{-1} \cdot \frac{24}{12}\right)=e^{x} \cdot \frac{25}{650} \\
	 & D^{-3}+(D+1-1)^{-3} \rightarrow(1+(D-1))^{3} x^2                                                                             \\
	 & \rightarrow\left(1+3(D-1)+3(D-1)^{2}\right) \gtrless^{2}                                                                     \\
	 & =x^2 +3(D-1) x^2 +3\left(D^{2}-2 D+1\right) x^2                                                                              \\
	 & \Rightarrow x^2 +(3 D-3) x^2 +\left(3 D^{2}-6 D+3\right) x^2                                                                 \\
	 & \Rightarrow x^2 +-3 x^2 +6                                                                                                   \\
	 & x^2 +6 x^{3}+6-12 x  \tag{1}                                                                                                 \\
	 & \left(D^{2}-2 D+1\right) y=x \cdot e^{x} \sin(x)                                                                             \\
	 & C \cdot f=\operatorname{se}\left(c_{1}+c_{2} x\right) e^{x}                                                                  \\
	 & P I=e^{x} \cdot \frac{x \sin(x)}{(D-1)^{2}} \stackrel{D \rightarrow D+1}{2} e^{x} \cdot \frac{x \sin(x)}{D^{2}}              \\
	 & =e^{x} \cdot D^{1}(x \cdot \sin(x))
\end{align*}


\begin{flalign*}
	 & e^{x} \cdot D^{-4}(-x \cdot \cos(x)+\sin(x))                                          \\
	 & e^{x} \cdot(-x \cdot \sin(x)+\cos(x)+\cos(x)                                          \\
	 & \Rightarrow e^{x} \cdot(x \sin(x)+2 \cos(x))                                          \\
	 & \therefore \text{ soln } \quad y=\left(c_{1}+c_{2} x-x \sin(x)-2 \cos(x)\right) e^{x}
\end{flalign*}
