\documentclass[12pt, a4paper]{article}

\usepackage[a4paper, top=2cm, bottom=3cm, left=2cm, right=2cm]{geometry}
\usepackage[export]{adjustbox}
\usepackage{graphicx}
\usepackage{mathtools}
\usepackage{hyperref}
\usepackage{amsmath}
\usepackage{amsfonts}
\usepackage{amssymb}
\usepackage[version=4]{mhchem}
\usepackage{stmaryrd}
\usepackage{polyglossia}
\usepackage{fontspec}
\usepackage{ucharclasses}
\usepackage{fancyhdr}
\usepackage{wrapfig}
\usepackage{subcaption}
\usepackage{relsize}
\usepackage{makecell}
\usepackage{framed}
\usepackage{changepage}
\usepackage{tabularray}
\usepackage{etoolbox}
\usepackage{xstring}
\usepackage{pstricks-add}
\usepackage{tikz}
\usepackage{empheq}
\usepackage[skins]{tcolorbox}
\usepackage[european,s traightvoltages, americanresistor, americaninductors]{circuitikz}
\usepackage{pgfplots}
\usepackage{tikz-3dplot}
\usepackage[T1]{fontenc}
\usepackage[en-GB]{datetime2}
\usepackage{marginnote}
\usepackage{tikzpagenodes}

\usetikzlibrary{
	angles,
	arrows,
	arrows.meta,
	backgrounds,
	calc,
	decorations,
	decorations.markings,
	decorations.pathmorphing,
	fit,
	patterns,
	positioning,
	quotes,
	shapes.arrows,
	shapes.callouts,
	shapes.geometric,
	shapes.misc,
	snakes,
}

\pgfplotsset{compat=1.18}

\hypersetup{colorlinks=true, linkcolor=blue, filecolor=magenta, urlcolor=cyan,}
\urlstyle{same}

\setmainlanguage{english}
\setotherlanguages{norwegian, arabic}
\newfontfamily\arabicfont{Noto Naskh Arabic}
\graphicspath{ {../images/} }

\renewcommand\theadfont{\bfseries}
\DTMlangsetup[en-GB]{abbr}


%%%%%%%%%% Fancy header %%%%%%%%%
\pagestyle{fancy}
\fancyhead[C]{}
\fancyfoot[C]{\medskip\thepage}
\renewcommand{\footrulewidth}{.4pt}
\renewcommand{\headrulewidth}{0pt}
\setlength{\headheight}{14.49998pt}
\addtolength{\topmargin}{-2.49998pt}
\newcommand{\figwidth}{8cm}
\newcommand{\floatfigwidth}{5cm}



%%%%%%%%%% constants/symbols %%%%%%%%%
\newcommand\longUparrow{\mathrel{\scalebox{1}[2]{$\uparrow$}}}
\DeclareRobustCommand{\rchi}{{\mathpalette\irchi\relax}}
\newcommand{\irchi}[2]{\raisebox{\depth}{$#1\chi$}}
\newcommand{\term}[1]{\underline{\textbf{#1}}}
\newcommand{\amstr}{\mathring{\textrm{A}}}
\newcommand{\h}{6.626 \times 10^{-34}}
\newcommand{\kB}{1.38 \times 10^{-23}}
\newcommand{\lc}{3 \times 10^{8}}
\newcommand{\uunit}[1]{\mathrm{~#1}}
\newcommand{\doparindent}{\setlength\parindent{.5cm}}
\newcommand{\noparindent}{\setlength\parindent{0pt}}



\DefTblrTemplate{caption-tag}{default}{}
\DefTblrTemplate{caption-sep}{default}{}
\DefTblrTemplate{caption-text}{default}{}
\DefTblrTemplate{contfoot-text}{default}{}
\DefTblrTemplate{conthead-text}{default}{}

\noparindent
\newcommand{\uprimary}[1]{
	\section*{\center \Huge \underline{#1}}
	\addcontentsline{toc}{section}{\protect\numberline{}#1}
}
\newcommand{\usecondary}[1]{
	\section*{\center \LARGE \underline{#1}}
	\addcontentsline{toc}{section}{\protect\numberline{}#1}
}
\newcommand{\topic}[1]{
	\section*{\LARGE \fontfamily{ppl}\selectfont#1}
	\addcontentsline{toc}{subsection}{\protect\numberline{}#1}
}
\newcommand{\subtopic}[1]{
	\section*{\Large \fontfamily{ppl}\selectfont #1}
	\addcontentsline{toc}{subsection}{\protect\numberline{}#1}
}
\newcommand{\ussubsection}[1]{
	\section*{\large \fontfamily{ppl}\selectfont#1}
	\addcontentsline{toc}{subsection}{\protect\numberline{}#1}
}
\newcommand{\ans}{\bigskip\underline{\textbf{Answer}}}
\newcommand{\ques}[1]{\noparindent\textbf{#1}\doparindent}
\newcommand{\rfloatingimg}[1]{
	\begin{wrapfigure}{r}{\floatfigwidth}
		\includegraphics[max width=\floatfigwidth]{#1}
	\end{wrapfigure}
}
\newcommand{\indentbox}[2]{
	\begin{adjustwidth}{#1}{0pt}
		#2
	\end{adjustwidth}
}
\newcommand{\qa}[3]{
	\noparindent
	\textbf{#1 #2}
	\indentbox{.76cm}{
		\ans
		#3
	}
	\vspace{.75cm}
}
\newcommand{\noskipqa}[2]{
	\noparindent
	\textbf{#1}
	\indentbox{.76cm}{
		\ans
		#2
	}
}
\newcommand{\eqnleft}[1]{
	\begin{flalign*}
		 & #1 &  &
	\end{flalign*}
}
\newcommand{\fullwidthimg}[1]{
	\begin{center}
		\includegraphics[max width=\textwidth]{#1}
	\end{center}
}
\newcommand{\umodule}[2]{
	\uprimary{Module - #1}
	\vspace{-.7cm}
	\usecondary{#2}
}

% {note header}{width}{actual note}
\newcommand{\note}[3]{
	\begin{tcolorbox}[
			width=#2,
			left=5pt,
			colframe=blue!50!black!70!white,
			colback=blue!10,
			title={\textbf{#1}}
		]
		#3
	\end{tcolorbox}
}
% {width}{content}[framecolor][bgcolor]
\NewDocumentCommand{\sidebox}{m m O{orange!80!black!90} O{orange!10}}{
	\begin{tcolorbox}[
			width=#1,
			left=5pt,
			colframe=#3,
			colback=#4
		]
		#2
	\end{tcolorbox}
}

\NewDocumentCommand{\multiskip}{m}{%
	\begingroup
	\newcount\i  % Define a new counter \i
	\i=0         % Initialize the counter
	\loop
	\ifnum\i<#1
	\bigskip  % Add \bigskip
	\advance\i by 1  % Increment the counter
	\repeat
	\endgroup
}


\newcommand{\termlist}[1]{
	\begin{tcolorbox}[
			colback=blue!10!white,
			colframe=blue!50!black,
			title={Some terms}
		]
		#1
	\end{tcolorbox}
}

\newcommand{\chapterheader}[2]{
	\resizebox{\textwidth-3pt}{!}{
		\begin{tikzpicture}[very thick]
			\node [
				draw=black,
				minimum height= 3cm,
				minimum width = \textwidth
			] at (.5\textwidth, 1.5) {
				\scshape
				\parbox{\textwidth}{\fontsize{28pt}{28pt}\selectfont{}\centering #2}
			};
			\node[fill=white, minimum width = 1.3cm] at (.5\textwidth, 3) {\Huge #1};
			\node at (.5\textwidth, -.5) {\tt \footnotesize \DTMnow};
		\end{tikzpicture}
	}
	\vspace*{.2cm}
}

\newcommand{\margindate}[1]{
	\begin{tikzpicture}[remember picture,overlay]
		\node[left=3mm, white, fill=orange!60!black, anchor=east] at (0, .1) {\small \tt #1};
	\end{tikzpicture}%
}
\newcommand{\ruler}{\rule{\textwidth}{1pt}}

\begin{document}

\chapterheader{1}{Analytic functions}

\subtopic{Representation of complex numbers}
\margindate{12 Aug}[1]%
\ssubsection{Cartesian}
\begin{flalign*}
	z & = x + yi                   & \\
	r & = \sqrt{x^2 + y^ 2}  = |z|   \\
\end{flalign*}
\ssubsection{Polar}
if $x = rcos(\theta)$ and $y = rsin(\theta)$
\begin{flalign*}
	z & = x + yi                                                                              & \\
	  & = rcos(\theta) + irsin(\theta)                                                          \\
	  & = re^{i\theta} \hspace*{1cm} [\text{where } \theta = \tan^-1\left(\frac{y}{x}\right)]
\end{flalign*}

\topic{Core ideas}
\subtopic{1. Complex functions and its values}
let $\omega = f(x)$ where $z = x + iy$ is a complex function.
\qa{1}{Find value of $f(z) = z^2 + iz + 2$ at $z = 1 - i$}{
	\begin{flalign*}
		f(1 - i) & = (1 - i)^2 + i(1 - i) + 2 & \\
		         & = 1 - 2i - 1 + i + 1 + 2     \\
		         & = 3 - i
	\end{flalign*}
}

\qa{2}{Find real and imaginary part of $f(z) = \ln(z)$}{
	\begin{flalign*}
		f(z) = \ln(z)                &                                                         & \\
		f\left(r e^{i \theta}\right) & =\ln \left(r e^{i \theta}\right)=\ln r+\ln e^{i \theta}   \\
		                             & =\ln r+i \theta \ln e                                     \\
		                             & =\ln r+i \theta                                           \\
		                             & =\ln \sqrt{x^2+y^2}+i \tan ^{-1}(y / x)
	\end{flalign*}
	Real part = $\ln \sqrt{x^2+y^2}$, imaginary part = $\tan ^{-1}(y / x)$
}

\subtopic{2. Analytic Function (Complex Differentiable Function)}
A complex function $f(z)$ i said to be analytic at the point $z_0$ if $f(3)$ is differentiable at $z_0$ in some neighbourhood of $z_0$.
$f'(z)$ exists at $z$ if
$$\lim_{z \rightarrow z_0} \frac{f(z) - f(z_0)}{z - z_0}$$
exists

A function $f(z)$ is analytic in a domain $\mathcal{D}$ if it is analytic in all points in $\mathcal{D}$

A function $f(z)$ is entire function if it is analytic in every point $z$ in the complex plane.
Eg: $f(z)=e^z$
\vspace*{.5cm}

The function $f(z)=\dfrac{z^2+2}{(z-3)(z+5)}$ fails to be analytical at z=3 and -5.

$\therefore f(z)$ is not an entire function.

\subtopic{3. Singular Points}
A point at which complex function $f(z)$ fails to be analytic is called singular point.

Eg:
\begin{longtblr}{
	colspec={Q[4cm, c]X[l]}
	}
	$f(3)=\dfrac{3^2+2}{(3-3)(3-5)}$ & $z=3,5$ are singular point    \\
	$f^{\prime}(z)=\dfrac{1}{3}$     & $z=0$ is a singular point     \\
	$f(z)=\dfrac{1}{z^2+1}$          & $z=+i,-i$ ane singular points
\end{longtblr}

\pagebreak

\subtopic{4. Canchy - Riemann Equation (CR Equation)}
Used to check whether a complex function $f(x)$ is analytic or not.

If $f(z)=u+i v$ is analytic, then $u$ \& $v$ must satisfy $C-R$ equation:

\begin{center}
	\begin{minipage}[t][][t]{3cm}
		\begin{flalign*}
			U_x & =V_y  & \\
			U_y & =-v_x
		\end{flalign*}
	\end{minipage}%
	\vrule%
	\hspace*{1cm}%
	\begin{minipage}[t][][r]{5cm}
		\begin{flalign*}
			\frac{\partial u}{\partial x} & =\frac{\partial v}{\partial y} & \\
			\frac{\partial u}{\partial y} & =-\frac{\partial v}{d x}         \\
		\end{flalign*}
	\end{minipage}%
\end{center}


\qa{1}{Prove that $f(z)=\overline{z}$ is not analytic}{
	\eqnleft{f(z)=\bar{z}=\overline{x+i y}=x-i y=u+i v}
	\eqnleft{u  =x \quad v=-y          }
	\eqnleft{U_x=\frac{d u}{\partial x}=1 \hspace*{1cm} v_y=\frac{\partial v}{\partial y}=-1}
	because $u_x \neq v_y \quad C-R$ equation is not satisfied $\therefore f(z)=\bar{z}$ not analytic.
}

\qa{2}{P.T $f(z)=z^2$ is analytic. Also find $f^{\prime}(z)$ at $z=1+i$}{
	\begin{flalign*}
		f(z) & =z^2                              \\
		     & =(x+i y)^2                        \\
		     & =x^2+i 2 x y+(i y)^2              \\
		     & =x^2+i 2 x y-y^2 =x^2-y^2+i 2 x y \\
		     & =u+i v
	\end{flalign*}
	\eqnleft{u=x^2-y^2 \hspace*{1cm} v = i2xy}
	$$
		\begin{array}{rlr}
			\frac{\partial u}{\partial x} & =2 x \quad  & \frac{\partial v}{\partial y}=2 x . & U_x=v_y \\
			\frac{\partial u}{\partial y} & =-2 y \quad & u_y=-v_x .
		\end{array}
	$$

	$30 \quad C-R$ equation satisfied. $\therefore$ It is analytic.
	$$
		f^{\prime}(3)=23
	$$
}


% Usually its good to include a brief introduction that spans atleast 3-4 lines.
% One can also write equations that might be generic for the whole right here:
% $$(f*g)(s) = \int_{-\infty}^\infty f(t)g(s-t)dt$$

% You can left align the equation with \verb|\eqnleft{}| without \verb|$$..$$|
% \eqnleft{(f*g)(s) = \int_{-\infty}^\infty f(t)g(s-t)dt}


% \subtopic{Sub Topic 1 - Tables}
% It might be good to use \verb|longtblr| for tables. it might helps save some space
% when there isn't much space for the whole table in the page.
% And it is much easier to work with

% \begin{longtblr}{
% 	colspec = {Q[c, 5cm]X[c]},
% 	hlines,	vlines,
% 	rowsep=5pt,	colsep=5pt,
% 	row{1}={bg=blue!20!white}
% 	}
% 	\thead{Stuff 1} & \thead{Stuff 2} \\
% 	Detail 1        & Detail a        \\
% 	Detail 2        & Detail b        \\
% 	Detail 3        & Detail 4
% \end{longtblr}
% \pagebreak

% \ssubsection{Minpages \& Boxed equations}
% \begin{minipage}[t][][t]{.499\textwidth}%
% 	\centering

% 	\textcolor{blue!30}{\rule{\textwidth}{1cm}}
% 	Minpage 1
% 	\begin{flalign*}
% 		\setlength\fboxsep{1em}
% 		\Aboxed{ f(x)   & = 0 }           & \Aboxed{ g(x) & = b} & \\
% 		\Aboxed{ h(x) } & \Aboxed{ i(x) } &
% 	\end{flalign*}

% 	Some proofs
% \end{minipage}%
% \hfill%
% \vline%
% \begin{minipage}[t][][t]{.5\textwidth}%
% 	\centering
% 	\textcolor{red!30}{\rule{\textwidth}{1cm}}
% 	Minpage 2

% 	\tikz \draw (0, 0) -- (4, 0) -- (4, 4) -- (0, 4) -- cycle;
% \end{minipage}


% \ssubsection{Boxes}

% \textbf{Info Box}: This may contain essential information that could be missed
% out while reading paragraphs of text.


% \note{Note:}{\textwidth}{
% 	\emph{Do not overuse this!}. It is meant for important stuff only.
% }

% % \vspace*{1cm}

% \sidebox{6cm}{
% 	Side boxes are Tik\textit{Z} overlays, and are by default anchored to north east.
% }%
% \begin{minipage}[t][][t]{.6\textwidth}
% 	\textbf{Sidebox}: Side box shows some minor notes, for.eg in equations etc. They should provide useful hints. If and must only be used in cases where adding that extra bit of info breaks the continuity of argument. The thing is its is hard to get it aligned.\\[10pt]
% 	\verb|\sidebox[x][y]{width}{content}[border][fill]|
% \end{minipage}

% \subtopic{Dates}
% \margindate{06 Aug}%
% Always add date to notes! \verb|\margindate{06 Aug}|

% \ruler

\end{document}